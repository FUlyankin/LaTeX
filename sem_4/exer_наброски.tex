%!TEX TS-program = xelatex
\documentclass[12pt, a4paper]{article}  

\usepackage{etex} % расширение классического tex в частности позволяет подгружать гораздо больше пакетов, чем мы и займёмся далее

%%%%%%%%%% Математика %%%%%%%%%%
\usepackage{amsmath,amsfonts,amssymb,amsthm,mathtools} 
%\mathtoolsset{showonlyrefs=true}  % Показывать номера только у тех формул, на которые есть \eqref{} в тексте.
%\usepackage{leqno} % Нумерация формул слева


%%%%%%%%%%%%%%%%%%%%%%%% Шрифты %%%%%%%%%%%%%%%%%%%%%%%%%%%%%%%%%
\usepackage{fontspec}         % пакет для подгрузки шрифтов
\setmainfont{Arial}   % задаёт основной шрифт документа

\defaultfontfeatures{Mapping=tex-text}

% why do we need \newfontfamily:
% http://tex.stackexchange.com/questions/91507/
\newfontfamily{\cyrillicfonttt}{Arial}
\newfontfamily{\cyrillicfont}{Arial}
\newfontfamily{\cyrillicfontsf}{Arial}

\usepackage{unicode-math}     % пакет для установки математического шрифта
\setmathfont{Asana Math}      % шрифт для математики
% \setmathfont[math-style=ISO]{Asana Math}
% Можно делать смену начертания с помощью разных стилей

% Конкретный символ из конкретного шрифта
% \setmathfont[range=\int]{Neo Euler}

\usepackage{polyglossia}      % Пакет, который позволяет подгружать русские буквы
\setdefaultlanguage{russian}  % Основной язык документа
\setotherlanguage{english}    % Второстепенный язык документа


%%%%%%%%%% Работа с картинками %%%%%%%%%
\usepackage{graphicx}                  % Для вставки рисунков
\usepackage{graphics}
\graphicspath{{images/}{pictures/}}    % можно указать папки с картинками
\usepackage{wrapfig}                   % Обтекание рисунков и таблиц текстом


%%%%%%%%%% Работа с таблицами %%%%%%%%%%
\usepackage{tabularx}            % новые типы колонок
\usepackage{tabulary}            % и ещё новые типы колонок
\usepackage{array}               % Дополнительная работа с таблицами
\usepackage{longtable}           % Длинные таблицы
\usepackage{multirow}            % Слияние строк в таблице
\usepackage{float}               % возможность позиционировать объекты в нужном месте
\usepackage{booktabs}            % таблицы как в книгах!
\renewcommand{\arraystretch}{1.3} % больше расстояние между строками

% Заповеди из документации к booktabs:
% 1. Будь проще! Глазам должно быть комфортно
% 2. Не используйте вертикальные линни
% 3. Не используйте двойные линии. Как правило, достаточно трёх горизонтальных линий
% 4. Единицы измерения - в шапку таблицы
% 5. Не сокращайте .1 вместо 0.1
% 6. Повторяющееся значение повторяйте, а не говорите "то же"
% 7. Есть сомнения? Выравнивай по левому краю!





%%%%%%%%%% Графика и рисование %%%%%%%%%%
\usepackage{tikz, pgfplots}  % язык для рисования графики из latex'a
\usepackage{amscd}                  %Пакеты для рисования
\usepackage[matrix,arrow,curve]{xy} %комунитативных диаграмм


%%%%%%%%%% Теоремы %%%%%%%%%%
\theoremstyle{plain}              % Это стиль по умолчанию.  Есть другие стили. 
\newtheorem{theorem}{Теорема}[section]
\newtheorem{result}{Следствие}[theorem]
% счётчик подчиняется теоремному, нумерация идёт по главам согласованно между собой

\theoremstyle{definition}         % убирает курсив и что-то еще наверное делает ;)
\newtheorem*{defin}{Определение}  % нумерация не идёт вообще

\newtheorem{fignia}{Какая-то фигня}


%%%%%%%%%% Свои команды %%%%%%%%%%
\usepackage{etoolbox}    % логические операторы для своих макросов
\usepackage{xparse}      % больше команд для создания команд


\DeclareMathOperator{\Corr}{Corr}
\DeclareMathOperator{\Cov}{Cov}
\DeclareMathOperator{\Var}{Var}


%\renewcommand{\thepage}{
%\ifoddpage
%	\Alph{page}
%\else
%	\Roman{page}
%\fi}

%\renewcommand{\thesection}{\Asbuk{section}}
% При этом начнём нумерацию с буквы А

%\newcounter{iodd}[section]
%\setcounter{iodd}{13}
%\newcounter{ieven}[section]

%\newbool{chap}
%\booltrue{chap}

%\renewcommand\thesection{\ifbool{chap}{\Alph{section} \boolfalse{chap}}{\Roman{section} \booltrue{chap}}} 

%\renewcommand \thesection {\Alph{section}} 



\begin{document} % конец преамбулы, начало документа

\def \s{\ensuremath{\sigma}}

В \s-алгебре лежат какие-то события! 



\newcommand{\xvect}{\ensuremath{x_1,\ldots,x_n}}

\xvect

\newcommand{\xvec}[2]{\ensuremath{x_{#1},\ldots,x_{#2}}}

\xvec{2}{6} 
\xvec{(a,b)}{(c,d)}
\xvec{y_1}{y_2}






\newcommand*{\MyPoint}{\tikz \draw [baseline, fill=blue,draw=blue] circle (2.5pt);}
\renewcommand{\labelitemi}{\MyPoint}

\begin{itemize}
\item Первый пункт
\item Второй пункт
\item Третий пункт
\end{itemize}

\newcommand{\llim}[2]{\lim\limits_{{#1}\to{#2}}}

Внутри текста предел написан  $ \llim{x}{0} \frac{\sin{x}}{x}$, а вне текста не испорчен: 

\[ \llim{x}{0} \frac{\sin{x}}{x}. \]


%\renewcommand{\lim}{\displaystyle \lim}

%Внутри текста предел написан  $ \lim{x}{0} \frac{\sin{x}}{x}$, а вне текста не испорчен: 

%\[ \lim{x}{0} \frac{\sin{x}}{x}. \]


\section{Рисунок} 

\renewcommand{\thefigure}{\thesection:\arabic{figure}}    

\begin{figure}[H]
\center \frame{тут должен быть рисунок} 
\caption{Проба пера}
\end{figure}


\renewcommand{\theequation}{Eq. (\arabic{equation})}

В уравнении \ref{eq:grau}

\begin{equation} 
D=\frac{\rho_{b}}{\rho_{bs}}\times100
\label{eq:grau}
\end{equation}


\newbool{up}
\newbool{down}

\newbool{m}
\newbool{nm}

\newcommand{\unicorn}[3]{%

\booltrue{#1}
\booltrue{#2}

\ifboolexpr{ bool{up} and bool{m}}{
	\reflectbox{\rotatebox{180}{#3}}
	}{
	\ifbool{up}{
		\rotatebox{180}{#3}
		}{\ifbool{m}{
			\reflectbox{#3}
			}{#3}
	} 
}

\boolfalse{#1} 
\boolfalse{#2}
}

\unicorn{up}{m}{fff}
\unicorn{up}{nm}{fff}
\unicorn{down}{nm}{fff}
\unicorn{down}{m}{fff}






\ifnumcomp{3}{>}{6}{true}{false}
\ifnumcomp{(7 + 5) / 2}{=}{6}{true}{false}
\ifnumcomp{(7+5) / 4}{>}{3*(12-10)}{true}{false}
\newcounter{countA}
\setcounter{countA}{6}
\newcounter{countB}
\setcounter{countB}{5}
\ifnumcomp{\value{countA} * \value{countB}/2}{=}{15}{true}{false}
\ifnumcomp{6/2}{=}{5/2}{true}{false}





\newcommand{\dragon}[2][0]{%
   \ifnumcomp{#1}{=}{3}{\reflectbox{\rotatebox{180}{#2}}}{
       \ifnumcomp{#1}{=}{1}{\rotatebox{180}{#2}}{
           \ifnumcomp{#1}{=}{2}{\reflectbox{#2}}{#2}
      }
   }
}

\dragon[1]{kkk} 
\dragon[2]{kkk}
\dragon[3]{kkk}
\dragon[0]{kkk}
\dragon[4]{kkk}
\dragon[333]{kkk}




\ifdimcomp{1cm}{=}{28.45274pt}{true}{false}
\ifdimcomp{(7pt + 5pt) / 2}{<}{2pt}{true}{false}
\ifdimcomp{(3.725pt + 0.025pt) * 2}{<}{7pt}{true}{false}
\newlength{\lengthA}
\setlength{\lengthA}{7.25pt}
\newlength{\lengthB}
\setlength{\lengthB}{4.75pt}
\ifdimcomp{(\lengthA + \lengthB) / 2}{>}{2.75pt * 2}{true}{false}
\ifdimcomp{(\lengthA + \lengthB) / 2}{>}{25pt / 6}{true}{false}



\[
+7=2 \tag{M}
\]


\begin{equation}
+7=2 \tag{M}
\end{equation}


\begin{flushleft}
вывввп
\end{flushleft}







%\newcommand{\stars}[1]{
%\begin{tikzpicture}
%\foreach \x in {1,...,#1}
%  \node [blue] at (\x/4,0) {$\star$};
%\end{tikzpicture}
%}
%
%\stars{2} 
%
%\stars{3}
%
%\stars{4}
%
%\newcounter{item}
%\addtocounter{item}{1}
%
%\stars{\arabic{item}}
%
%\newcounter{jtem}
%\newcommand{superstars}{\addtocounter{jtem}{1}\stars{\arabic{jtem}}}
%
%
%%\superstars
%%\superstars
%
%\stars{\arabic{equation}}
%
%%\renewcommand{\theequation}{\stars{\arabic{equation}}}
%
%В уравнении \ref{eq:grau2}
%
%\begin{equation} 
%D=\frac{\rho_{b}}{\rho_{bs}}\times100
%\label{eq:grau2}
%\end{equation}
%
%\stars{\arabic{equation}}
%
%
%\newpage
%
%\newcounter{matriz}
%\newenvironment{matriz}{\refstepcounter{matriz}\equation}{\tag{M\thematriz}\endequation}
%
%The of next matriz equation is 1. \eqref{test}
%\begin{matriz}\label{test}
%y = \sqrt x
%\end{matriz}
%
%The of next matriz equation is 1. \eqref{test}
%\begin{matriz}\label{test}
%y = \sqrt x
%\end{matriz}



%\newcounter{ltem}
%\newenvironment{starseq}{\refstepcounter{ltem}\equation}{\tag{\stars{\arabic{ltem}}}\endequation} 
%
%\begin{starseq}
%D=\frac{\rho_{b}}{\rho_{bs}}\times100
%\label{eqstar:1}
%\end{starseq}
%
%В уравнении \eqref{eqstar:1}
%
%\begin{starseq}
%D=\frac{\rho_{b}}{\rho_{bs}}\times100
%\label{eqstar:2}
%\end{starseq}

\newpage

f

\newpage

f

\end{document}



\begin{problem}
Сделать так, чтобы чётные главы выводились римскими цифрами, а нечётные русскими буквами.
\begin{sol}
\end{sol}
\end{problem}







