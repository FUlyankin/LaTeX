\documentclass[12pt, a4paper]{article}

%         Классы: 
% article   ---   статья
% report    ---   отчет
% book      ---   книга
% beamer    ---   презентация


% В преамбуле находятся различные служебные команды. А именно:
% а) Команды, подключающие пакеты
% б) Команды, которые определяют вид документа в целом
% в) Команды, которые создают новые команды, чтобы удобнее использовать старые команды!
% г) Ещё какие-нибудь другие команды

\usepackage[british,russian]{babel} % выбор языка для документа
\usepackage[utf8]{inputenc} % задание utf8 кодировки исходного tex файла
\usepackage[X2,T2A]{fontenc} % кодировка

\usepackage{amsmath,amsfonts,amssymb,amsthm,mathtools}  %Пакеты для математики!


\begin{document}


% Элементы структуры:
% part -> chapter -> section -> subsection -> subsubsection -> paragraph -> subparagraph 

chapter есть в классах book и report

\tableofcontents

% Файл будет создаваться интерактивно от Привет, Мир! До конца! 

\section{Приветсивие миру}
Привет, мир! 

\section{Команды}

Хэй, чувак сделай ка мне слово \textbf{дождь} жирным, а слово \textit{причиной} курсивным!

\subsection{Эксперименты!}

Володя любит          Аню, а Аня любит   кушать мороженое! 

Если случилось так, что один Братан пообещал (навсегда) место на переднем сиденье своей машины одновременно двум своим Братанам, то Второй Пилот определяется следующими способами:


% Здесь будет расположен список того как определяется второй пилот!
% Надо бы добавить ещё пару способов.

\begin{enumerate}
\item забег до машины
\item аукцион; а в случае если поездка превышает 700 км ---
\item бой без правил насмерть.  % этот способ не очень хороший
\end{enumerate}

Кроме того можно определить первого пилота с помощью скоростного интегрирования. Кто первым возьмёт интеграл 

\[ \int_{-\infty}^{+\infty} e^{-x^2} dx \] 

тот и победит. Будет позорно забыть, что это $\sqrt{\pi}$!



\end{document}









