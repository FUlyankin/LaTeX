\documentclass[12pt, a4paper]{article}  

%         Классы: 
% article   ---   статья
% report    ---   отчет
% book      ---   книга
% beamer    ---   презентация

\usepackage{amsmath,amsfonts,amssymb,amsthm,mathtools}  % пакеты для математики

%%%%%%%%%%%%%%%%%%%%%%%% Шрифты %%%%%%%%%%%%%%%%%%%%%%%%%%%%%%%%%

\usepackage{fontspec}         % пакет для подгрузки шрифтов
\setmainfont{Roboto}       % задаёт основной шрифт документа

\usepackage{unicode-math}     % пакет для установки математического шрифта
\setmathfont{Asana Math}      % шрифт для математики

\usepackage{polyglossia}      % Пакет, который позволяет подгружать русские буквы
\setdefaultlanguage{russian}  % Основной язык документа
\setotherlanguage{english}    % Второстепенный язык документа

\begin{document} % тут заканчивается преамбула и начинается документ


% Элементы структуры:
% part -> chapter -> section -> subsection -> subsubsection -> paragraph -> subparagraph 
% chapter есть в классах book и report

\tableofcontents

\section{Приветсивие миру}
Привет, мир! 

\section{Команды}

Хэй, чувак сделай ка мне слово \textbf{дождь} жирным, а слово \textit{причиной} курсивным!

\subsection{Эксперименты!}

Володя любит          Аню, а Аня любит   кушать мороженое! 

Если случилось так, что один Братан пообещал (навсегда) место на переднем сиденье своей машины одновременно двум своим Братанам, то Второй Пилот определяется следующими способами:


% Здесь расположен список того как определяется второй пилот!
% Надо бы добавить ещё пару способов.

\begin{enumerate}
\item забег до машины
\item аукцион; а в случае если поездка превышает 700 км ---
\item бой без правил насмерть.  % этот способ не очень хороший
\end{enumerate}



% Плевать на колчество пробелов и пропущенных строк!



Кроме того можно определить первого пилота с помощью скоростного интегрирования. Кто первым возьмёт интеграл 

\[ \int_{-\infty}^{+\infty} e^{-x^2} dx \] 

тот и победит. Будет позорно забыть, что это $\sqrt{\pi}$!

\end{document}









