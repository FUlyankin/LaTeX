%!TEX TS-program = xelatex
\documentclass[12pt, a4paper]{article}

\input{preamble}


% Специальный пакет для оформления задач! 
\newtheorem{problem}{Задача}

\usepackage{answers}
\Newassociation{sol}{solution}{solution_file}
% sol --- имя окружения внутри задач
% solution --- имя окружения внутри solution_file
% solution_file --- имя файла в который будет идти запись решений


\begin{document}

% Открываем файл, куда будут записываться решения. 
\Opensolutionfile{solution_file}[all_solutions]


\begin{problem}
В разложении \[\left(\sqrt{x} + \frac{1}{\sqrt[3]{2}}\right)^n\] коэффициент пятого члена относится к коэффициенту третьего члена, как 7 к 2. Найти коэффициент перед $x$ в первой степени.
\begin{sol}

\end{sol}
\end{problem}



\begin{problem}
Сколько рациональных членов содержится в разложении \[(\sqrt{2} + \sqrt[3]{3})^{100} \]
\begin{sol}

\end{sol}
\end{problem}



\begin{problem}
Дан многочлен \[x (2-3x)^5 + x^3 (1 + 2x^2)^7 - x^4 (3 + 2x^3)^9.\] Найти коэффициент члена, содержащего $x^5$, не раскрывая скобок.
\begin{sol}

\end{sol}
\end{problem}



\begin{problem}
Найти все рациональные члены разложения \[\left( \sqrt[3]{2} - \frac{1}{\sqrt{2}} \right),\] не выписывая иррациональные.
\begin{sol}

\end{sol}
\end{problem}



\begin{problem}
В разложении \[\left( x \sqrt{x} - \frac{1}{x^4} \right)^n\] биномиальный коэффициент третьего члена на 44 больше коэффициента второго члена. Найти свободный член.
\begin{sol}

\end{sol}
\end{problem}



\begin{problem}
Найти член разложения \[\left( \frac{b \sqrt{b}}{a} - \frac{a}{\sqrt[3]{b}} \right)^{14},\] содержащий $b^{10}$.
\begin{sol}

\end{sol}
\end{problem}



\begin{problem}
Найти следующие суммы
\begin{itemize}
\item $C_n^0 + C_n^1 + C_n^2 + \ldots + C_n^n$
\item $1 - C_n^1 + C_n^2 - C_n^3 + \ldots + (-1)^n C_n^n$
\item $C_n^1 + C_n^3 + C_n^5 + \ldots + C_n^{n-1}$, где $n$ --- чётное.
\item $C_n^0 + 2C_n^1 + 2^2 C_n^2 + \ldots + 2^n C_n^n$
\item $C_n^1 + 2C_n^2 + 3C_n^3 + \ldots + n C_n^n$
\end{itemize}
\begin{sol}

\end{sol}
\end{problem}



\begin{problem}
На дереве висит 10 различных яблок. Сколькими способами можно сорвать нечётное количество яблок?
\begin{sol}

\end{sol}
\end{problem}



\begin{problem}
Пусть $A = \{a_1,a_2, \ldots a_6\}$. Чему равна мощность списка всех подмножеств множества $A$?  Сколькими способами можно 
выбрать подмножество из чётного числа элементов?
\begin{sol}

\end{sol}
\end{problem}



\begin{problem}
Верно ли, что $(1 + x)^n + (1 - x)^n \le 2^n$ при $ n \ge 2$ и $|x| \le 1$ 
\begin{sol}

\end{sol}
\end{problem}




% Закрываем файл, куда мы записывали решения и вставляем его в конце списка задач. 
\Closesolutionfile{solution_file}

% Вставляем решения. Можно их не вставлять или настроить пакет так, чтобы они шли непосредственно после каждой задачи.
% \begin{solution}{1}
\end{solution}
\begin{solution}{3}
\end{solution}
\begin{solution}{4}
\end{solution}



\end{document}
