\documentclass[12pt, a4paper]{article} 
\usepackage{etex}
\usepackage{fontspec}
\usepackage{polyglossia}
\setdefaultlanguage{russian}
\setotherlanguage{english}
\newfontfamily\cyrillicfont{Phorssa}
 
\usepackage{hyperref}
\usepackage{amsmath,amsfonts,amssymb,amsthm,mathtools}  
 
\usepackage{graphicx}
\usepackage{unicode-math}     % пакет для установки математического шрифта
 
\usepackage{graphicx}                  % Для вставки рисунков
\usepackage{graphics} 
\graphicspath{{images/}{pictures/}}    % можно указать папки с картинками
\usepackage{wrapfig}                   % Обтекание рисунков и таблиц текстом
\usepackage{subfigure} 
\setmathfont{Phorssa}
 
\title{Вторая уютная домашка}
\author{Дарина Шебзухова}
 
 
\begin{document}
 
\maketitle
\begin{center}
 
\large{СОГЛАСНО ЗАКОНУ ВЫСОКОГОРНЫХ УЩЕЛИЙ КАВКАЗА:\\ если ты, брат, украдёшь у своего брата что-то ценное (например, сон!), то ты, брат, обрекаешь себя на жизнь в страхе. \\Братья обиженного брата с черными-черными бородами на черных-черных посаженных приорах будут следить за тобой каждый день, брат. \\А так как главная заповедь истинного кавказца, что брат за брата это за основу взято, то, отвечаю, брат, ты ответишь за содеянное. \\ За то, что ты лишил брата сна тебя, брат, завернут в бурку и увезут в самую глухую деревню великого Дагестана и будешь ты каждый день убаюкивать младенцев.}

 
\end{center}
\newpage
\begin{equation}
f'(x_0)=\lim\limits_{x \to x_0}\frac{f\left(x_0+\Delta x\right)-f(x_0)}{\Delta x}
\end{equation}
 
\begin{equation}
\sin{\alpha}\pm \sin{\beta} = 2 \cdot \sin{\dfrac{\alpha \pm \beta}{2}} \cdot \cos{\dfrac{\alpha \mp \beta}{2}}
\end{equation}
 
 
\begin{equation}
\lim\limits_{x \to 0}\dfrac{\ln{(1+x)}}{x}=1
\end{equation}
\end{document}
 
\begin{equation}
G= \sum\limits_{i=1}^n p_i \cdot q_{i+1} - \sum\limits_{i=1}^n p_{i+1}\cdot q_i
\end{equation}
 
\end{document }