%!TEX TS-program = xelatex
\documentclass[14pt,a4paper, oneside]{report}

\usepackage{amsmath,amsfonts,amssymb,amsthm,mathtools}
%\mathtoolsset{showonlyrefs=true}  % Показывать номера только у тех формул, на которые есть \eqref{} в тексте.
%\usepackage{leqno} % Нумерация формул слева

\usepackage{fontspec}         % пакет для подгрузки шрифтов
\setmainfont{Arial}   % задаёт основной шрифт документа
\defaultfontfeatures{Mapping=tex-text}

\newfontfamily{\cyrillicfonttt}{Arial}
\newfontfamily{\cyrillicfont}{Arial}
\newfontfamily{\cyrillicfontsf}{Arial}

\usepackage{polyglossia}      % Пакет, который позволяет подгружать русские буквы
\setdefaultlanguage{russian}  % Основной язык документа
\setotherlanguage{english}    % Второстепенный язык документа
\setkeys{russian}{babelshorthands=true}
\usepackage{graphicx}                  % Для вставки рисунков
\usepackage{graphics}
\graphicspath{{images/}{pictures/}}    % можно указать папки с картинками
\usepackage{hyperref}
\hypersetup{
    unicode=true,           % позволяет использовать юникодные символы
    colorlinks=true,       	% true - цветные ссылки, false - ссылки в рамках
    urlcolor=black,          % цвет ссылки на url
    linkcolor=black,          % внутренние ссылки
    citecolor=green,        % на библиографию
	pdfnewwindow=true,      % при щелчке в pdf на ссылку откроется новый pdf
	breaklinks              % если ссылка не умещается в одну строку, разбивать ли ее на две части?
}

\usepackage{extsizes} 
\usepackage[paper=a4paper,top=10mm, bottom=10mm,left=35mm,right=35mm,includefoot,includehead]{geometry}

\usepackage{setspace}
\setstretch{1.33}  % Межстрочный интервал
\setlength{\parindent}{0em} % Красная строка.
\setlength{\parskip}{5mm}   % Расстояние 
\flushbottom                            % 
\righthyphenmin=2                       % Разрешение переноса двух и более символов
\widowpenalty=300                     % Небольшое наказание за вдовствующую строку (одна строка абзаца на этой странице, остальное --- на следующей)
\clubpenalty=3000                     % Приличное наказание за сиротствующую строку (омерзительно висящая одинокая строка в начале страницы)
\tolerance=1000     % Ещё какое-то наказание.

\newcounter{i}
\newcommand{\pr}[1]{%
\addtocounter{i}{1}    % увеличение счетчика на единицу
\textbf{\large{Приложение \Asbuk{i}:}}
#1}
\usepackage{fancyhdr} % Колонтитулы
\pagestyle{fancy}
\renewcommand{\headrulewidth}{0.2pt}  
 	\chead{}
	\lhead{Приложение \Asbuk{i}}
	\rhead{\thepage} % номер страницы
	\cfoot{}

\usepackage{afterpage}  
\renewcommand{\labelitemi}{\includegraphics[height=8mm]{123.jpg}}

\begin{document}
\thispagestyle{empty}
\begin{center}
\includegraphics[height=6cm]{hogv.png}
\end{center}

\vspace{1cm}

{\fontsize{12}{1.33}\selectfont Мистеру Даши Ешиеву.}

\vspace{2.5cm}

Мистер Даши!
\\

Рады сообщить Вам, что по~результатам предварительного тестирования вы набрали достаточное количество баллов для~поступления на~первый курс "Хогвартса".Просим Вас отправить через~Вашу сову ответное письмо с~согласием на~зачисление не~позднее 25~августа 2017~года, а~также изучить приложения к~данному письму.
\\

Ваш куратор по борьбе с темной магией, Аластор Грюм.
\\ 

{\fontspec{Silver Age Queens}{Alastor Moody}}

\vfill
\begin{center}
\href{http://vhogwarts.ru}{Школа Чародейства и Волшебства "Хогвартс"}
\end{center}
\newpage
\pr{\textbf{\large{Список необходимых книг и предметов.}}}
\begin{itemize}
\item Волшебная палочка
\item Телескоп
\item Остроконечная шляпа
\item Котел
\item Учебник по астрономии
\end{itemize}
\vfill
\begin{center}
\href{http://vhogwarts.ru}{Школа Чародейства и Волшебства "Хогвартс"}
\end{center}
\newpage
\pr{\textbf{\large{Список изучаемых предметов.}}}
\begin{itemize}
\item Астрономия
\item Зельеварение
\item Защита от темных искусств
\item Травология
\item Заклинания
\end{itemize}
\vfill
\begin{center}
\href{http://vhogwarts.ru}{Школа Чародейства и Волшебства "Хогвартс"}
\end{center}
\end{document}