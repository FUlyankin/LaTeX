\documentclass[12pt,a4paper]{article}
\usepackage{etex} % расширение классического tex в частности позволяет подгружать гораздо больше пакетов, чем мы и займёмся далее


\usepackage{fontspec}         % пакет для подгрузки шрифтов
\setmainfont{Victorian Gothic Two}   % задаёт основной шрифт документа


\defaultfontfeatures{Mapping=tex-text}
% why do we need \newfontfamily:
% http://tex.stackexchange.com/questions/91507/
\newfontfamily{\cyrillicfonttt}{Victorian Gothic Two}
\newfontfamily{\cyrillicfont}{Arial}
\newfontfamily{\cyrillicfontsf}{Arial}

\usepackage{pgf,tikz}
\usepackage{mathrsfs}
\usetikzlibrary{arrows}


\usepackage{polyglossia}      % Пакет, который позволяет подгружать русские буквы
\setdefaultlanguage{russian}  % Основной язык документа
\setotherlanguage{english}    % Второстепенный язык документа



%%%%%%%%%% Работа с картинками %%%%%%%%%
\usepackage{graphicx}                  % Для вставки рисунков
\usepackage{graphics}
\graphicspath{{Pics/}}    % можно указать папки с картинками




\usepackage{float}               % возможность позиционировать объекты в нужном месте
\usepackage[top=0mm,
bottom=0mm,
left=0mm,
right=0mm]{geometry}
 

\begin{document}
\pagestyle{empty}
\definecolor{ffqqqq}{rgb}{1.,0.,0.}
\begin{figure}(50,50)
\setlength{\unitlength}{4mm}
\put(-3.5,0){\includegraphics{bumaga.jpg}}
\put(15,85){{\fontsize{64}{1.33}\fontspec{Victorian Gothic Two}{\textit{Контракт}}}} 
\put(11,65){\Large{Я согласна выполнять эту работу.}}
	%\put(47.5,93){\line(0,-1){70}}
	%\put(-1.5,93){\line(0,-1){70}}
	\put(-1.5,93){\line(1,0){49}}
	%\put(-1.5,23){\line(1,0){49}}
\put(15,50){\begin{tikzpicture}[line cap=round,line join=round,>=triangle 45,x=0.28936732976352336cm,y=0.3538647342995175cm]
\clip(-1.69,-1.23) rectangle (4.96,7.05);
\draw [line width=2.8pt,color=ffqqqq] (1.56,4.78)-- (0.96,4.)-- (0.64,3.3)-- (0.3,2.14)-- (0.44,1.3)-- (0.8,0.62)-- (1.48,0.3)-- (2.22,0.26)-- (2.8,0.58)-- (3.14,1.28)-- (3.14,1.96)-- (3.04,2.66)-- (2.76,3.2)-- (2.44,3.8)-- (2.2,4.6)-- (2.34,5.64);
\draw [line width=2.8pt,color=ffqqqq] (2.34,5.64)-- (1.56,4.78);
\end{tikzpicture}}
\put(20,54){\begin{tikzpicture}[line cap=round,line join=round,>=triangle 45,x=0.28936732976352336cm,y=0.3538647342995175cm]
\clip(-1.69,-1.23) rectangle (4.96,7.05);
\draw [line width=2.8pt,color=ffqqqq] (1.56,4.78)-- (0.96,4.)-- (0.64,3.3)-- (0.3,2.14)-- (0.44,1.3)-- (0.8,0.62)-- (1.48,0.3)-- (2.22,0.26)-- (2.8,0.58)-- (3.14,1.28)-- (3.14,1.96)-- (3.04,2.66)-- (2.76,3.2)-- (2.44,3.8)-- (2.2,4.6)-- (2.34,5.64);
\draw [line width=2.8pt,color=ffqqqq] (2.34,5.64)-- (1.56,4.78);
\end{tikzpicture}}
\put(25,49){\begin{tikzpicture}[line cap=round,line join=round,>=triangle 45,x=0.28936732976352336cm,y=0.3538647342995175cm]
\clip(-1.69,-1.23) rectangle (4.96,7.05);
\draw [line width=2.8pt,color=ffqqqq] (1.56,4.78)-- (0.96,4.)-- (0.64,3.3)-- (0.3,2.14)-- (0.44,1.3)-- (0.8,0.62)-- (1.48,0.3)-- (2.22,0.26)-- (2.8,0.58)-- (3.14,1.28)-- (3.14,1.96)-- (3.04,2.66)-- (2.76,3.2)-- (2.44,3.8)-- (2.2,4.6)-- (2.34,5.64);
\draw [line width=2.8pt,color=ffqqqq] (2.34,5.64)-- (1.56,4.78);
\end{tikzpicture}}
	\put (16.5,40){{\Large{Место для крови}}}
	\put (13,30){{\noindent Место для подписи \hfill /\rule{10em}{0.5pt}/}}
	\put(-3.3,20){\includegraphics[width=1\linewidth]{venz.png}}
	\put(-1.55,93.1){\includegraphics[width=0.25\linewidth, angle=180]{venz2.png}}
	\put(34.6,80.15){\includegraphics[width=0.25\linewidth]{venz3.png}}
\end{figure}

\end{document}