\documentclass[12pt, a4paper]{article}

\usepackage[title,titletoc]{appendix}

%%%%%%%%%% Програмный код %%%%%%%%%%
\usepackage{minted}
% Включает подсветку команд в программах!
% Нужно, чтобы на компе стоял питон, надо поставить пакет Pygments, в котором он сделан, через pip.

% Для Windows: Жмём win+r, вводим cmd, жмём enter. Открывается консоль.
% Прописываем easy_install Pygments
% Заходим в настройки texmaker и там прописываем в PdfLatex:
% pdflatex -shell-escape -synctex=1 -interaction=nonstopmode %.tex

% Для Linux: Открываем консоль. Убеждаемся, что у вас установлен pip командой pip --version
% Если он не установлен, ставим его: sudo apt-get install python-pip
% Ставим пакет sudo pip install Pygments

% Для Mac: Всё то же самое, что на Linux, но через brew.

% После всего этого вы должны почувствовать себя тру-программистами!
% Документация по пакету хорошая. Сам читал, погуглите!
%%%%%%%%%%%%%%%%%%%%%%%% Шрифты %%%%%%%%%%%%%%%%%%%%%%%%%%%%%%%%%
\usepackage{fontspec}         % пакет для подгрузки шрифтов
\setmainfont{Arial}   % задаёт основной шрифт документа

% Команда, которая нужна для корректного отображения длинных тире и некоторых других символов.
\defaultfontfeatures{Mapping=tex-text}

% why do we need \newfontfamily:
% http://tex.stackexchange.com/questions/91507/
\newfontfamily{\cyrillicfonttt}{Arial}
\newfontfamily{\cyrillicfont}{Arial}
\newfontfamily{\cyrillicfontsf}{Arial}

\usepackage{unicode-math}     % пакет для установки математического шрифта
\setmathfont{Asana Math}      % шрифт для математики
% \setmathfont[math-style=ISO]{Asana Math}
% Можно делать смену начертания с помощью разных стилей

% Конкретный символ из конкретного шрифта
% \setmathfont[range=\int]{Neo Euler}

\usepackage{polyglossia}      % Пакет, который позволяет подгружать русские буквы
\setdefaultlanguage{russian}  % Основной язык документа
\setotherlanguage{english}    % Второстепенный язык документа

\begin{document}

\section{Оформление кода}

\begin{minted}[mathescape]{python}
template = '''{} bottles of beer on the wall.
Take one down and pass it around, {} bottles of beer on the wall.'''
count = 0  # Никому не нужный счётчик

for i in s.upper():  # Игнорируем регистр
    if i == 'H':
        print('Hello, world!')  # Выводим 'Hello, world!'
    elif i == 'Q':
        print(s)  # Выводим саму программу
    elif i == '9':
        for i in range(99, 1, -1):
            print(template.format(i, i-1))  # Выводим текст песни
        print('1 bottle of beer on the wall.\nTake one down and pass it around, 
        no more bottles of beer on the wall.')
        print('No more bottles of beer on the wall.\nGo to the store and buy some more, 
        99 bottles of beer on the wall.')
    elif i == '+':
        count += 1       
\end{minted}

\end{document}