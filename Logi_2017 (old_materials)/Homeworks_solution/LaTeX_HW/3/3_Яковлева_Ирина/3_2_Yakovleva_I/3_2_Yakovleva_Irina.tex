\documentclass[10pt, a4paper]{article}
\usepackage{etex} % расширение классического tex в частности позволяет подгружать гораздо больше пакетов, чем мы и займёмся далее
%%%%%%%%%% Математика %%%%%%%%%%
\usepackage{amsmath,amsfonts,amssymb,amsthm,mathtools} 
%\mathtoolsset{showonlyrefs=true}  % Показывать номера только у тех формул, на которые есть \eqref{} в тексте.

%%%%%%%%%%%%%%%%%%%%%%%% Шрифты %%%%%%%%%%%%%%%%%%%%%%%%%%%%%%%%%
\usepackage{fontspec}         % пакет для подгрузки шрифтов
\setmainfont{Arial}   % задаёт основной шрифт документа
\defaultfontfeatures{Mapping=tex-text}
%why do we need \newfontfamily:
% http://tex.stackexchange.com/questions/91507/
\newfontfamily{\cyrillicfonttt}{Arial}
\newfontfamily{\cyrillicfont}{Arial}
\newfontfamily{\cyrillicfontsf}{Arial}
\usepackage{unicode-math}     % пакет для установки математического шрифта
\setmathfont{Asana Math}      % шрифт для математики
% \setmathfont[math-style=ISO]{Asana Math}
% Можно делать смену начертания с помощью разных стилей

\usepackage{polyglossia}      % Пакет, который позволяет подгружать русские буквы
\setdefaultlanguage{russian}  % Основной язык документа
\setotherlanguage{english}    % Второстепенный язык документа
\usepackage{xcolor}

%%%%%%%%%% Графика и рисование %%%%%%%%%%
\usepackage{tikz, pgfplots}  % язык для рисования графики из latex'a
\usepackage{amscd}                  %Пакеты для рисования
\usepackage[matrix,arrow,curve]{xy} %комунитативных диаграмм

\usepackage{pgf,tikz}
\usepackage{mathrsfs}
\usetikzlibrary{arrows}
\pagestyle{empty}
\begin{document}
\newcommand{\cowsay}[1]{
\begin{tikzpicture}[line cap=round,line join=round,>=triangle 45,x=0.3106011608681261cm,y=0.19094836678775734cm]
\clip(-23.290941943260634,-13.110174447518817) rectangle (31.44162876730988,23.548950308340604);
\draw [shift={(2.,-1.)}] plot[domain=-1.5707963267948966:1.5707963267948966,variable=\t]({1.*1.*cos(\t r)+0.*1.*sin(\t r)},{0.*1.*cos(\t r)+1.*1.*sin(\t r)});
\draw [shift={(0.,1.)}] plot[domain=1.5707963267948966:4.71238898038469,variable=\t]({1.*1.*cos(\t r)+0.*1.*sin(\t r)},{0.*1.*cos(\t r)+1.*1.*sin(\t r)});
\draw [shift={(0.,-1.)}] plot[domain=1.5707963267948966:4.71238898038469,variable=\t]({1.*1.*cos(\t r)+0.*1.*sin(\t r)},{0.*1.*cos(\t r)+1.*1.*sin(\t r)});
\draw [shift={(2.,1.)}] plot[domain=-1.5707963267948966:1.5707963267948966,variable=\t]({1.*1.*cos(\t r)+0.*1.*sin(\t r)},{0.*1.*cos(\t r)+1.*1.*sin(\t r)});
\draw (0.,-2.)-- (2.,-2.);
\draw (0.,2.4)-- (2.,2.4);
\draw(0.,1.3) ellipse (0.1541796130602356cm and 0.09478504595261328cm);
\draw(2.002,1.302) ellipse (0.1452763677038447cm and 0.08931159519292559cm);
\draw (-0.6,2.7)-- (-0.3,3.2);
\draw (-0.3,3.2)-- (0.,2.7);
\draw (2.,2.7)-- (2.3,3.2);
\draw (2.6,2.7)-- (2.3,3.2);
\draw (3.,2.)-- (4.,0.);
\draw (3.,0.)-- (4.,-2.);
\draw (4.,-2.5)-- (4.,-4.9);
\draw (4.004,-5.4)-- (4.,-8.);
\draw (4.9,-2.5)-- (4.9,-4.9);
\draw (4.9,-5.4)-- (4.9,-8.);
\draw (5.2,-4.)-- (10.934,-4.044);
\draw (14.,-5.4)-- (14.,-8.);
\draw (13.1,-8.)-- (13.1,-5.4);
\draw (13.1,-4.)-- (12.65,-5.254);
\draw (12.,-4.)-- (12.65,-5.254);
\draw (12.,-4.)-- (11.55,-5.298);
\draw (10.934,-4.044)-- (11.55,-5.298);
\draw (13.1,-4.)-- (13.1,-4.9);
\draw (14.,-4.)-- (14.,-4.9);
\draw [shift={(12.,-2.)}] plot[domain=-0.7853981633974483:0.7853981633974483,variable=\t]({1.*2.8284271247461903*cos(\t r)+0.*2.8284271247461903*sin(\t r)},{0.*2.8284271247461903*cos(\t r)+1.*2.8284271247461903*sin(\t r)});
\draw (4.,0.)-- (14.,0.);
\draw (14.608,0.)-- (15.246,-1.558);
\draw (15.422,0.)-- (15.246,-1.558);
\draw (15.422,0.)-- (16.016,-1.558);
\draw (-1.5713728811563021,0.2674627227783202)-- (-1.8773728811563022,4.06546272277832);
\draw (-10.90560294004714,3.9755961976228056)-- (6.793577879726595,4.0410589204011265);
\draw [dash pattern=on 3pt off 3pt] (-13.276393904358637,16.217125357168715)-- (8.919114606350691,16.086199911612077);
\begin{scriptsize}
\draw[color=black] (-2.219515355285205,8.361598623770274) node {#1};
\end{scriptsize}
\end{tikzpicture}}
\cowsay{Hellow!}
\cowsay{Добрый день!Какая завтра погода?}
\end{document}
