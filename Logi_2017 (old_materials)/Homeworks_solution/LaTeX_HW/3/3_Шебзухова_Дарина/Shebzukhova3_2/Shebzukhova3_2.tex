%!TEX TS-program = xelatex
\documentclass[12pt, a4paper]{article}  

\usepackage{etex} % расширение классического tex в частности позволяет подгружать гораздо больше пакетов, чем мы и займёмся далее

%%%%%%%%%% Математика %%%%%%%%%%
\usepackage{amsmath,amsfonts,amssymb,amsthm,mathtools} 
%\mathtoolsset{showonlyrefs=true}  % Показывать номера только у тех формул, на которые есть \eqref{} в тексте.
%\usepackage{leqno} % Нумерация формул слева



%%%%%%%%%%%%%%%%%%%%%%%% Шрифты %%%%%%%%%%%%%%%%%%%%%%%%%%%%%%%%%
\usepackage{fontspec}         % пакет для подгрузки шрифтов
\setmainfont{Arial}   % задаёт основной шрифт документа

\defaultfontfeatures{Mapping=tex-text}

% why do we need \newfontfamily:
% http://tex.stackexchange.com/questions/91507/
\newfontfamily{\cyrillicfonttt}{Arial}
\newfontfamily{\cyrillicfont}{Arial}
\newfontfamily{\cyrillicfontsf}{Arial}

\usepackage{unicode-math}     % пакет для установки математического шрифта
\setmathfont{Asana Math}      % шрифт для математики
% \setmathfont[math-style=ISO]{Asana Math}
% Можно делать смену начертания с помощью разных стилей

% Конкретный символ из конкретного шрифта
% \setmathfont[range=\int]{Neo Euler}

\usepackage{polyglossia}      % Пакет, который позволяет подгружать русские буквы
\setdefaultlanguage{russian}  % Основной язык документа
\setotherlanguage{english}    % Второстепенный язык документа


%%%%%%%%%% Работа с картинками %%%%%%%%%
\usepackage{graphicx}                  % Для вставки рисунков
\usepackage{graphics}
\graphicspath{{images/}{pictures/}}    % можно указать папки с картинками
\usepackage{wrapfig}                   % Обтекание рисунков и таблиц текстом
\usepackage{subfigure}                 % для создания нескольких рисунков внутри одного


%%%%%%%%%% Работа с таблицами %%%%%%%%%%
\usepackage{tabularx}            % новые типы колонок
\usepackage{tabulary}            % и ещё новые типы колонок
\usepackage{array}               % Дополнительная работа с таблицами
\usepackage{longtable}           % Длинные таблицы
\usepackage{multirow}            % Слияние строк в таблице
\usepackage{float}               % возможность позиционировать объекты в нужном месте
\usepackage{booktabs}            % таблицы как в книгах!
\renewcommand{\arraystretch}{1.3} % больше расстояние между строками

% Заповеди из документации к booktabs:
% 1. Будь проще! Глазам должно быть комфортно
% 2. Не используйте вертикальные линни
% 3. Не используйте двойные линии. Как правило, достаточно трёх горизонтальных линий
% 4. Единицы измерения - в шапку таблицы
% 5. Не сокращайте .1 вместо 0.1
% 6. Повторяющееся значение повторяйте, а не говорите "то же"
% 7. Есть сомнения? Выравнивай по левому краю!

%%%%%%%%%% Графика и рисование %%%%%%%%%%
\usepackage{tikz, pgfplots}  % язык для рисования графики из latex'a
\usepackage{amscd}                  %Пакеты для рисования
\usepackage[matrix,arrow,curve]{xy} %комунитативных диаграмм

% Всякие странные команды из Geogebra и с сайта для TikZ
\usepackage{pgf}
\usepackage{mathrsfs}
\usetikzlibrary{arrows}
\pagestyle{empty}

\definecolor{ffzzzz}{rgb}{1.,0.6,0.6}
\definecolor{xdxdff}{rgb}{0.49019607843137253,0.49019607843137253,1.}
\definecolor{qqqqff}{rgb}{0.,0.,1.}
\definecolor{cqcqcq}{rgb}{0.7529411764705882,0.7529411764705882,0.7529411764705882}

\usetikzlibrary{calc}
\usepackage{relsize}
\newcommand\LM{\ensuremath{\mathit{LM}}}
\newcommand\IS{\ensuremath{\mathit{IS}}}




\title{Cow}
\date{\today}
\author{Darina}


\begin{document} 
\definecolor{qqzzcc}{rgb}{0.,0.6,0.8}
\definecolor{uuuuuu}{rgb}{0.26666666666666666,0.26666666666666666,0.26666666666666666}
\definecolor{ffffff}{rgb}{1.,1.,1.}
\definecolor{cqcqcq}{rgb}{0.7529411764705882,0.7529411764705882,0.7529411764705882}
\definecolor{yqyqyq}{rgb}{0.5019607843137255,0.5019607843137255,0.5019607843137255}
\definecolor{sqsqsq}{rgb}{0.12549019607843137,0.12549019607843137,0.12549019607843137}

\newenvironment{totoro}{\hfill \break \hfill}{ \hfill
\begin{tikzpicture}[line cap=round,line join=round,>=triangle 45,x=1.0cm,y=1.0cm]
\clip(-3.64,-5.08) rectangle (19.36,6.06);
\draw [color=ffffff,fill=ffffff,fill opacity=1.0] (2.9,4.42) circle (0.3757658845611186cm);
\draw [color=ffffff,fill=ffffff,fill opacity=1.0] (4.84,4.4) circle (0.3841874542459709cm);
\draw [rotate around={90.:(4.,0.5)},color=yqyqyq,fill=yqyqyq,fill opacity=1.0] (4.,0.5) ellipse (4.720767669347601cm and 3.1679089930043096cm);
\draw [rotate around={-87.81604977232399:(3.95,-0.06)},color=cqcqcq,fill=cqcqcq,fill opacity=1.0] (3.95,-0.06) ellipse (4.050730307193611cm and 3.2910053208125998cm);
\draw [shift={(5.824510642606494,-0.38241325937213916)},color=yqyqyq,fill=yqyqyq,fill opacity=1.0]  plot[domain=-0.856637004197351:0.7062128781552685,variable=\t]({1.*1.7594230660910835*cos(\t r)+0.*1.7594230660910835*sin(\t r)},{0.*1.7594230660910835*cos(\t r)+1.*1.7594230660910835*sin(\t r)});
\draw [shift={(1.4700579744364364,-0.15072889908802614)},color=yqyqyq,fill=yqyqyq,fill opacity=1.0]  plot[domain=2.169466395112312:4.188085023310635,variable=\t]({1.*1.2909985355499514*cos(\t r)+0.*1.2909985355499514*sin(\t r)},{0.*1.2909985355499514*cos(\t r)+1.*1.2909985355499514*sin(\t r)});
\draw [shift={(2.649882523654429,-3.3628891779459273)},color=yqyqyq,fill=yqyqyq,fill opacity=1.0]  plot[domain=2.815292570495524:5.385596881654315,variable=\t]({1.*0.9100865311221047*cos(\t r)+0.*0.9100865311221047*sin(\t r)},{0.*0.9100865311221047*cos(\t r)+1.*0.9100865311221047*sin(\t r)});
\draw [shift={(5.092078702448439,-3.3760726577010693)},color=yqyqyq,fill=yqyqyq,fill opacity=1.0]  plot[domain=-2.1605543743426328:0.1227743558752664,variable=\t]({1.*0.9202894438686645*cos(\t r)+0.*0.9202894438686645*sin(\t r)},{0.*0.9202894438686645*cos(\t r)+1.*0.9202894438686645*sin(\t r)});
\draw [rotate around={-89.54086562247514:(3.0040732317788117,5.49170873309938)},color=yqyqyq,fill=yqyqyq,fill opacity=1.0] (3.0040732317788117,5.49170873309938) ellipse (0.5683550454333064cm and 0.25426532293040904cm);
\draw [rotate around={-88.854237161825:(4.59,5.52)},color=yqyqyq,fill=yqyqyq,fill opacity=1.0] (4.59,5.52) ellipse (0.5521566086863932cm and 0.2340446976883859cm);
\draw (5.66,4.18)-- (6.54,5.02);
\draw (5.66,4.18)-- (6.68,4.34);
\draw (5.66,4.18)-- (6.8,3.82);
\draw (2.2,4.14)-- (1.4,4.8);
\draw (2.2,4.14)-- (1.28,4.34);
\draw (2.2,4.14)-- (1.3,3.94);
\draw (1.28,4.34)-- (1.28,4.34);
\draw (3.56,4.02)-- (4.18,4.02);
\begin{scriptsize}
\draw [fill=sqsqsq] (2.9,4.42) circle (2.5pt);
\draw [fill=sqsqsq] (4.84,4.4) circle (2.5pt);
\draw[color=yqyqyq] (2.3,-3.81) node {$k$};
\draw[color=yqyqyq] (5.74,-3.85) node {$p$};
\draw [fill=black,shift={(3.86,4.38)},rotate=180] (0,0) ++(0 pt,3.75pt) -- ++(3.2475952641916446pt,-5.625pt)--++(-6.495190528383289pt,0 pt) -- ++(3.2475952641916446pt,5.625pt);
\end{scriptsize}
\end{tikzpicture}
}

\begin{center}


\begin{totoro}

Я полупингвин - полумышь, хотя должен быть Тоторо.\\
Но я не виноват, это просто кто-то плохо рисует!\\
В geogebra у меня хоть глазные яблоки были, а тут совсем печалька!
И ещё у меня есть третий глаз для закрепления всей красоты.
\end{totoro}

\end{center}


\end{document}
