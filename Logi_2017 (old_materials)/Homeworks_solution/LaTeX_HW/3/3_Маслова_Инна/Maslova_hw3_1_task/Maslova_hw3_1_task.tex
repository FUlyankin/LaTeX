\documentclass[12pt, a4paper]{article}  

\usepackage{etex} % расширение классического tex в частности позволяет подгружать гораздо больше пакетов

%%%%%%%%%% Математика %%%%%%%%%%
\usepackage{amsmath,amsfonts,amssymb,amsthm,mathtools} 


%%%%%%%%%%%%%%%%%%%%%%%% Шрифты %%%%%%%%%%%%%%%%%%%%%%%%%%%%%%%%%
\usepackage{fontspec}         % пакет для подгрузки шрифтов
\setmainfont{Arial}   % задаёт основной шрифт документа

\defaultfontfeatures{Mapping=tex-text}

% why do we need \newfontfamily:
% http://tex.stackexchange.com/questions/91507/
\newfontfamily{\cyrillicfonttt}{Arial}
\newfontfamily{\cyrillicfont}{Arial}
\newfontfamily{\cyrillicfontsf}{Arial}


\usepackage{unicode-math}     % пакет для установки математического шрифта
\setmathfont{Asana Math}      % шрифт для математики


\usepackage{polyglossia}      % Пакет, который позволяет подгружать русские буквы
\setdefaultlanguage{russian}  % Основной язык документа
\setotherlanguage{english}    % Второстепенный язык документа


%%%%%%%%%% Работа с картинками %%%%%%%%%
\usepackage{graphicx}                  % Для вставки рисунков
\usepackage{graphics}
\usepackage{wrapfig}                   % Обтекание рисунков и таблиц текстом


%%%%%%%%%% Работа с таблицами %%%%%%%%%%
\usepackage{tabularx}            % новые типы колонок
\usepackage{tabulary}            % и ещё новые типы колонок
\usepackage{array}               % Дополнительная работа с таблицами
\usepackage{longtable}           % Длинные таблицы
\usepackage{multirow}            % Слияние строк в таблице
\usepackage{float}               % возможность позиционировать объекты в нужном месте
\usepackage{booktabs}            % таблицы как в книгах!
\renewcommand{\arraystretch}{1.3} % больше расстояние между строками


%%%%%%%%%% Графика и рисование %%%%%%%%%%
\usepackage{tikz, pgfplots}  % язык для рисования графики из latex'a
\usepackage{amscd}                  %Пакеты для рисования
\usepackage[matrix,arrow,curve]{xy} %комунитативных диаграмм


%%%%%%%%%% Свои команды %%%%%%%%%%
\usepackage{etoolbox}    % логические операторы для своих макросов
\usepackage{xparse}      % больше команд для создания команд
\DeclareMathOperator{\Var}{Var} 
\DeclareMathOperator{\Cov}{Cov}
\usepackage{ccaption}

% Пакет, который ставит в каждом первом абзаце главы красную строку
% Просто, чтобы эта pdf-ка нормально смотрелась :)
\usepackage{indentfirst}  
\setkeys{russian}{babelshorthands=true}



\begin{document}

\section{Создание операторов}
\subsection{Математические операторы}
$\Cov$ \\

$\Var$
\subsection{$\sigma$-алгебра}
\def \s {\ensuremath{\sigma}}
\s-алгебра
\subsection{Последовательность}

\def \p {\ensuremath{x_1 \ldots x_n}}
\p

\section{Усовершенствование}
\newcommand{\com}[2]{\ensuremath{x_#1 \ldots x_#2}}
\com{a}{z} \\

\com{1}{6} \\

\com{{a,b}}{{c,d}}
\renewcommand{\labelitemi}{\LARGE{\textcolor{blue}{\textbullet}}}
\section{Перечисление}
\begin{itemize}
\item Первый пункт
\item Второй пункт
\item Третий пункт
\end{itemize}
\subsection{Переопределение}
\newcommand{\llim}{\lim\limits}
$\llim_{x \to 0} \frac{\sin{x}}{x}$

\renewcommand{\thefigure}{\thesection:\arabic{figure}}
\captiondelim{ } 
\begin{figure}[H]
\begin{center}
\includegraphics[width=0.3\textheight]{i.jpg}
\end{center}
\caption{}
\end{figure}

\renewcommand{\theequation}{Eq.~(\arabic{equation})}
\section{Нумерация формул}
\begin{equation}
 D= \frac{\rho_b}{\rho_{bs}}\times 100
\end{equation}
\begin{equation}
a+b=c
\end{equation}
\subsection{Перевороты с текстом}
\newcommand{\goods} [2] {\ifstrequal{#1}{Зеркально}{\reflectbox{#2}}{\ifstrequal{#1}{Поворот}{\rotatebox{180}{#2}}{\reflectbox{\rotatebox{180}{#2}}}}}
\goods{Зеркально}{Привет}
\goods{Поворот}{Hello}


\end{document}