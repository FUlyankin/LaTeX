%!TEX TS-program = xelatex
\documentclass[12pt, a4paper]{article}  

%         Классы: 
% article   ---   статья
% report    ---   отчет
% book      ---   книга
% beamer    ---   презентация

\usepackage{amsmath,amsfonts,amssymb,amsthm,mathtools}  % пакеты для математики
\usepackage{graphicx}
%%%%%%%%%%%%%%%%%%%%%%%% Шрифты %%%%%%%%%%%%%%%%%%%%%%%%%%%%%%%%%

\usepackage{fontspec}         % пакет для подгрузки шрифтов
\setmainfont{Linux Libertine O}   % задаёт основной шрифт документа
\usepackage{hyperref}

\defaultfontfeatures{Mapping=tex-text}

% why do we need \newfontfamily:
% http://tex.stackexchange.com/questions/91507/
\newfontfamily{\cyrillicfonttt}{Linux Libertine O}
\newfontfamily{\cyrillicfont}{Linux Libertine O}
\newfontfamily{\cyrillicfontsf}{Linux Libertine O}

\usepackage{unicode-math}     % пакет для установки математического шрифта
\setmathfont{Asana Math}      % шрифт для математики

\usepackage{polyglossia}      % Пакет, который позволяет подгружать русские буквы
\setdefaultlanguage{russian}  % Основной язык документа
\setotherlanguage{english} % Второстепенный язык документа 
\usepackage[usenames]{color}
\usepackage{colortbl}



\begin{document}
 \pagecolor[rgb]{1,0.97,0.82}
\title{Домашняя работа}
\tableofcontents
\newpage
\section{Несколько фактов обо мне :) }
\begin{enumerate}
\item Я люблю поспать
\item Я люблю послушать музыку
\item Я люблю велопрогулки
\item Я щедрый
\item Я не люблю имя {\it {\bf ГАЛЯ}}
\item Обожаю бухучет \textbf{\textit{(Нет)}}
\item Совершенно не имею представления как доучился до 3 курса
\item Я не люблю вранья
\item Я учился у Бабашкина
\item Я не знаю что писать в этом пункте
\end{enumerate}
\newpage
\section{{\bf Эт я}}
\includegraphics[scale=0.65]{doge.jpg}
\newpage
\section{The Forмулы}
\subsection{Любимые}

\begin{equation}\label{æ} 
\int\frac{dx}{{\cos}^2(x)}= \tan(x)
\tag{æ}
\end{equation}

\begin{equation}\label{ææ} 
\frac{\sum p_{1}q_{1}}{\sum p_{0}q_{1}}
\tag{ææ}
\end{equation}

\begin{equation}\label{æææ} 
\lim_{n \to \infty}\frac{1}{n}=0
\tag{æææ}
\end{equation}

\begin{equation}\label{ææææ} 
tr \begin{pmatrix} 

a_{11} & ... & ... & ... & ... \\ 
     ...   &...  & ... & ... & ... \\ 
 ... & ... & a_{ii}  & ... & ... \\ 
    ...    & ... & ... & ... & ...\\ 
       ...    & ... & ... & ... & a_{nn} 
\end{pmatrix}
= \sum a_{ii}
\tag{ææææ}
\end{equation}

\begin{equation}\label{æææææ} 
a^{\log_{a}b}=b
\tag{æææææ}
\end{equation}

\subsection{Нелюбимые}

\begin{equation}\label{ææææææ} 
\textcolor{red}{D=\lim_{n \to \infty} f(t)=\left [ \frac{\infty}{\infty} \right ]=\frac{S(ABC)i}{ABC}=const}
\tag{ææææææ}
\end{equation}

\newpage
\section{Пару слов о том почему я люблю одни формулы, а другие нет :) }

Я люблю (\ref{æ}), потому что она напоминает мне о первом курсе. А (\ref{ææ}) я люблю, потому что экономисту стыдно ее не знать. Формулу (\ref{æææ}) пришлось полюбить, т.к. лень писать определение предела, хоть и люблю я его больше, да и наверняка неоригинален был бы я. К (\ref{ææææ}) я неровно дышу так как это единственное, что я запомнил из линала. Ну а о (\ref{æææææ}) и говорить нечего – напоминает о школе.

По поводу (\ref{ææææææ}) можно просто посмотреть \href{https://www.youtube.com/watch?v=NrDe9O2odbw}{\textcolor{red}{РОЛИК}} (ссылка кликабельна), а потом разделить мое мнение =b 
\end{document}