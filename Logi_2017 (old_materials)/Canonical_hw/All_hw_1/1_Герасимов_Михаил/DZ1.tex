\documentclass[12pt, a4paper]{article}  
\usepackage{amsmath,amsfonts,amssymb,amsthm,mathtools}  
\usepackage{fontspec}     
\setmainfont{Arial}
  
\usepackage{graphicx}


\defaultfontfeatures{Mapping=tex-text}
\newfontfamily{\cyrillicfonttt}{Arial}
\newfontfamily{\cyrillicfont}{Arial}
\newfontfamily{\cyrillicfontsf}{Arial}
\usepackage{unicode-math}    
\setmathfont{Asana Math}     
\usepackage{polyglossia}      
\setdefaultlanguage{russian} 
\setotherlanguage{english}   

\title{Домашняя работа №1}
\author{Герасимов Михаил}
\date{}
\begin{document}
\maketitle

\section{Факты о том, что я люблю:}
1. Жизнь\\
2. Активный отдых\\
3. Девушек\\
4. Людей\\
5. Самообразование\\
6. Целеустремленность\\
7. Музыку\\
8. Спорт\\
9. Сильные эмоции\\
10. LaTex(но это не точно)

\section{Фотография}


\includegraphics[scale=0.4]{m}
\newpage
\section{Формулы}
\subsection{Любимые}
\begin{enumerate}
\item \begin{equation}\label{1}
PV=\sum_{i=1}^n\frac{CF_t}{(1+r)^t}\tag{æ}
\end{equation}
\item \begin{equation}\label{2}
Y=C+G+I+Ex \tag{ææ}
\end{equation}



\item \begin{equation} \label{3} \begin{vmatrix}
		a_{11} & a_{12} \\
		a_{21} & a_{22}
	\end{vmatrix} = a_{11}a_{22}-a_{12}a_{21} \tag{æææ}
\end{equation}
\item \begin{multline}\label{4}
\sum_{k=1}^{\infty} \frac{f^{(k)}(a)}{k!} (x-a)^k = f(a) + \frac{f'(a)}{1!} (x-a)+ \frac{f"(a)}{2!} (x-a)^2 + \cdots + \\ +\frac{f^{(n)}(a)}{n!} (x-a)^n+ \cdots \tag{ææææ}
\end{multline}

\item \begin{equation} \label{5}\lim_{x \rightarrow \infty}{\frac{\sin x}{x}}=1 \tag{æææææ}
\end{equation}
В связи с необъяснимой симпатией к ДКБ ставлю формулу \eqref{1} на первое место.
Обожаю формулу \eqref{2}, потому что совокупный выпуск это наше все. Куда же без определителей матриц \eqref{3}, проще её только совокупный выпуск.  Формула Тейлора \eqref{4} оставила довольно приятные воспоминания, она не так уж и страшна, как кажется. Ну и куда же без \eqref{5} - первого замечательного предела.
\end{enumerate}
\subsection{Так себе}
\begin{equation} \label{6} S_n=\frac{b_1(q^n-1)}{q-1}, \, q\neq 1
\tag{ææææææ}
\end{equation}
Формула \eqref{6}. Постоянно забываю сумму первых n членов геометрической прогрессии.
\end{document}
