%!TEX TS-program = xelatex
\documentclass[12pt, a4paper, oneside]{article}

\usepackage{amsmath,amsfonts,amssymb,amsthm,mathtools}  % пакеты для математики

\usepackage[utf8]{inputenc} % задание utf8 кодировки исходного tex файла
\usepackage[british,russian]{babel} % выбор языка для документа

\usepackage{fontspec}         % пакет для подгрузки шрифтов
\setmainfont{Helvetica}   % задаёт основной шрифт документа

% why do we need \newfontfamily:
% http://tex.stackexchange.com/questions/91507/
\newfontfamily{\cyrillicfonttt}{Helvetica}
\newfontfamily{\cyrillicfont}{Helvetica}
\newfontfamily{\cyrillicfontsf}{Helvetica}

\usepackage{unicode-math}     % пакет для установки математического шрифта
\setmathfont{Neo Euler}      % шрифт для математики
% \setmathfont[math-style=ISO]{Asana Math}
% Можно делать смену начертания с помощью разных стилей

% Конкретный символ из конкретного шрифта
% \setmathfont[range=\int]{Neo Euler}

%%%%%%%%%% Работа с картинками %%%%%%%%%
\usepackage{graphicx}                  % Для вставки рисунков
\usepackage{graphics}
\graphicspath{{images/}{pictures/}}    % можно указать папки с картинками
\usepackage{wrapfig}                   % Обтекание рисунков и таблиц текстом

%%%%%%%%%%%%%%%%%%%%%%%% Графики и рисование %%%%%%%%%%%%%%%%%%%%%%%%%%%%%%%%%
\usepackage{tikz, pgfplots}  % язык для рисования графики из latex'a

%%%%%%%%%% Гиперссылки %%%%%%%%%%
\usepackage{xcolor}              % разные цвета

\usepackage{hyperref}
\hypersetup{
	unicode=true,           % позволяет использовать юникодные символы
	colorlinks=true,       	% true - цветные ссылки, false - ссылки в рамках
	urlcolor=blue,          % цвет ссылки на url
	linkcolor=red,          % внутренние ссылки
	citecolor=green,        % на библиографию
	pdfnewwindow=true,      % при щелчке в pdf на ссылку откроется новый pdf
	breaklinks              % если ссылка не умещается в одну строку, разбивать ли ее на две части?
}


\usepackage{todonotes} % для вставки в документ заметок о том, что осталось сделать
% \todo{Здесь надо коэффициенты исправить}
% \missingfigure{Здесь будет Последний день Помпеи}
% \listoftodos --- печатает все поставленные \todo'шки

\usepackage{enumitem} % дополнительные плюшки для списков
%  например \begin{enumerate}[resume] позволяет продолжить нумерацию в новом списке

\usepackage[paper=a4paper, top=20mm, bottom=15mm,left=20mm,right=15mm]{geometry}
\usepackage{indentfirst}       % установка отступа в первом абзаце главы

\usepackage{setspace}
\setstretch{1.15}  % Межстрочный интервал
\setlength{\parskip}{4mm}   % Расстояние между абзацами
% Разные длины в латехе https://en.wikibooks.org/wiki/LaTeX/Lengths


\usepackage{xcolor} % Enabling mixing colors and color's call by 'svgnames'

\definecolor{MyColor1}{rgb}{0.2,0.4,0.6} %mix personal color
\newcommand{\textb}{\color{Black} \usefont{OT1}{lmss}{m}{n}}
\newcommand{\blue}{\color{MyColor1} \usefont{OT1}{lmss}{m}{n}}
\newcommand{\blueb}{\color{MyColor1} \usefont{OT1}{lmss}{b}{n}}
\newcommand{\red}{\color{LightCoral} \usefont{OT1}{lmss}{m}{n}}
\newcommand{\green}{\color{Turquoise} \usefont{OT1}{lmss}{m}{n}}

\usepackage{titlesec}
\usepackage{sectsty}
%%%%%%%%%%%%%%%%%%%%%%%%
%set section/subsections HEADINGS font and color
\sectionfont{\color{MyColor1}}  % sets colour of sections
\subsectionfont{\color{MyColor1}}  % sets colour of sections

%set section enumerator to arabic number (see footnotes markings alternatives)
\renewcommand\thesection{\arabic{section}.} %define sections numbering
\renewcommand\thesubsection{\thesection\arabic{subsection}} %subsec.num.

%define new section style
\newcommand{\mysection}{
	\titleformat{\section} [runin] {\usefont{OT1}{lmss}{b}{n}\color{MyColor1}} 
	{\thesection} {3pt} {} } 


%	CAPTIONS
\usepackage{caption}
\usepackage{subcaption}
%%%%%%%%%%%%%%%%%%%%%%%%
\captionsetup[figure]{labelfont={color=Turquoise}}

\usepackage[normalem]{ulem}  % для зачекивания текста

\pagestyle{empty}

\begin{document}
	
\section*{Задание 2:  дружба R и LaTeX (20  баллов) }

Не забывай, где находится  \href{https://github.com/FUlyankin/LaTeX}{страничка курса} с кучей шпаргалок! А также где лежит \href{https://docs.google.com/forms/d/e/1FAIpQLSe11kxKVfv07iCL1E9yNX7ll9swKImiVwRr1H70lslGzInRSg/viewform}{уютная гугл-форма.} Не стесняйтесь просить о помощи, если она вам необходима. 

В этом задании вам нужно подружить R и \LaTeX{}. Конечно же мы будем делать это, пытаясь создать положительную экстерналию. Во многих вузах, на многих курсах просят оформлять домашки в техе. Потому что так удобнее их собирать и проверять. Вам нужно оформить одну из своих текущих домашек по эконометрике/финансам/макроэкономике или любому другому предмету в связке R+\LaTeX.

Например, у третьего курса, скорее всего, будет проект по финансам, где нужно будет оценить модель CAPM. Выстройте в R + \LaTeX{} пайплайн для этого. Через пакет \texttt{quantmod}  загрузите в R данные по динамике различных ценных бумаг, затем проведите с ними все необходимые расчёты, постройте графики, таблицы и напишите поясняющий текст. В итоге выйдет так, что за одну работу вы получите баллы сразу на двух предметах. И на финансах, и на техе. 

У второго курса, скорее всего, будет какое-то похожее задание по макре. Если возможности оформить какие-нибудь вычисления в связке так и не возникнет, вы можете оформить в связке абсолютно любые расчёты. Например, можно попробовать выяснить \href{http://www.algorithmist.ru/2011/01/chickens-eggs-and-causality-or-which.html}{что было раньше: курица или яйцо,}  код для этого выглядит \href{https://www.r-bloggers.com/2013/06/chicken-or-the-egg-granger-causality-for-the-masses/}{примерно так.} Ещё можно попробовать \href{https://rpubs.com/iezepov}{оценить VaR} или сделать любые расчёты для нира. Или просто хоть что-нибудь. 

Напоследок ещё разок прочитайте \href{https://github.com/FUlyankin/LaTeX/raw/master/Logi_2018/sem_5/Anatolyev.pdf}{статью Анатольева} о правильном оформлении эконометрических расчётов. Никогда не вставляйте в свою работу протоколы из статы или ивзглядов (это шутка про Eviews). Оформляйте всё по-человечески!  И вообще, делайте эконометрику в R!

\end{document}

