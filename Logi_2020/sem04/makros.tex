%!TEX TS-program = xelatex
\documentclass[12pt, a4paper]{article}

%%%%%%%%%% Математика %%%%%%%%%%
\usepackage{amsmath,amsfonts,amssymb,amsthm,mathtools}
%\mathtoolsset{showonlyrefs=true}  % Показывать номера только у тех формул, на которые есть \eqref{} в тексте.
%\usepackage{leqno} % Нумерация формул слева

%%%%%%%%%%%%%%%%%%%%%%%% Шрифты %%%%%%%%%%%%%%%%%%%%%%%%%%%%%%%%%
\usepackage[british,russian]{babel} % выбор языка для документа
\usepackage[utf8]{inputenc} % задание utf8 кодировки исходного tex файла
\usepackage[X2,T2A]{fontenc}        % кодировка

\usepackage{fontspec}         % пакет для подгрузки шрифтов
\setmainfont{Arial}   % задаёт основной шрифт документа

\usepackage{unicode-math}     % пакет для установки математического шрифта
%\setmathfont{Asana Math}      % шрифт для математики
% \setmathfont[math-style=ISO]{Asana Math}
% Можно делать смену начертания с помощью разных стилей

% Конкретный символ из конкретного шрифта
% \setmathfont[range=\int]{Neo Euler}

%%%%%%%%%% Работа с картинками %%%%%%%%%
\usepackage{graphicx}                  % Для вставки рисунков
\usepackage{graphics} 
\graphicspath{{images/}{pictures/}}    % можно указать папки с картинками
\usepackage{wrapfig}                   % Обтекание рисунков и таблиц текстом

%%%%%%%%%% Работа с таблицами %%%%%%%%%%
\usepackage{tabularx}            % новые типы колонок
\usepackage{tabulary}            % и ещё новые типы колонок
\usepackage{array}               % Дополнительная работа с таблицами
\usepackage{longtable}           % Длинные таблицы
\usepackage{multirow}            % Слияние строк в таблице
\usepackage{float}               % возможность позиционировать объекты в нужном месте 
\usepackage{booktabs}            % таблицы как в книгах!  
\renewcommand{\arraystretch}{1.3} % больше расстояние между строками

% Заповеди из документации к booktabs:
% 1. Будь проще! Глазам должно быть комфортно
% 2. Не используйте вертикальные линни
% 3. Не используйте двойные линии. Как правило, достаточно трёх горизонтальных линий
% 4. Единицы измерения - в шапку таблицы
% 5. Не сокращайте .1 вместо 0.1
% 6. Повторяющееся значение повторяйте, а не говорите "то же"
% 7. Есть сомнения? Выравнивай по левому краю!

%%%%%%%%%% Графика и рисование %%%%%%%%%%
\usepackage{tikz, pgfplots}  % язык для рисования графики из latex'a

%%%%%%%%%% Гиперссылки %%%%%%%%%%
\usepackage{xcolor}              % разные цвета

% Два способа включить в пакете какие-то опции:
%\usepackage[опции]{пакет}
%\usepackage[unicode,colorlinks=true,hyperindex,breaklinks]{hyperref}

\usepackage{hyperref}
\hypersetup{
	unicode=true,           % позволяет использовать юникодные символы
	colorlinks=true,       	% true - цветные ссылки, false - ссылки в рамках
	urlcolor=blue,          % цвет ссылки на url
	linkcolor=red,          % внутренние ссылки
	citecolor=green,        % на библиографию
	pdfnewwindow=true,      % при щелчке в pdf на ссылку откроется новый pdf
	breaklinks              % если ссылка не умещается в одну строку, разбивать ли ее на две части?
}

\usepackage{csquotes}            % Еще инструменты для ссылок

%%%%%%%%%% Другие приятные пакеты %%%%%%%%%
\usepackage{multicol}       % несколько колонок
\usepackage{verbatim}       % для многострочных комментариев

\usepackage{enumitem} % дополнительные плюшки для списков
%  например \begin{enumerate}[resume] позволяет продолжить нумерацию в новом списке

\usepackage{todonotes} % для вставки в документ заметок о том, что осталось сделать
% \todo{Здесь надо коэффициенты исправить}
% \missingfigure{Здесь будет Последний день Помпеи}
% \listoftodos --- печатает все поставленные \todo'шки

%%%%%%%%%%%%%%%%%%%%%%%% Оформление %%%%%%%%%%%%%%%%%%%%%%%%%%%%%%%%%

\usepackage[paper=a4paper,top=15mm, bottom=15mm,left=35mm,right=10mm,includefoot]{geometry}
\usepackage{indentfirst}       % установка отступа в первом абзаце главы

%--------------------------------------------------------------------

%%%%%%%%%% Теоремы %%%%%%%%%%
\theoremstyle{plain}  % Это стиль по умалчанию для оформления теорем, есть другие стили 

\newtheorem{theorem}{Теорема}[section]
\renewcommand{\thetheorem}{\arabic{theorem}}

% Для следствий счётчик подчиняется теоремному счётчику
\newtheorem{result}{Следствие}[theorem]

\theoremstyle{definition} % убирает курсив и что-то ещё делает наверное 
\newtheorem{defin}{Определение}

\newtheorem{fig}{Какая-то фигня}


%%%%%%%%%% Свои команды %%%%%%%%%%
\usepackage{etoolbox}    % логические операторы для своих макросов

% Все свои команды лучше всего определять не по ходу текста, как это сделано в этом документе, а в преамбуле!

% Объявление своих математических операторов
\DeclareMathOperator{\Var}{Var}

% Короткие обозначения для частых команд
\def \a{\alpha} 
\def \R{\ensuremath{\mathbb{R}}}

% Если хочу поменять все списки, втыкаю правила сюда
% \renewcommand{\labelenumi}{[\Roman{enumi}]}
% \renewcommand{\labelenumii}{\alph{enumii})}

\usepackage{comment}
\usepackage{microtype}

\title{Свои команды и макросы}
\date{\today}

\begin{document}

\maketitle

\section{Новые команды}
\subsection{Создание простых команд}

Греческой буквой $\a$ обозначаются коэффициенты, которые лежат на \R для дисперсии $\Var(X)$.

% \newcommand{имя команды}{что надо делать} 
\newcommand{\RR}{{\color{blue} \ensuremath{\mathbb{R}} }} 

% ensuremath - при упоминании \RR проверяет включен ли математический режим, и если нет, сама его включает

\RR

\newcommand{\iid}{i.\hspace{3pt}i.\hspace{3pt}d.} 

$i.i.d$ или $\iid$ 

\subsection{Команды с аргументами}

% \newcommand{имя команды}[количество аргументов]{что надо делать} 

\newcommand{\bb}[1]{\ensuremath{\mathbb{#1}} }

\bb{R}  \bb{N} \bb{D}

\newcommand{\fr}[2]{^{#1}/_{#2}}

$\fr{4}{5}$  $4/5$

\LaTeX{} 

\subsection{Переопределение команд}

% \renewcommand{имя команды, которое переопределяем}{что надо делать} 

$\phi$ $\epsilon$

$\varphi$ $\varepsilon$

\renewcommand{\phi}{\varphi} 
\renewcommand{\epsilon}{\varepsilon} 

$\phi$ $\epsilon$

% на такую строку тех ругнется, так как команда \phi у него уже есть
% \newcommand{\phi}{\varphi} 

\renewcommand{\phi}{{\color{blue} \ensuremath{\mathbb{R}} }} 

$\phi$ 


\section{Счётчики}

\subsection{Дефолтные теховские счетчики}

% в какой по порядку секции я сейчаснахожусь? 
\arabic{section} 
\arabic{page} 
\asbuk{page}
\Asbuk{page}
\Roman{page}

% Примеры дефолтных счётчиков:  part, chapter, section, subsection, page, figure, table, footnote, equation

% вывести счетчик в том формате, в каком он задан 
\thesection -- \thesubsection 

% можно поменять тип счётчика в любой нумерации
% сделаем нумерацию секций римскими цифрами
\renewcommand{\thesection}{\Roman{section}}

% сделаем нумерацию страничек русскими буквами 
\renewcommand{\thepage}{\asbuk{page}}

% установить номер счётчика на конкретное значение 
\setcounter{page}{10}

\subsection{Свои счетчики}

% \newcounter{имя счётчика}[то, чему подчиняется счётчик (необязательно)]

\newcounter{jtem}[subsection]
\setcounter{jtem}{13} % установили счётчик в число 13 

\arabic{jtem} 

\subsection{Тест зависимостей}

\arabic{jtem} 


\section{Списки}

\begin{enumerate}
    \item Звёздный путь от Тарантино 
    \item Бэтман с Патисоном
    \item Снайдеркат Лиги Справедливости 
    \begin{enumerate}
        \item Фарго 4 сезон
        \item Экспансия 5 сезон
    \end{enumerate}
\end{enumerate}

\vspace{2cm} 

% \labelenumi    enumi
% \labelenumii   enumii 
% \labelenumiii  enumiii

\begin{enumerate}
    % Применятся только к текущему списку 
    \renewcommand{\labelenumi}{[\Roman{enumi}]}
    \renewcommand{\labelenumii}{\alph{enumii})}
    
    \item Звёздный путь от Тарантино 
    \item Бэтман с Патисоном
    \item Снайдеркат Лиги Справедливости 
    \begin{enumerate}
        \item Фарго 4 сезон
        \item Экспансия 5 сезон
    \end{enumerate}
\end{enumerate}

\vspace{2cm} 

% Если мы хотим применить правило оформления списка ко всем спискам, его надо вынести в преамбулу 

\begin{enumerate}
    \item Звёздный путь от Тарантино 
    \item Бэтман с Патисоном
    \item Снайдеркат Лиги Справедливости 
    \begin{enumerate}
        \item Фарго 4 сезон
        \item Экспансия 5 сезон
    \end{enumerate}
\end{enumerate}


\section{Задача}

% объявляем счётчик 
\newcounter{taskitem}[section]

% задаем команду для оформления упражнений 
\newcommand{\ex}[1]{%
\addtocounter{taskitem}{1} % увеличение счётчика на 1 
\noindent \textcolor{blue}{Задача \thesection.\arabic{taskitem} \\ \\}%
#1 \\}

\ex{Шестеро друзей пришли в театр! В ложе есть 22 места. Среди друзей три парня и три девушки. Сколько надо покупать виски?}

\ex{Шестеро друзей пришли в театр! В ложе есть 22 места. Среди друзей три парня и три девушки. Сколько надо покупать виски?}

\ex{Шестеро друзей пришли в театр! В ложе есть 22 места. Среди друзей три парня и три девушки. Сколько надо покупать виски?}

\section{Ещё дача}

\ex{Шестеро друзей пришли в театр! В ложе есть 22 места. Среди друзей три парня и три девушки. Сколько надо покупать виски?}


\section{Теоремы}

\begin{defin} 
\textbf{Биномиальным коэффициентом} называется выражение $C_n^k = \frac{n!}{k!(n-k)!}$
\end{defin} 

\begin{theorem}[Комбинаторное тождество]
\[
C_n^0 + C_n^1 + C_n^2 + \ldots + C_n^n = 2^n 
\]
\end{theorem} 
\begin{proof} 
Из бинома Ньютона очевидно, что 

\[
2^n = (1 + 1)^n = \sum_{k=0}^n C_n^k \cdot 1^k \cdot 1^{n-k} = \sum_{k=0}^n C_n^k
\]
\end{proof} 

\begin{result} 
Мощность множества всех подмножеств непустого множества равна $2^n$
\end{result} 

\begin{result} 
Никто не помнит дискретную математику на $10$ из $10$
\end{result} 

\todo[inline]{Для новой теоремы следствия будут нумероваться заново!}

\begin{theorem}
Если события $A$ и $B$ зависимы, тогда 

\[
P(A \cup B) = P(A) + P(B) - P(A \cap B)
\]
\end{theorem} 

\begin{result} 
\[ P(A \cup B) \ge P(A) + P(B) \] 
\end{result} 

\todo[inline]{Ниже идет фигня} 

\begin{fig} 
Каждый, кто любит пельмени, помнит бином Ньютона. 
\end{fig} 

\section{Свои окружения}
\subsection{Задача}

% \newenviroment{имя окружения}[число аргументов]{что должно быть до}{что должно быть после}

% счётчик для задач
\newcounter{iexer}

% задаём окружение 
\newenvironment{exer2}{% 
\addtocounter{iexer}{1}
\noindent \textbf{\large Задаченька \arabic{iexer} }% 
\vspace{2mm} 
} % сюда тех вставит текст
{% то, что будет после 
\begin{center} 
\begin{tikzpicture} 
\draw[blue](2,2) circle [radius = 0.5];
\end{tikzpicture} 
\end{center}
}


\begin{exer2} 
Шестеро друзей пришли в театр! В ложе есть 22 места. Среди друзей три парня и три девушки. Сколько надо покупать виски?
\end{exer2} 

% Ворованное окружение, которое рисует рамочки
%Defining a new environment
\newenvironment{boxed_2}[1]
    {\begin{center}
    #1\\[1ex]
    \begin{tabular}{|p{0.9\textwidth}|}
    \hline\\
    }
    { 
    \\\\\hline
    \end{tabular} 
    \end{center}
    }

\begin{boxed_2}{Title of the Box}
This is the text formatted by the boxed environment
\end{boxed_2}



\section{etoolbox}
\subsection{Скрываем кусок текста}

% одно из применений пакета - уничтожение какого-то кусочка текста 

\newbool{answer} 
% \booltrue{answer}
\boolfalse{answer}

\renewcommand{\ex}[2]{%
\addtocounter{taskitem}{1} % увеличение счётчика на 1 
\noindent \textcolor{blue}{Задача \thesection.\arabic{taskitem} \\ \\}%
#2 \\
\ifbool{answer}{\textbf{Ответ:} #1}{ }
% если {условие}{делай при правде}{делай при лжи}
}

\ex{много}{Шестеро друзей пришли в театр! В ложе есть 22 места. Среди друзей три парня и три девушки. Сколько надо покупать виски?}

\booltrue{answer}

\ex{много}{Шестеро друзей пришли в театр! В ложе есть 22 места. Среди друзей три парня и три девушки. Сколько надо покупать виски?}

\end{document}
