%!TEX TS-program = xelatex
\documentclass[12pt, a4paper]{article}  

%%%%%%%%%% Математика %%%%%%%%%%
\usepackage{amsmath,amsfonts,amssymb,amsthm,mathtools} 

%%%%%%%%%%%%%%% Шрифты %%%%%%%%%%%

\usepackage[british,russian]{babel}  % выбор языка для документа
\usepackage[utf8]{inputenc}          % задание utf8 кодировки исходного tex файла
\usepackage[X2,T2A]{fontenc}          % кодировка

\usepackage{fontspec}         % пакет для подгрузки шрифтов
\setmainfont{Arial}         % задаёт основной шрифт документа

\usepackage{unicode-math}     % пакет для установки математического шрифта
% шрифт для математики
\setmathfont{Asana-Math}


\usepackage{pgf,tikz,pgfplots}
\usetikzlibrary{arrows,calc}
\usepackage{relsize}
\usepackage{mathrsfs}
\pagestyle{empty}

\newcommand\LM{\ensuremath{\mathit{LM}}}
\newcommand\IS{\ensuremath{\mathit{IS}}}

% Заголовок
\author{Уютный факультатив} 
\title{Xe\LaTeX. Работа со шрифтами.}
\date{\today}

\begin{document}

\section{Украденная IS-LM} 

\begin{tikzpicture}[
        scale=2,
        IS/.style={blue, thick},
        LM/.style={red, thick},
        axis/.style={very thick, ->, >=stealth', line join=miter},
        important line/.style={thick}, dashed line/.style={dashed, thin},
        every node/.style={color=black},
        dot/.style={circle,fill=black,minimum size=4pt,inner sep=0pt,
            outer sep=-1pt},
    ]
    % axis
    \draw[axis,<->] (2.5,0) node(xline)[right] {$Y$} -|
                    (0,2.5) node(yline)[above] {$i$};
    % % IS-LM diagram
    % \draw[LM] (0.2,0.3) coordinate (LM_1) parabola (4,4)
    %     coordinate (LM_2) node[above] {\LM};
        
    \draw[IS] (0.2,1.8) coordinate (IS_1) parabola[bend at end]
         (1.8,.3) coordinate (IS_2) node[right] {\IS};
    %Intersection is calculated "manually" since Tikz does not offer
    %intersection calculation for parabolas
    \node[dot,label=above:$A$] at (1,.68) (int1) {};
    %shifted IS-LM diagram
    \draw[xshift=.7cm, LM, red!52] (0.2,0.2) parabola (1.8,1.7)
        node[above] {\LM'};
    \draw[xshift=.4cm, yshift=.3cm, IS, blue!60] (0.2,1.8)
        parabola[bend at end] (1.8,.3)
        node[right] {\IS'};
    %Intersection of shifted IS-LM
    \path[xshift=.36cm, yshift=.35cm] (.98,.7)
        node[dot,label=above:{$B$}] (int2) {};
    \path[xshift=.805cm] (1,.68) node[dot,label=above:$C$] (int3) {};
    %arrows between intersections
    \draw[->, very thick, black, >=stealth']
        ($(int1)+1/2*(-.80,1)$) -- ($(int2)+1/2*(-.8,1)$)
        node[sloped, above, midway] {$\mathsmaller{\Delta G > 0}$};
    \draw[->, very thick, black, >=stealth']
        ($(int2)+2*(.14,.2)$) -- ($(int2)!.2cm!270:(int2)+(.9,0)$)
        node[sloped,above, midway] {$\mathsmaller{\Delta M>0}$};
        
    \begin{scope}[xshift=4cm]
        %E-diagram
        \draw[axis,<->] (0,2.5) node(eyline)[above] {$i$} |-
                        (2.5,0) node(exline)[right] {$E$};

        \draw[important line, green, xshift=.5cm]
            (.2,.2) coordinate (es) -- (1.5,1.5) coordinate (ee)
            node [above right] {Interest rate parity};
    \end{scope}
    %Lines connecting IS LM coordinates and E coordinates
    \draw[dashed] 
        let
            % Store the intersection point in \p1 for later retrieval. 
            % A convenient feature of the let operation is that we can
            % access the x and y component of the coordinate directly 
            % using the \x1 and \y1 syntax. 
            \p1=(intersection of int2--[xshift=1]int2 and es--ee)
        in
            (0,\y1) node[left]{$i'$} -|  (\x1,0)
            node[pos=0.5,dot,label=above:$B'$] {} node[below] {$E'$};

    \draw[dashed line] let
        \p1=(intersection of int3--[xshift=1]int3 and es--ee)
            in
        (0,\y1) node[left]{$i\phantom{'}$} -| (\x1,0)
        node[dot,label=above:$C'$,pos=0.5] {} node[below] {$E$};
\end{tikzpicture}

\section{Основные команды Tikz}

\begin{center}
    \begin{tikzpicture}
        \draw (0,0) -- (1,1);
    \end{tikzpicture}
\end{center}

\vspace{2cm} 

\begin{tikzpicture}
    \draw[help lines] (0,0) grid (3,3);
\end{tikzpicture}

\vspace{2cm} 

\begin{tikzpicture}
    \draw[help lines] (0,0) grid (3,3);
    \draw[->] (0,0) -- (1,1);
    \draw[<->, thick, blue] (2,1) -- (0,3);
    \draw[<-, thick, dashed] (2,2) -- (3,3);
\end{tikzpicture}

\vspace{2cm} 

\begin{tikzpicture}
    \draw[help lines] (0,0) grid (3,3);
    % axes 
    \draw[<->, thick] (0,3.5) -- (0,0) -- (3.5,0);
    \draw (1.5,0.5) -- (2.5,1.5) -- (1.5,2.5) -- (0.5, 1.5) -- cycle;
\end{tikzpicture}

\vspace{2cm} 

\begin{tikzpicture}
    \draw[help lines] (0,0) grid (3,3);
    % axes 
    \draw[<->, thick] (0,3.5) -- (0,0) -- (3.5,0);
    \draw[thick, blue, fill=pink] (1.5,0.5) -- (2.5,1.5) -- (1.5,2.5) -- (0.5, 1.5) -- cycle;
\end{tikzpicture}

\vspace{2cm} 

\begin{tikzpicture}
 \foreach \x in {0,...,9}
 \draw [->] (\x, 0) -- (\x, -2);
\end{tikzpicture}


\vspace{2cm} 

\begin{tikzpicture}
    \draw[help lines] (0,0) grid (3,3);
    \draw (1.5, 2.0) circle (0.5);
    \draw (0.5, 0.5) rectangle (2, 1);
\end{tikzpicture}

\vspace{2cm} 

% Непонятно в чём ошибка :( 
% \begin{tikzpicture}
%     \draw[<->, thick] (0,3.5) -- (0,0) -- (3.5,0);
%     \draw[domain=0:2*pi] plot (\x, {cos(\x r)}) };
% \end{tikzpicture}

\begin{tikzpicture}[domain=0:4]
    \draw[very thin,color=gray] (-0.1,-1.1) grid (3.9,3.9);
    
    \draw[->] (-0.2,0) -- (4.2,0) node[right] {$x$};
    \draw[->] (0,-1.2) -- (0,4.2) node[above] {$f(x)$};
    
    \draw[color=red]    plot (\x,\x) node[right] {$f(x) =x$};
    
    \draw[color=blue]   plot (\x,{sin(\x r)})   node[right] {$f(x) = \sin x$};
    
    \draw[color=orange] plot (\x,{0.05*exp(\x)}) node[right] {$f(x) = \frac{1}{20} \mathrm e^x$};
\end{tikzpicture}

\section{Geogebra} 

\begin{center}
\definecolor{dcrutc}{rgb}{0.86,0.07,0.23}
\definecolor{ududff}{rgb}{0.30,0.30,1.}
\begin{tikzpicture}[line cap=round,line join=round,x=1.0cm,y=1.0cm]

\clip(0,0) rectangle (7,7);

\draw [line width=4.4pt,color=dcrutc] (2.54,1.96)-- (5.54,1.98);

\draw [line width=2.pt] (4.,3.) circle (3cm);

\begin{scriptsize}
\draw [fill=black] (3.,4.) circle (5pt);
\draw [fill=black] (5.,4.) circle (5pt);
\end{scriptsize}
\end{tikzpicture}
\end{center} 

\section{Примеры картинок}

\begin{figure}[H]
\begin{minipage}[H]{0.39\linewidth}
\begin{center}
\[ f(p) = \begin{cases}
             1&, p \in [0;1] \\
             0&, \text{иначе}\\
            \end{cases} \]
\end{center}
\end{minipage}
\hfill
\begin{minipage}[H]{0.59\linewidth}
\begin{center}
\begin{tikzpicture}
% оси
\draw [->] (-3.8,0) -- (4,0);
\draw [->] (0,0) -- (0,3.5);
% график
\draw [blue, thick, domain=0:2] plot (\x, 2);
\draw [->, blue, thick] (-3.8,0)--(-0.05,0);
\draw [<-, blue, thick] (2.05,0)--(4,0);
\draw [blue, thick,dashed] (2,0)--(2,2);
% точки
\draw[fill,blue] (2,2) circle [radius=0.03];
\draw[fill,blue] (0,2) circle [radius=0.03];
% подписи
\node [below] at (0,0) {0};
\node [below] at (2,0) {1};
\node [left] at (0,2) {1};
\node [below right] at (4,0) {$p$};
\node [left] at (0,3.3) {$f(p)$};
\end{tikzpicture}
\end{center}
\end{minipage}
\end{figure}

\vspace{2cm} 

\begin{center}
\definecolor{qqqqff}{rgb}{0.,0.,1.}
\definecolor{qqttcc}{rgb}{0.,0.2,0.8}
\definecolor{qqqqcc}{rgb}{0.,0.,0.8}

\begin{tikzpicture}[line cap=round,line join=round,>=triangle 45,x=1.0cm,y=1.0cm]
\clip(-0.7666616663680121,-3.3866122567793884) rectangle (11.148382983444685,3.1344594772397105);
\draw [rotate around={90.:(2.,0.)},line width=1.6pt,color=qqqqcc] (2.,0.) ellipse (2.7489679841754673cm and 1.8859546595880108cm);
\draw [rotate around={90.:(8.,0.)},line width=1.6pt,color=qqttcc] (8.,0.) ellipse (2.7489679841754686cm and 1.885954659588012cm);
\draw [->] (2.,2.) -- (7.8,2.);
\draw [->,line width=0.4pt] (7.8,2.) -- (2.,2.);
\draw [->,line width=0.4pt] (2.,1.2) -- (7.8,1.2);
\draw [->,line width=0.4pt] (7.8,1.2) -- (2.,1.2);
\draw [->,line width=0.4pt] (2.,0.2) -- (7.8,0.2);
\draw [->,line width=0.4pt] (7.8,0.2) -- (2.,0.2);
\draw [->,line width=0.4pt] (2.,-0.8) -- (7.8,-0.8);
\draw [->,line width=0.4pt] (7.8,-0.8) -- (2.,-0.8);
\draw (1.5,2.463567323534042) node[anchor=north west] {$\mathbf{1}$};
\draw (1.5,1.6584967390872398) node[anchor=north west] {$\mathbf{2}$};
\draw (1.5,0.6387406654546235) node[anchor=north west] {$\mathbf{3}$};
\draw (1.5,-0.35417972202976605) node[anchor=north west] {$\mathbf{4}$};
\draw (7.8,2.530656538904609) node[anchor=north west] {$\mathbf{?}$};
\draw (7.8,1.7121681113836933) node[anchor=north west] {$\mathbf{2}$};
\draw (7.8,0.7326655669734171) node[anchor=north west] {$\mathbf{4}$};
\draw (7.8,-0.2870905066591992) node[anchor=north west] {$\mathbf{6}$};
\draw [->,line width=0.4pt] (2.,-2.) -- (7.8,-2.);
\draw [->,line width=0.4pt] (7.8,-2.) -- (2.,-2.);
\draw (7.5,-1.19) node[anchor=north west] {$\dots$};
\draw (1.5814608716018326,-1.5483677556258562) node[anchor=north west] {$\mathbf{n}$};
\draw (7.8,-1.4946963833294027) node[anchor=north west] {\small{$\mathbf{2n-2}$}};
\draw (7.5,-2.22) node[anchor=north west] {$\dots$};
\draw (1.7,-1.19) node[anchor=north west] {$\dots$};
\draw (1.7,-2.22) node[anchor=north west] {$\dots$};
\begin{scriptsize}
\draw [fill=qqqqff] (2.,2.) circle (2.0pt);
\draw [fill=qqqqff] (2.,1.2) circle (2.0pt);
\draw [fill=qqqqff] (2.,0.2) circle (2.0pt);
\draw [fill=qqqqff] (2.,-0.8) circle (2.0pt);
\draw [fill=qqqqff] (7.8,2.) circle (2.0pt);
\draw [fill=qqqqff] (7.8,1.2) circle (2.0pt);
\draw [fill=qqqqff] (7.8,0.2) circle (2.0pt);
\draw [fill=qqqqff] (7.8,-0.8) circle (2.0pt);
\draw [fill=qqqqff] (2.,-2.) circle (2.0pt);
\draw [fill=qqqqff] (7.8,-2.) circle (2.0pt);
\end{scriptsize}
\end{tikzpicture}
\end{center}



\end{document}