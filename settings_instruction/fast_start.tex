%!TEX TS-program = xelatex
\documentclass[12pt, a4paper]{article}  % Любой документ начинается с такой строки! В ней мы выбираем размер шрифта, размер бумаги и класс документа. У каждого класса свои свойства!

% Знак процента используется для комментариев. Все, что написано под знаком процента, LaTeX не видит. 
%
%         Классы: 
% article     ---   статья
% report     ---   отчет
% book       ---   книга
% beamer    ---  презентация
%
% Каждый документ состоит из двух частей. Часть от \documentclass  до \begin{document} - преамбула. Часть до \end{document} - тело документа.
%
% В преамбуле находятся различные служебные команды. А именно:
% а) Команды, подключающие пакеты
% б) Команды, которые определяют вид документа в целом
% в) Команды, которые создают новые команды, чтобы удобнее использовать старые команды
% г) Ещё какие-нибудь другие команды


\usepackage{amsmath,amsfonts,amssymb,amsthm,mathtools}  % пакеты для математики

%%%%%%%%%%%%%%%%%%%%%%%% Шрифты %%%%%%%%%%%%%%%%%%%%%%%%%%%%%%%%%

\usepackage{fontspec}         % пакет для подгрузки шрифтов
\setmainfont{Arial}   % задаёт основной шрифт документа

\defaultfontfeatures{Mapping=tex-text}

% why do we need \newfontfamily:
% http://tex.stackexchange.com/questions/91507/
\newfontfamily{\cyrillicfonttt}{Arial}
\newfontfamily{\cyrillicfont}{Arial}
\newfontfamily{\cyrillicfontsf}{Arial}

\usepackage{unicode-math}     % пакет для установки математического шрифта
%\setmathfont{Asana Math}      % шрифт для математики

\usepackage[british,russian]{babel} % выбор языка для документа
\usepackage[utf8]{inputenc} % задание utf8 кодировки исходного tex файла

% Внимание! Все, кто использует кодировку cp1251 будут гореть в аду! Используйте только utf-8! 


\begin{document} % Тут заканчиваются служебные команды и начинается документ!

\section*{Четыре благородные истины}

\begin{enumerate}
\item Жизнь полна страданий.
\item Причина страданий сама жизнь с её страстями и желаниями.
\item Человек может устранить причину своих страданий, обуздав свои страсти и желания.
\item Есть путь, который ведёт к освобождению. И изучение \LaTeX{ } часть этого пути.
\end{enumerate}

Также, никогда не забывайте, что 
\[ \sum_{i=0}^{\infty} q^i = \frac{1}{1-q} \] 

\end{document}
