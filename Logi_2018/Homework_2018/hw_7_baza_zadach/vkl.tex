%!TEX TS-program = xelatex
\documentclass[12pt, a4paper]{article}

\input{preamble}


% Специальный пакет для оформления задач! 
\newtheorem{problem}{Задача}

\usepackage{answers}
\Newassociation{sol}{solution}{solution_file}
% sol --- имя окружения внутри задач
% solution --- имя окружения внутри solution_file
% solution_file --- имя файла в который будет идти запись решений


\begin{document}

% Открываем файл, куда будут записываться решения. 
\Opensolutionfile{solution_file}[all_solutions]


\begin{problem}
В России 1112 городов (наверное). Ярополк посетил половину. Мирослав тоже. Эти половины не совпадают. Оба они побывали в четверти этих городов. Сколько городов не посетил ни один из них?
\begin{sol}

\end{sol}
\end{problem}



\begin{problem}
Попечительский совет выделил ученикам 10 класса денежные средства на сбор статистики. На эти деньги ученики должны были купить фломастеры, бумагу и т.п. Ученики собрали информацию следующего характера. В школе 45 учеников, в том числе 25 мальчиков. На хорошо и отлично учатся 30 учеников, в том числе 16 мальчиков. Спортом заняты 28 учеников, среди которых 18 мальчиков и 17 учеников, учащихся на отлично и хорошо. 15 мальчиков учатся на отлично и хорошо и занимаются спортом.
Сторож Афанасий, совершавший вечерний обход помещений, заподозрил что-то неладное, заметив за одной из батарей в школьном коридоре спрятанную пустую бутылку с надписью "Хортица". Об этом он сообщил директрисе. После тщательной проверки собранной статистики ученики 10 класса получили выговор. Что именно нашла директриса в отчёте 10-классников? 
\begin{sol}

\end{sol}
\end{problem}



\begin{problem}
Сколько существует способов расселить 5 туристов по 3 домикам так, чтобы ни один домик не оказался пустым? Все туристы и домики различны. Способы расселения отличающиеся только перестановкой туристов, заселённых в один домик, считаются одинаковыми.
\begin{sol}

\end{sol}
\end{problem}



\begin{problem}
В институте работают 100 сотрудников. Английский знают 60 из них. Немецкий не знают 55 из них. Французский знают 25 из них. 25 не знают ни французского, ни английского. Английский и немецкий знают 15. Французский и немецкий знают 15. Все три языка знают 5. 
\begin{itemize}
\item Сколько человек знают немецкий?
\item Сколько человек знают английсий и французский? 
\item Сколько человек знают хотя бы один язык?
\item Сколько человек знают немецкий и французский, но не знают английского?
\item Сколько человек знают французский, но не знают ни английского ни немецкого?
\end{itemize}
\begin{sol}

\end{sol}
\end{problem}



\begin{problem}
Эконом играет в киллера. Всего играет $n$ человек. Каждый получает конверт со своей жертвой. Сколько существует способов раздать конверты так, чтобы ни один человек не получил в качестве своей жертвы себя? 
\begin{sol}

\end{sol}
\end{problem}



\begin{problem}
Три художника пытаются воссоздать картину Ван Гога. Художники встретились, когда у каждого было готово не менее $^2/_3$ картины. Может ли быть, что у каждых двоих общая часть нарисованного составляет не более $^1/_4$ исходной картины?
\begin{sol}

\end{sol}
\end{problem}



\begin{problem}
Переплётчик должен переплести 12 различных книг в красный, зелёный и коричневый переплёты. Сколькими способами он может это сделать, если в каждый цвет должна быть переплетена хотя бы одна книга? (Подсказка: смотри упражнение 3!)
\begin{sol}

\end{sol}
\end{problem}



\begin{problem}
Сколько целых чисел от 0 до 999 не делятся ни на 5 ни на 7? А сколько чисел от 0 до 999 не делятся ни на 2, ни на 3, ни на 5, ни на 7?
\begin{sol}

\end{sol}
\end{problem}



\begin{problem}
Сколько целых чисел от 100 до 1000 не делятся ни на одно из чисел 6, 9 и 15? Делятся ровно на одно из этих чисел? Делятся ровно на два из них? 
\begin{sol}

\end{sol}
\end{problem}





% Закрываем файл, куда мы записывали решения и вставляем его в конце списка задач. 
\Closesolutionfile{solution_file}

% Вставляем решения. Можно их не вставлять или настроить пакет так, чтобы они шли непосредственно после каждой задачи.
% \begin{solution}{1}
\end{solution}
\begin{solution}{3}
\end{solution}
\begin{solution}{4}
\end{solution}



\end{document}
