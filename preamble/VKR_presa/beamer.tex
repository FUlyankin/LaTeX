%!TEX TS-program = xelatex

% Данный шаблон подготовлен для курса LaTeX в РАНХиГС
% на основе шаблона 
% Данилы Фёдоровых (danil@fedorovykh.ru),
%  который использовал его в курсе 
% <<Документы и презентации в \LaTeX>> НИУ ВШЭ
% Исходная версия шаблона --- 
% https://www.writelatex.com/coursera/latex/5.1
\documentclass[c, dvipsnames]{beamer}  % [t], [c], или [b] --- вертикальное 
%\documentclass[handout, dvipsnames, c]{beamer} % Раздаточный материал (на слайдах всё сразу)
%выравнивание на слайдах (верх, центр, низ)
%\documentclass[handout, dvipsnames]{beamer} % Раздаточный материал (на слайдах всё сразу)
%\documentclass[aspectratio=169, dvipsnames]{beamer} % Соотношение сторон
\setbeamertemplate{navigation symbols}{}%remove navigation symbols

%\usetheme{Berkeley} % Тема оформленияLLL
%\usetheme{Bergen}
%\usetheme{CambridgeUS}
\usetheme{Boadilla}

\usecolortheme{crane} % Цветовая схема

%\useoutertheme{infolines} % Навигация 
%\useoutertheme{tree}
%\useoutertheme{miniframes}
%\useoutertheme{shadow}
%\useoutertheme{sidebar}
%\useoutertheme{smoothbars}
%\useoutertheme{smoothtree}
%\useoutertheme{split}
%\useoutertheme{default}


%\useinnertheme{circles}
\useinnertheme{rectangles}
%\useinnertheme{rounded}
%\useinnertheme{inmargin}


%%% Работа с русским языком
\usepackage[english,russian]{babel}   %% загружает пакет многоязыковой вёрстки
\usepackage{fontspec}      %% подготавливает загрузку шрифтов Open Type, True Type и др.
\defaultfontfeatures{Ligatures={TeX},Renderer=Basic}  %% свойства шрифтов по умолчанию
\setmainfont[Ligatures={TeX,Historic}]{Arial} %% задаёт основной шрифт документа
\setsansfont{Arial}                    %% задаёт шрифт без засечек
\setmonofont{Arial}
\usepackage{indentfirst}
\frenchspacing



%% Beamer по-русски
\newtheorem{rtheorem}{Теорема}
\newtheorem{rproof}{Доказательство}
\newtheorem{rexample}{Пример}

%%% Дополнительная работа с математикой
\usepackage{amsmath,amsfonts,amssymb,amsthm,mathtools} % AMS
\usepackage{icomma} % "Умная" запятая: $0,2$ --- число, $0, 2$ --- перечисление

%% Номера формул
\mathtoolsset{showonlyrefs=true} % Показывать номера только у тех формул, на которые есть \eqref{} в тексте.
%\usepackage{leqno} % Нумерация формул слева

%% Свои команды
\DeclareMathOperator{\sgn}{\mathop{sgn}}

%% Перенос знаков в формулах (по Львовскому)
\newcommand*{\hm}[1]{#1\nobreak\discretionary{}
{\hbox{$\mathsurround=0pt #1$}}{}}

%%% Работа с картинками
\usepackage{graphicx}  % Для вставки рисунков
\graphicspath{{images/}{images2/}}  % папки с картинками
\setlength\fboxsep{3pt} % Отступ рамки \fbox{} от рисунка
\setlength\fboxrule{1pt} % Толщина линий рамки \fbox{}
\usepackage{wrapfig} % Обтекание рисунков текстом

%%% Работа с таблицами
\usepackage{array,tabularx,tabulary,booktabs} % Дополнительная работа с таблицами
\usepackage{longtable}  % Длинные таблицы
\usepackage{multirow} % Слияние строк в таблице

%%% Программирование
\usepackage{etoolbox} % логические операторы

%%% Другие пакеты
\usepackage{lastpage} % Узнать, сколько всего страниц в документе.
%\usepackage{soul} % Модификаторы начертания
\usepackage{csquotes} % Еще инструменты для ссылок
\usepackage{multicol} % Несколько колонок


\usepackage{hyperref}
\usepackage{xcolor}
\hypersetup{        % Гиперссылки
    unicode=true,           % русские буквы в раздела PDF
    pdftitle={Заголовок},   % Заголовок
    pdfauthor={Автор},      % Автор
    pdfsubject={Тема},      % Тема
    pdfcreator={Создатель}, % Создатель
    pdfproducer={Производитель}, % Производитель
    pdfkeywords={keyword1} {key2} {key3}, % Ключевые слова
    colorlinks=true,        % false: ссылки в рамках; true: цветные ссылки
    linkcolor=,          % внутренние ссылки
    citecolor=green,        % на библиографию
    filecolor=magenta,      % на файлы
    urlcolor=blue           % на URL
} 

\usepackage{dcolumn}

%fffff3
\definecolor{backgr}{RGB}{146,26,29}
\definecolor{backgr1}{RGB}{230,43,37}
\definecolor{ex1}{RGB}{231,142,36}
\definecolor{ex2}{RGB}{249,155,28}
\definecolor{ex3}{RGB}{242,103,36}

\definecolor{red}{RGB}{230,43,37}
%\setbeamercolor{normal text}{fg=black,bg=backgr}
\setbeamercolor{frametitle}{bg=backgr,fg=white}
%\setbeamercolor{footline}{bg=backgr,fg=white}
%\setbeamercolor{normal text}{bg=yellow}
%\setbeamercolor{section in toc}{fg=yellow}
%\setbeamercolor{subsection in toc}{fg=blue}

% How to change colour of Navigation Bar in Beamer -  много интересного

%Пример команд, задающих внешний вид блока
\setbeamercolor{block title}{fg=white,bg=ex1}
\setbeamerfont{block title}{family=\sffamily}
\setbeamercolor{block body}{bg=white}
\setbeamertemplate{blocks}[rounded][shadow=fasle]
\setbeamercolor{title}{bg=backgr, fg=white}
\setbeamercolor{alerted text}{fg=backgr1}

\newlength\subtitwd
\setlength\subtitwd{4cm}% change the width here

\makeatletter
\newcommand\titlegraphicii[1]{\def\inserttitlegraphicii{#1}}
\titlegraphicii{}
\newcommand\superviser[1]{\def\insertsuperviser{Научный руководитель: #1}}
\superviser{}

\setbeamertemplate{title page}
{
  \vbox{}
   {\usebeamercolor[fg]{titlegraphic} \hspace{0.35ex} \inserttitlegraphic\hfill\inserttitlegraphicii \hspace{1ex} \par }\vspace{1.5ex}
  \begin{centering}
    \begin{beamercolorbox}[sep=8pt,center]{institute}
      \usebeamerfont{institute}\insertinstitute
    \end{beamercolorbox}
    \begin{beamercolorbox}[sep=8pt,center]{title}
    
      \usebeamerfont{title}\inserttitle\par%
      \ifx  \insertsubtitle\@empty%
      \else%
        \vskip0.5em%
        {\usebeamerfont{subtitle}\usebeamercolor[fg]{subtitle}\insertsubtitle\par}%
      \fi%     
    \end{beamercolorbox}%
    \vskip1em\par
    \begin{beamercolorbox}[sep=5pt,center]{date}
      \usebeamerfont{date}\insertdate
    \end{beamercolorbox}%\vskip0.5em
    \begin{beamercolorbox}[sep=5pt,center]{author}
      \usebeamerfont{author}\insertauthor
    \end{beamercolorbox}
        \begin{beamercolorbox}[sep=4pt,center]{institute}
      \usebeamerfont{institute}\insertsuperviser
    \end{beamercolorbox}
  \end{centering}
  %\vfill
}
\makeatother

\setbeamercolor{item projected}{bg=ex3}
\setbeamertemplate{enumerate items}[default]

\setbeamercolor{palette primary}{bg=white}
\setbeamercolor{palette primary}{fg=black}
\setbeamercolor{palette secondary}{bg=white}
\setbeamercolor{palette secondary}{fg=black}
\setbeamercolor{palette tertiary}{bg=white}
\setbeamercolor{palette tertiary}{fg=black}

\setbeamercolor{itemize item}{fg=ex3}
\setbeamercolor{itemize subitem}{fg=ex2}
\setbeamercolor{itemize subsubitem}{fg=ex1}

\setbeamercolor{enumerate item}{fg=ex3}
\setbeamercolor{enumerate subitem}{bg=ex3}
\setbeamercolor{enumerate subsubitem}{bg=ex3}


\setbeamertemplate{itemize subitem}{$\Rightarrow$}
\setbeamertemplate{itemize item}{$\blacktriangleright$}



\usepackage{todo}
\newcolumntype{a}{>{\columncolor{red}}c}


\usefonttheme{professionalfonts}

\title[Эмпирический анализ моделей эк. роста ]{Эмпирический анализ выводов \\ моделей экономического роста}
\subtitle{Защита выпускной квалификационной работы}


\author[Александр Тишин]{Александр Тишин \\ \smallskip \scriptsize ЭО-13-02 \\ \smallskip \scriptsize \href{mailto:phenyard@gmail.com}{\nolinkurl{phenyard@gmail.com} }}

\superviser{к.э.н. Перевышин Ю.Н.}

%\author[Имя автора]{Имя автора \\ \smallskip \scriptsize \href{mailto:author@ranepa.ru}{author@ranepa.ru} \\ \smallskip  \href{http://ranepa.ru}{http://ranepa.ru} }

\institute[РАНХиГС]{ \uppercase{
  Российская Академия Народного Хозяйства и  \\ Государственной Службы при Президенте Российской Федерации}}
\date{23 июня 2017 г.}


\titlegraphic{\includegraphics[scale=0.5]{logo1}}
\titlegraphicii{\includegraphics[scale=0.5]{logo2}}

\begin{document}

\frame[plain]{\titlepage}	% Титульный слайд


\begin{frame}[c]{Актуальность исследования} 
\begin{itemize}
	\item  Зуб даю, актуально! 
	\item  Ничего нет актуальнее!!!
	\item  Актуальность --- зашкаливает
	\item  Актуальная актуальность с актуальным актуалитетом.
\end{itemize}
\end{frame}



\begin{frame}[c]
\frametitle{Анализ предметной области}
{ \small   % вместо small можно поставить scriptsize чтобы влезло
	\begin{table}[]
		\centering
		\resizebox{\textwidth}{!}{ 
			\begin{tabular}{|p{2.2cm}|p{1.8cm}|p{3.5cm}|p{7cm}|}
				\hline\rowcolor{backgr}
				\textcolor{white}{Авторы} & \textcolor{white}{Выборка, период}  & \textcolor{white}{Метод исследования}& \textcolor{white}{Результат} \\			
				\hline
				(Kuper, 2003)  & США, 998-2008  &  Коинтеграци и VECM & Получились значимые результаты с интересной интерпретацией.\\
				\hline
		\end{tabular} }
	\end{table}
}
\end{frame}


\begin{frame}[shrink=3]
	\frametitle{Цели и задачи}
	\begin{block}{Цель:}
	\begin{itemize}
		\item Единственная цель
	\end{itemize}
		
	\end{block}

	 	\begin{block}{Задачи:}
			\begin{enumerate}
	\item Найти красивый шаблон
	\item Заручиться поддержкой научного руководителя 
	\item Ходить на научные семинары
	\item ?????
	\item Profit
	 \end{enumerate}	
	\end{block}
\end{frame}



\section{Неоклассические модели экономического роста}
\subsection{Страны, пространственная выборка}

\begin{frame}[shrink=5]
\frametitle{\insertsection} 
\framesubtitle{\insertsubsection}
Модель (Mankiw N. G., Romer D., Weil D. N., 1992): 
\begin{equation} 
\log(y_i)=a+\frac{\alpha}{1-\alpha}\log((s_k)_i)-\frac{\alpha}{1-\alpha}\log(n_i+g+\delta)+\varepsilon_i
\end{equation}
где: 
\begin{itemize}
\item $y$ "--- ВВП на душу населения по ППС  в долларах США в 2013 гг.
\item $s_k$ "--- средняя доля инвестиций в  ВВП (Gross capital formation), так как $I=S$
\item $n$ "--- средний темп прироста  численности населения  ($g+\delta=5\%$)
\end{itemize}
Рассматриваются три группы стран:
\begin{itemize}
\item Высокий доход, не нефтедобывающие~---~68 стран 
\item Средний доход --- 67 страны
\item Высокий доход, входящие в ОЭСР --- 22 страны
\end{itemize}
Отличия:
\begin{itemize}
	\item Непересекающийся с MRW временной интервал с 1990 по 2013 гг.

\end{itemize}
\end{frame}

\begin{frame}
 			\frametitle{\insertsection}
	\framesubtitle{\insertsubsection}
	\vspace{-1ex}
	\begin{itemize}
		\item \alert{ Значимые коэффициенты }для первой и второй групп стран
		\item \alert{Предполагаемые теорией знаки:} отрицательный перед темпами прироста населения и положительный перед долей инвестиций 
		\item Достаточно высокий $R^2_{adj}$
	\end{itemize}
	\vspace{-2ex}
	
\begin{table}[!htbp] \centering 
\resizebox{0.9\textwidth}{!}{ 
\begin{tabular}{@{\extracolsep{5pt}}lD{.}{.}{-2} D{.}{.}{-2} D{.}{.}{-2} } 
\\[-1.8ex]\hline 
\hline \\[-1.8ex] 
\\[-1.8ex] & \multicolumn{3}{c}{Зависимая переменная: ВВП ($y$)} \\ 
 & \multicolumn{1}{c}{Первая группа} & \multicolumn{1}{c}{Вторая группа} & \multicolumn{1}{c}{Третья группа} \\ 
\hline \\[-1.8ex] 
 $\log(n+g+\delta)$ & -7.68^{***} & -6.60^{***} & -0.96 \\ 
  & (0.82) & (0.87) & (1.07) \\ 
  $\log(s_k)$ & 2.04^{***} & 1.53^{***} & 1.20 \\ 
  & (0.48) & (0.57) & (1.15) \\ 
  Constant & 17.28^{***} & 17.01^{***} & 8.58^{**} \\ 
  & (2.31) & (2.55) & (3.62) \\ 
 \textit{N} & \multicolumn{1}{c}{86} & \multicolumn{1}{c}{67} & \multicolumn{1}{c}{22} \\ 

Adjusted R$^{2}$ & \multicolumn{1}{c}{0.58} & \multicolumn{1}{c}{0.51} & \multicolumn{1}{c}{-0.02} \\ 

\hline 
\hline \\[-1.8ex] 

\end{tabular} }
\end{table} 
\end{frame}



\subsection{Страны, панель}



\begin{frame}[shrink=12]
\frametitle{\insertsection} 
\framesubtitle{\insertsubsection}
Модель панельной регрессии с фиксированными временными эффектами: 
\begin{equation} 
\log(y_{i,t})=a_t+\frac{\alpha}{1-\alpha}\log((s_k)_{i,t})-\frac{\alpha}{1-\alpha}\log(n_{i,t}+g+\delta)+\varepsilon_{i,t}
\end{equation}
где 
\begin{itemize}
\item $y$ "--- средний ВВП на душу населения по ППС  в долларах США в постоянных ценах  за шесть лет
\item $s_k$ "--- средняя доля инвестиций в  ВВП  за шесть лет
\item $n$ "--- средний темп прироста  численности населения  за шесть лет
\end{itemize}
Рассматриваются три группы стран:
\begin{itemize}
\item Высокий доход, не нефтедобывающие~---~68 стран 
\item Средний доход --- 67 страны
\item Высокий доход, входящие в ОЭСР --- 22 страны
\end{itemize}
Построение панелей:
\begin{itemize}
	\item Данные с 1990 по 2013 гг. усреднялись по 6 лет $\Rightarrow$ 4 временных промежутка 
\end{itemize}

\end{frame}

\begin{frame}
 			\frametitle{\insertsection}
	\framesubtitle{\insertsubsection}
	\begin{itemize}
		\item Результаты \alert{ совпадают с пространственным анализом}: статистически значимые коэффициенты и предполагаемые теорией знаки
	\end{itemize}
	\vspace{-2ex}
\begin{table}[!htbp] \centering 
\resizebox{\textwidth}{!}{ 
\begin{tabular}{@{\extracolsep{5pt}}lD{.}{.}{-2} D{.}{.}{-2} D{.}{.}{-2} } 
\\[-1.8ex]\hline 
\hline \\[-1.8ex] 
\\[-1.8ex] & \multicolumn{3}{c}{Зависимая переменная: ВВП ($y$)} \\ 
 & \multicolumn{1}{c}{Первая группа} & \multicolumn{1}{c}{Вторая группа} & \multicolumn{1}{c}{Третья группа} \\ 
\hline \\[-1.8ex] 
 $\log(n+g+\delta)$ & -5.52^{***} & -5.36^{***} & -0.79^{*} \\ 
  & (0.42) & (0.45) & (0.41) \\ 
  $\log(s_k)$ & 1.43^{***} & 1.08^{***} & 1.05^{**} \\ 
  & (0.22) & (0.26) & (0.43) \\ 
 \textit{N} & \multicolumn{1}{c}{359} & \multicolumn{1}{c}{277} & \multicolumn{1}{c}{88} \\ 
Adjusted R$^{2}$ & \multicolumn{1}{c}{0.38} & \multicolumn{1}{c}{0.37} & \multicolumn{1}{c}{0.02} \\ 
\hline 
\hline \\[-1.8ex] 
\end{tabular} }
\end{table}
\end{frame}

\section{Выводы по неоклассическим моделям}
\begin{frame}
	\frametitle{\insertsection}
	Страны:
	\begin{itemize}
		\item Норма сбережений физического и человеческого капитала и темп прироста населения являются факторами объясняющими различия в уровне подушевого дохода 
		\item Модель Солоу объясняет $60\%-70\%$ вариации подушевого дохода для первой и второй групп стран. Добавление человеческого капитала улучшает выводы (\textit{уменьшаются коэффициенты})
	\end{itemize}	
	Регионы 
	\begin{itemize}
		\item Свидетельств  того, что различия подушевого дохода регионов объясняются за счет факторов неоклассических моделей экономического роста не выявлено
		\item Возможные объяснения: нарушены предпосылки моделей, например, закрытая экономика 
	\end{itemize}
\end{frame}
 
\section{Конвергенция}


\begin{frame}
\frametitle{\insertsection} 
 
\begin{block}{Безусловная $\beta$-конвергенция }
 \begin{equation}
\dfrac{1}{T}\log(\dfrac{y_{i,T}}{y_{i,0}})=\alpha-\beta \log(y_{i,0})+\varepsilon_{i}
 \end{equation} 
 Бедные страны растут быстрее богатых, и разница в уровнях подушевого дохода постепенно снижается независимо от  характеристик экономики
 \end{block}
 \begin{block}{Условная $\beta$-конвергенция}
  \begin{equation}
\dfrac{1}{T}\log(\dfrac{y_{i,T}}{y_{i,0}})=\alpha-\beta \log(y_{i,0})+\gamma X+ \varepsilon_{i}
 \end{equation} 
 Бедные страны растут быстрее богатых при условиях одинаковых характеристик экономики 
  \end{block}
  
  $y_i$ --- подушевой доход для $i$ субъекта,  $0$ --- в начальный момент времени и $T$ --- в конечный момент времени


\end{frame}


\subsection{Страны, пространственная выборка, безусловная конвергенция}

\frame<beamer:0>{
% Можно и вернуть обратно
 			\frametitle{\insertsection}
	\framesubtitle{\insertsubsection}
		\begin{itemize}
		\item \alert{Не отвергается гипотеза о  безусловной конвергенции} для первой и второй групп: статистически значимые отрицательные коэффициенты 
        
	\end{itemize}
\begin{table}[!htbp] \centering 
\resizebox{\textwidth}{!}{ 
\begin{tabular}{@{\extracolsep{5pt}}lD{.}{.}{-2} D{.}{.}{-2} D{.}{.}{-2} } 
\\[-1.8ex]\hline 
\hline \\[-1.8ex] 
\\[-1.8ex] & \multicolumn{3}{c}{Зависимая переменная: средний темп роста ВВП} \\
 & \multicolumn{1}{c}{Первая группа} & \multicolumn{1}{c}{Вторая группа} & \multicolumn{1}{c}{Третья группа} \\ 
\hline \\[-1.8ex] 
 $\log(y_{1990})$ & -0.14^{***} & -0.15^{***} & -0.22 \\ 
  & (0.04) & (0.04) & (0.20) \\ 
  Constant & 1.76^{***} & 1.89^{***} & 2.43 \\ 
  & (0.31) & (0.37) & (2.11) \\ 
 \textit{N} & \multicolumn{1}{c}{85} & \multicolumn{1}{c}{69} & \multicolumn{1}{c}{22} \\ 
Adjusted R$^{2}$ & \multicolumn{1}{c}{0.14} & \multicolumn{1}{c}{0.15} & \multicolumn{1}{c}{0.01} \\ 
\hline 
\hline \\[-1.8ex] 
\end{tabular} }
\end{table} 
}

\subsection{Страны, панель, безусловная конвергенция}

\begin{frame}
 			\frametitle{\insertsection}
	\framesubtitle{\insertsubsection}
	\begin{itemize}
 
		\item Статистически значимые и отрицательные  коэффициенты для всех групп $\Rightarrow$ \alert{безусловная конвергенция}
		\item В теории из безусловной конвергенции следует условная, подробнее в раздаточных материалах 
			\item Лучше пространственной регрессии
	\end{itemize}
	\vspace{-3ex}
\begin{table}[!htbp] \centering 
	\resizebox{\textwidth}{!}{ 
\begin{tabular}{@{\extracolsep{5pt}}lD{.}{.}{-2} D{.}{.}{-2} D{.}{.}{-2} } 
\\[-1.8ex]\hline 
\hline \\[-1.8ex] 
\\[-1.8ex] & \multicolumn{3}{c}{Зависимая переменная: средний темп роста ВВП} \\
 & \multicolumn{1}{c}{Первая группа} & \multicolumn{1}{c}{Вторая группа} & \multicolumn{1}{c}{Третья группа} \\ 
\hline \\[-1.8ex] 
 $\log(y_{0})$ & -0.10^{***} & -0.12^{***} & -0.51^{**} \\ 
  & (0.03) & (0.04) & (0.23) \\ 
 \textit{N} & \multicolumn{1}{c}{306} & \multicolumn{1}{c}{242} & \multicolumn{1}{c}{71} \\ 
Adjusted R$^{2}$ & \multicolumn{1}{c}{0.02} & \multicolumn{1}{c}{0.02} & \multicolumn{1}{c}{0.01} \\ 
\hline 
\hline \\[-1.8ex] 
\end{tabular} }
\end{table}
\end{frame}

\section{Выводы по конвергенции}
\begin{frame}
\frametitle{\insertsection}
 
\begin{itemize}
	\item Конвергенция по странам не отвергается для всех трех групп (панель)
	\item Конвергенция по регионам не отвергается  для обеих групп 
	\item Таким образом, как на страновом уровне, так и на региональном есть <<конвергенционные клубы>>, подушевой доход их <<участников>> имеет тенденцию к сближению 
\end{itemize} 
\end{frame}
 

\begin{frame}[c, plain]
\begin{center}

{\LARGE Спасибо за внимание}

\bigskip

{\Large \inserttitle}

\bigskip

{\insertauthor} 

\bigskip

\bigskip\bigskip

{\large \insertdate}
\end{center}
\end{frame}

\section{Приложения}
\begin{frame}[plain,c]
\begin{center}
\Huge{\textbf{
	ПРИЛОЖЕНИЯ}}
\end{center}
	
\end{frame}


\begin{frame}
	\frametitle{\insertsection}
	\framesubtitle{Группы регионов }
\begin{table}[ht]
	\centering
	\resizebox{\textwidth}{!}{ 
		\begin{tabular}{|lr|lr|lr|}
 \hline
 регион & N & регион & N & регион & N \\
 \hline
 Белгородская область & 1 & Брянская область & 2 & Республика Марий Эл & 2 \\ Липецкая область & 1 & Владимирская область & 2 & Республика Мордовия & 2 \\ Московская область & 1 & Воронежская область & 2 & Удмуртская Республика & 2 \\ Республика Коми & 1 & Ивановская область & 2 & Чувашская Республика & 2 \\ Вологодская область & 1 & Калужская область & 2 & Кировская область & 2 \\ Ленинградская область & 1 & Костромская область & 2 & Нижегородская область & 2 \\ Новгородская область & 1 & Курская область & 2 & Пензенская область & 2 \\ г.Санкт-Петербург & 1 & Орловская область & 2 & Саратовская область & 2 \\ Краснодарский край & 1 & Рязанская область & 2 & Ульяновская область & 2 \\ Астраханская область & 1 & Смоленская область & 2 & Курганская область & 2 \\ Республика Башкортостан & 1 & Тамбовская область & 2 & Республика Алтай & 2 \\ Республика Татарстан & 1 & Тверская область & 2 & Республика Бурятия & 2 \\ Пермский край & 1 & Тульская область & 2 & Республика Тыва & 2 \\ Оренбургская область & 1 & Ярославская область & 2 & Республика Хакасия & 2 \\ Самарская область & 1 & Республика Карелия & 2 & Алтайский край & 2 \\ Свердловская область & 1 & Калининградская область & 2 & Новосибирская область & 2 \\ Челябинская область & 1 & Мурманская область & 2 & Омская область & 2 \\ Иркутская область & 1 & Псковская область & 2 & Приморский край & 2 \\ Кемеровская область & 1 & Республика Адыгея & 2 & Хабаровский край & 2 \\ Томская область & 1 & Республика Калмыкия & 2 & Амурская область & 2 \\ Республика Саха (Якутия) & 1 & Волгоградская область & 2 & Магаданская область & 2 \\ Чукотский автономный округ & 1 & Ростовская область & 2 & Еврейская автономная область & 2 \\
 \hline
	\end{tabular} }
\end{table}

\end{frame}



\begin{frame}
\frametitle{\insertsection}
	\framesubtitle{Оценки $\alpha$ и $\beta$ }
	\begin{table}[ht]
\centering 
	\resizebox{\textwidth}{!}{ 
\begin{tabular}{|l|c|c|c|c|c|c|}
  \hline
 Модель & \multicolumn{2}{c|}{Первая группа} & \multicolumn{2}{c|}{Вторая группа} & \multicolumn{2}{c|}{Третья группа} \\ 
Солоу & GDP PPP & GDP PC & GDP PPP & GDP PC & GDP PPP & GDP PC \\ 
  \hline
Оценка $\alpha$ & 0.75 & 0.78 & 0.71 & 0.76 & 0.27 & 0.52 \\ 
  Левая граница & 0.70 & 0.73 & 0.64 & 0.68 & -2.40 & -1.58 \\ 
  Правая граница & 0.79 & 0.82 & 0.76 & 0.81 & 0.59 & 0.73 \\ 
   \hline
\end{tabular} }
\end{table}

\begin{table}[ht]
\centering
	\resizebox{\textwidth}{!}{ 
\begin{tabular}{|l|c|c|c|c|c|c|}
  \hline
Модель Солоу & \multicolumn{2}{c|}{Первая группа} & \multicolumn{2}{c|}{Вторая группа} & \multicolumn{2}{c|}{Третья группа} \\ 
с чел. капиталом & GDP PPP & GDP PC & GDP PPP & GDP PC & GDP PPP & GDP PC \\ 
  \hline
Оценка $\alpha$ & 0.46 & 0.45 & 0.37 & 0.34 & -0.14 & 0.19 \\ 
  Левая граница & 0.37 & 0.33 & 0.22 & 0.12 & 2.47 & 1.78 \\ 
  Правая граница & 0.51 & 0.51 & 0.45 & 0.45 & 0.43 & 0.56 \\ 
   \hline
\end{tabular} }
\end{table}

\begin{table}[ht]
\centering
	\resizebox{\textwidth}{!}{ 
\begin{tabular}{|l|c|c|c|c|c|c|}
  \hline
Модель Солоу & \multicolumn{2}{c|}{Первая группа} & \multicolumn{2}{c|}{Вторая группа} & \multicolumn{2}{c|}{Третья группа} \\ 
с чел. капиталом & GDP PPP & GDP PC & GDP PPP & GDP PC & GDP PPP & GDP PC \\ 
  \hline
Оценка $\beta$ & 0.26 & 0.30 & 0.31 & 0.37 & 0.20 & 0.20 \\ 
  Левая граница & 0.25 & 0.29 & 0.30 & 0.34 & 0.20 & 0.20 \\ 
  Правая граница & 0.26 & 0.32 & 0.33 & 0.44 & 0.21 & 0.21 \\ 
   \hline
\end{tabular} }
\end{table}
	
\end{frame}

\end{document}