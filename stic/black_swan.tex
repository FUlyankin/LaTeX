\documentclass{standalone}

\usepackage{amsmath,amsfonts,amssymb,amsthm,mathtools} 
\usepackage{fontspec}            % пакет для подгрузки шрифтов
\setmainfont{SF UI Text}   % задаёт основной шрифт документа

% why do we need \newfontfamily:
% http://tex.stackexchange.com/questions/91507/
\newfontfamily{\cyrillicfonttt}{SF UI Text}
\newfontfamily{\cyrillicfont}{SF UI Text}
\newfontfamily{\cyrillicfontsf}{SF UI Text}
% Иногда тех не видит структуры шрифтов. Эти трое бравых парней спасают ситуацию и доопределяют те куски, которые Тех не увидел.

\usepackage{unicode-math}     % пакет для установки математического шрифта
\setmathfont[math-style=upright]{Asana Math}      % шрифт для математики

\usepackage{polyglossia}      % Пакет, который позволяет подгружать русские буквы
\setdefaultlanguage{russian}  % Основной язык документа
\setotherlanguage{english}    % Второстепенный язык документа

\usepackage{pgf,tikz,pgfplots}
\usetikzlibrary{arrows,calc}
\usepackage{relsize} 

\usepackage{graphicx} 
\usepackage{rotating}
\usepackage{xcolor}
\usepackage{color}

%\newcommand{\Big}{\fontsize{50}{60}\selectfont Foo}

\definecolor{bl}{HTML}{333333}
\definecolor{rred}{HTML}{B32A28}

\pgfmathdeclarefunction{gauss}{2}{%
  \pgfmathparse{1/(#2*sqrt(2*pi))*exp(-((x-#1)^2)/(2*#2^2))}%
}
\begin{document}

\centering

\begin{tikzpicture}[scale=2]
% picture with bayes
%\node[inner sep=0pt] (russell) at (-0.2,-0.15){\includegraphics[angle=0,scale=0.07]{Black_Swan.jpg}}; 

%\node[inner sep=0pt] (russell) at (-0.2,-0.15){\includegraphics[angle=0,scale=0.2]{black_swan.png}};  

% Radius of regular polygons
\begin{scope}[rotate=30]
  \newdimen\R
  \R=2cm
  \coordinate (center) at (0,0);
 \draw (0:\R)
     \foreach \x in {60,120,...,360} {  -- (\x:\R) }
              -- cycle (300:\R)
              -- cycle (240:\R)
              -- cycle (180:\R)
              -- cycle (120:\R)
              -- cycle (60:\R)
              -- cycle (0:\R)  [line width=1.9mm,color=rred];
\end{scope}

% normal plot
\begin{scope}[scale=0.4,xshift=-3.5cm,yshift=-2.5cm]               
\begin{axis}[every axis plot post/.append style={
  mark=none,samples=50,smooth},
  axis x line*=bottom,
  axis y line= none, 
  enlargelimits=upper,
  xtick=\empty]
  
  \addplot[domain=-1:1,color=white]{gauss(0,1.3)};
  \addplot[domain=-7:7]{gauss(0,2)};
  \addplot[domain=-7:-3]{gauss(0,5)};
  
\end{axis}  
\end{scope}



%\begin{scope}[scale=0.42,xshift=-3.5cm,yshift=-2.3cm]
%\begin{axis}[
%axis y line=none,
%axis x line*=bottom,
%ymin=0,
%xtick=\empty
%]
%\addplot[domain=-9:-2.98]{\gauss{0}{4}};
%\addplot[domain=-3.01:9]{\gauss{0}{2}};
%\draw[dashed] (axis description cs:0.5,0) -- (axis description cs:0.5,0.92);
%\end{axis}
%\end{scope}              



             
%\draw[color=black,draw,align=left] (-1.2,1.05) node[right,scale=1.15] { $\displaystyle
%P(A\mid B) = \frac{P(B\mid A)P(A)}{P(B)}
%$}; 


% caption 
%\draw[color=white,draw,align=left] (-0.68,-0.9) node[right,scale=1.8] {church of};
%\draw[color=white,draw,align=left] (-0.47,-1.21) node[right,scale=1.8] {Bayes};                    
\end{tikzpicture}

\end{document}

























