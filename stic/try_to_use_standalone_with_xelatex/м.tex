\documentclass{standalone}

\usepackage{amsmath,amsfonts,amssymb,amsthm,mathtools} 
\usepackage{fontspec}            % пакет для подгрузки шрифтов
\setmainfont{Roboto}   % задаёт основной шрифт документа

% why do we need \newfontfamily:
% http://tex.stackexchange.com/questions/91507/
\newfontfamily{\cyrillicfonttt}{Roboto}
\newfontfamily{\cyrillicfont}{Roboto}
\newfontfamily{\cyrillicfontsf}{Roboto}
% Иногда тех не видит структуры шрифтов. Эти трое бравых парней спасают ситуацию и доопределяют те куски, которые Тех не увидел.

\usepackage{unicode-math}     % пакет для установки математического шрифта
\setmathfont{Asana Math}      % шрифт для математики

\usepackage{polyglossia}      % Пакет, который позволяет подгружать русские буквы
\setdefaultlanguage{russian}  % Основной язык документа
\setotherlanguage{english}    % Второстепенный язык документа

\usepackage{pgf,tikz}
\usetikzlibrary{arrows,calc}
\usepackage{relsize} 

\usepackage{graphicx} 
\usepackage{rotating}
\usepackage{xcolor}
\usepackage{color}

\definecolor{gitt}{HTML}{713015}

\begin{document}


\begin{tikzpicture}[
        scale=2,
        dot/.style={circle,fill=black,minimum size=4pt,inner sep=0pt,
            outer sep=-1pt},
    ]
    
% fixation
\node[inner sep=0pt,scale=13,color=white] (russell) at (0,0){$\hat{\mupbeta}$};                 
% beta-coef     
\node[inner sep=0pt,scale=11] (russell) at (0,0){$\hat{\mupbeta}$};   
% left hand
\draw (0.4,0.4) .. controls (0.6,-0.5) and (0.6,-0.5) .. (0.6,0) [line width=1.8pt];         
% guitar
\node[inner sep=0pt] (russell) at (-0.07,-0.3){\includegraphics[angle=0,scale=0.12]{git2.png}};        
% right hand
\draw (-0.2,0.2) .. controls (-1,0) and (-1,0) .. (-0.45,-0.4) [line width=1.8pt];
% fingers on the left hand
\draw (0.6,-0.1) -- (0.65,0.055)[line width=1.1pt]; 
\draw (0.6,-0.2) -- (0.6,-0.1)[line width=1.8pt]; 
\draw (0.6,-0.1) -- (0.6,0.055)[line width=1.1pt]; 
\draw (0.6,-0.1) -- (0.55,0.055)[line width=1.1pt]; 
% fingers on the right hand
\draw (-0.45,-0.4) -- (-0.35,-0.45)[line width=1.1pt]; 
\draw (-0.45,-0.4) -- (-0.35,-0.40)[line width=1.1pt]; 
\draw (-0.45,-0.4) -- (-0.38,-0.50)[line width=1.1pt];         
% formula
\node[inner sep=0pt,scale=2] (russell) at (2.5,-0.1){$\displaystyle y_t = \sum_{s=0}^{\infty} \gamma_s$};
\node[inner sep=0pt] (russell) at (3.8,-0.15){\includegraphics[scale=0.15]{ples.jpg}};   
\node[inner sep=0pt,scale=2] (russell) at (4.3,-0.15){$+$};
\node[inner sep=0pt] (russell) at (4.9,-0.15){\includegraphics[scale=0.058]{honey.jpg}}; 
% caption
\draw[color=black,draw,align=left] (1.5,0.6) node[right] {{\color{black} \textbf{\small \fontspec{Roboto}{Он разложился на плесень и на липовый мёд...} }}}; 
\draw[color=black,draw,align=left] (4.53,-0.27) node[right] {{\color{black} \textbf{\tiny \fontspec{Roboto}{липовый} }}};

\node[inner sep=0pt] (russell) at (0,0.9){\includegraphics[scale=0.45]{colp.png}}; 

\end{tikzpicture}

\end{document}