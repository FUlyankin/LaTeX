%!TEX TS-program = xelatex
\documentclass[12pt, a4paper, oneside]{article}

\usepackage{amsmath,amsfonts,amssymb,amsthm,mathtools}  % пакеты для математики

\usepackage[utf8]{inputenc} % задание utf8 кодировки исходного tex файла
\usepackage[british,russian]{babel} % выбор языка для документа

\usepackage{fontspec}         % пакет для подгрузки шрифтов
\setmainfont{Helvetica}   % задаёт основной шрифт документа

% why do we need \newfontfamily:
% http://tex.stackexchange.com/questions/91507/
\newfontfamily{\cyrillicfonttt}{Helvetica}
\newfontfamily{\cyrillicfont}{Helvetica}
\newfontfamily{\cyrillicfontsf}{Helvetica}

\usepackage{unicode-math}     % пакет для установки математического шрифта
\setmathfont{Neo Euler}      % шрифт для математики
% \setmathfont[math-style=ISO]{Asana Math}
% Можно делать смену начертания с помощью разных стилей

% Конкретный символ из конкретного шрифта
% \setmathfont[range=\int]{Neo Euler}

%%%%%%%%%% Работа с картинками %%%%%%%%%
\usepackage{graphicx}                  % Для вставки рисунков
\usepackage{graphics}
\graphicspath{{images/}{pictures/}}    % можно указать папки с картинками
\usepackage{wrapfig}                   % Обтекание рисунков и таблиц текстом

%%%%%%%%%%%%%%%%%%%%%%%% Графики и рисование %%%%%%%%%%%%%%%%%%%%%%%%%%%%%%%%%
\usepackage{tikz, pgfplots}  % язык для рисования графики из latex'a

%%%%%%%%%% Гиперссылки %%%%%%%%%%
\usepackage{xcolor}              % разные цвета

% Два способа включить в пакете какие-то опции:
%\usepackage[опции]{пакет}
%\usepackage[unicode,colorlinks=true,hyperindex,breaklinks]{hyperref}

\usepackage{hyperref}
\hypersetup{
	unicode=true,           % позволяет использовать юникодные символы
	colorlinks=true,       	% true - цветные ссылки, false - ссылки в рамках
	urlcolor=blue,          % цвет ссылки на url
	linkcolor=red,          % внутренние ссылки
	citecolor=green,        % на библиографию
	pdfnewwindow=true,      % при щелчке в pdf на ссылку откроется новый pdf
	breaklinks              % если ссылка не умещается в одну строку, разбивать ли ее на две части?
}


\usepackage{todonotes} % для вставки в документ заметок о том, что осталось сделать
% \todo{Здесь надо коэффициенты исправить}
% \missingfigure{Здесь будет Последний день Помпеи}
% \listoftodos --- печатает все поставленные \todo'шки

\usepackage[paper=a4paper, top=20mm, bottom=15mm,left=20mm,right=15mm]{geometry}
\usepackage{indentfirst}       % установка отступа в первом абзаце главы

\usepackage{setspace}
\setstretch{1.15}  % Межстрочный интервал
\setlength{\parskip}{4mm}   % Расстояние между абзацами
% Разные длины в латехе https://en.wikibooks.org/wiki/LaTeX/Lengths


\usepackage{xcolor} % Enabling mixing colors and color's call by 'svgnames'

\definecolor{MyColor1}{rgb}{0.2,0.4,0.6} %mix personal color
\newcommand{\textb}{\color{Black} \usefont{OT1}{lmss}{m}{n}}
\newcommand{\blue}{\color{MyColor1} \usefont{OT1}{lmss}{m}{n}}
\newcommand{\blueb}{\color{MyColor1} \usefont{OT1}{lmss}{b}{n}}
\newcommand{\red}{\color{LightCoral} \usefont{OT1}{lmss}{m}{n}}
\newcommand{\green}{\color{Turquoise} \usefont{OT1}{lmss}{m}{n}}

\usepackage{titlesec}
\usepackage{sectsty}
%%%%%%%%%%%%%%%%%%%%%%%%
%set section/subsections HEADINGS font and color
\sectionfont{\color{MyColor1}}  % sets colour of sections
\subsectionfont{\color{MyColor1}}  % sets colour of sections

%set section enumerator to arabic number (see footnotes markings alternatives)
\renewcommand\thesection{\arabic{section}.} %define sections numbering
\renewcommand\thesubsection{\thesection\arabic{subsection}} %subsec.num.

%define new section style
\newcommand{\mysection}{
	\titleformat{\section} [runin] {\usefont{OT1}{lmss}{b}{n}\color{MyColor1}} 
	{\thesection} {3pt} {} } 


%	CAPTIONS
\usepackage{caption}
\usepackage{subcaption}
%%%%%%%%%%%%%%%%%%%%%%%%
\captionsetup[figure]{labelfont={color=Turquoise}}

\pagestyle{empty}

\begin{document}

\section*{Задание 1: фактчек  (10 баллов)}

\textbf{Дедлайн:} вечер 18 сентября

Изучить \href{https://github.com/FUlyankin/LaTeX}{страничку курса}, если вы ещё её не изучили. Установить LaTeX, если он ещё не установлен либо зарегистрироваться в Overleaf. Создать свой первый документ в LaTeX. Помните о том, что любое творчество поощряется. 


\begin{itemize}

\item [$(2)$] Написать в нём перечень с 10 фактами о себе

\item[$(2)$] Вставить свою фотографию.  Вставить мемас, с которого ты нормально орнул в последнее время. Пояснить почему орнул. Картинки должны быть пронумерованы. По ходу текста на них должны быть ссылки. 

\item[$(2)$] Нарисовать таблицу из трёх колонок: название предмета, насколько ты любишь его по 10-бальной шкале, ассоциация, которая возникает с этим предметом. Например, тервер, $10$, кубик. Первая колонка должна быть выравнена по левому краю, вторая по центру, третья по правому краю. Заполните таблицу как минимум $5$ строками.

\item[$(3)$] Написать 5 своих любимых формул и одну ненавистную (если такая конечно найдётся).  Рассказать почему ты любишь какие-то формулы, а какие-то нет. Сослаться в тексте на эти формулы. 

\begin{itemize}
	\item  В этих формулах должны встретиться:  хотябы один интеграл, сумма,  предел, дробь, матрица
	\item  Хотябы одна формула должна занять несколько строк. 
	\item  Хотябы раз вы должны создать в преамбуле для какой-то формулы или обозначения для удобства короткий алиас через \texttt{def} или \texttt{newcommand}.
	\item  Все формулы должны быть реальными. Нельзя набирать формулы из рандомных символов и объявлять их своими любимыми. 
\end{itemize}

\todo[inline]{В формулах нельзя делать фактических ошибок. В прошлые разы многие неправильно вбили формулу для геометрической прогрессии. При этом они написали, что она у них любимая. ХА! НЕ ВЕРЮ!}

\item[$(1)$] Все формулы, которые упоминаются в тексте должны быть пронумерованы символами æ, ææ, æææ и так далее. Нумерацию можно задать вручную, не прописывая в преамбуле никаких опций. Команду для этого придётся нагуглить. 
\end{itemize}

Итоговый pdf-файл, tex-файл и все картинки, которые использовались в документе, нужно положить в папку на свой Dropbox, Github, yandex-disk или другой репозиторий. После нужно заполнить \href{https://docs.google.com/forms/d/e/1FAIpQLSe11kxKVfv07iCL1E9yNX7ll9swKImiVwRr1H70lslGzInRSg/viewform}{уютную гугл-форму.}  

Если вы выполняете работу в overleaf, то прикрепляйте в форму ссылку на неё. Если вы прикрепите ссылку с возможностью редактирования вашего файла, при проверке мы оставим у вас в файле полезные комментарии. Сдача домашке через overleaf - предпочтительна.  Однако вы можете залить свою работу на дроп-бокс или яндекс-диск и прикрепить ссылку на архив для скачки. 
\end{document}
