\documentclass[12pt, a4paper]{article}  

\usepackage{etex} % расширение классического tex в частности позволяет подгружать гораздо больше пакетов, чем мы и займёмся далее

%%%%%%%%%% Математика %%%%%%%%%%
\usepackage{amsmath,amsfonts,amssymb,amsthm,mathtools} 
%\mathtoolsset{showonlyrefs=true}  % Показывать номера только у тех формул, на которые есть \eqref{} в тексте.
%\usepackage{leqno} % Нумерация формул слева


%%%%%%%%%%%%%%%%%%%%%%%% Шрифты %%%%%%%%%%%%%%%%%%%%%%%%%%%%%%%%%
\usepackage{fontspec}         % пакет для подгрузки шрифтов
\setmainfont{Arial}   % задаёт основной шрифт документа

\defaultfontfeatures{Mapping=tex-text}

% why do we need \newfontfamily:
% http://tex.stackexchange.com/questions/91507/
\newfontfamily{\cyrillicfonttt}{Arial}
\newfontfamily{\cyrillicfont}{Arial}
\newfontfamily{\cyrillicfontsf}{Arial}

\usepackage{unicode-math}     % пакет для установки математического шрифта
\setmathfont{Asana Math}      % шрифт для математики

\usepackage{polyglossia}      % Пакет, который позволяет подгружать русские буквы
\setdefaultlanguage{russian}  % Основной язык документа
\setotherlanguage{english}    % Второстепенный язык документа


%%%%%%%%%% Работа с картинками %%%%%%%%%
\usepackage{graphicx}                  % Для вставки рисунков
\usepackage{graphics} 
\graphicspath{{images/}{pictures/}}    % можно указать папки с картинками
\usepackage{wrapfig}                   % Обтекание рисунков и таблиц текстом
\usepackage{subfigure}                 % для создания нескольких рисунков внутри одного


%%%%%%%%%% Работа с таблицами %%%%%%%%%%
\usepackage{tabularx}            % новые типы колонок
\usepackage{tabulary}            % и ещё новые типы колонок
\usepackage{array}               % Дополнительная работа с таблицами
\usepackage{longtable}           % Длинные таблицы
\usepackage{multirow}            % Слияние строк в таблице
\usepackage{float}               % возможность позиционировать объекты в нужном месте 
\usepackage{booktabs}            % таблицы как в книгах!  
\renewcommand{\arraystretch}{1.3} % больше расстояние между строками



\usepackage{tikz}  % Пакет для графики. Будем разбирать его на следующей паре.
\usepackage{pgfplots}   % Аналогично  
\usetikzlibrary{arrows}
\usepackage{color}
\usepackage{xcolor}
\usepackage{lscape}
\usepackage{makecell}

\usepackage{pgf,tikz}
\usepackage{mathrsfs}
\usetikzlibrary{arrows}
\pagestyle{empty}

\begin{document}
\newenvironment{peppa}{}{ \vspace{2mm}	
  \begin{center}	
	\definecolor{ffvvqq}{rgb}{1.,0.3333333333333333,0.}
	\definecolor{ccqqqq}{rgb}{0.8,0.,0.}
	\definecolor{xdxdff}{rgb}{0.49019607843137253,0.49019607843137253,1.}
	\definecolor{qqqqff}{rgb}{0.,0.,1.}
	\definecolor{cqcqcq}{rgb}{0.7529411764705882,0.7529411764705882,0.7529411764705882}
	\begin{tikzpicture}[line cap=round,line join=round,>=triangle 45,x=1.0cm,y=1.0cm]
	\clip(-0.74,3.94) rectangle (15.56,11.56);
	\draw [rotate around={-89.7297388935065:(7.01,4.88)},color=black,fill=green] (7.01,4.88) ellipse (2.4583946869193305cm and 1.2446704128696924cm);
	\draw [shift={(6.682184690157958,7.90300121506683)},color=black,fill=white] plot[domain=-3.2025215179626563:1.4615941112304696,variable=\t]({1.*1.2645311403640185*cos(\t r)+0.*1.2645311403640185*sin(\t r)},{0.*1.2645311403640185*cos(\t r)+1.*1.2645311403640185*sin(\t r)});
	\draw (5.920328037910039,6.083071262472925)-- (4.34,4.96);
	\draw (8.106378882577305,6.028842791515977)-- (9.68,4.84);
	\draw [shift={(6.087142857142863,9.7975)},color=black] plot[domain=3.694951603145163:4.360592028998656,variable=\t]({1.*1.9360748543991728*cos(\t r)+0.*1.9360748543991728*sin(\t r)},{0.*1.9360748543991728*cos(\t r)+1.*1.9360748543991728*sin(\t r)});
	\draw (6.82,9.16)-- (5.52,9.52);
	\draw [shift={(5.1132,8.9556)},color=black, fill=pink] plot[domain=0.9462693638927046:3.3967508563190765,variable=\t]({1.*0.6957252331200864*cos(\t r)+0.*0.6957252331200864*sin(\t r)},{0.*0.6957252331200864*cos(\t r)+1.*0.6957252331200864*sin(\t r)});
	\draw [shift={(4.634130434782608,9.654782608695651)},color=black, fill=pink] plot[domain=4.49400949295222:6.132196034439149,variable=\t]({1.*0.8960643047154899*cos(\t r)+0.*0.8960643047154899*sin(\t r)},{0.*0.8960643047154899*cos(\t r)+1.*0.8960643047154899*sin(\t r)});
	\draw [color=black, fill=white](6.44,8.6) circle (0.26cm);
	\draw[color=black, fill=white](7.06,8.32) circle (0.2668332812825266cm);
	\draw [shift={(6.23,7.67)},line width=1.6pt,color=ccqqqq]  plot[domain=2.525972764160683:5.667565417750476,variable=\t]({1.*0.502195181179589*cos(\t r)+0.*0.502195181179589*sin(\t r)},{0.*0.502195181179589*cos(\t r)+1.*0.502195181179589*sin(\t r)});
	\draw [rotate around={61.38954033403483:(7.,9.49)},line width=1.6pt,fill=pink] (7.,9.49) ellipse (0.4180350850198669cm and 0.18290252132636337cm);
	\draw [rotate around={63.43494882292202:(7.53,9.36)},line width=1.6pt,fill=pink] (7.53,9.36) ellipse (0.45763357036490254cm and 0.17008375796986158cm);
	\draw [line width=1.6pt] (8.26,8.42) circle (0.2842534080710382cm);
	\draw [line width=1.6pt] (9.,9.) circle (0.38626415831655897cm);
    \draw [line width=1.6pt] (9.8,10.26) circle (0.5946427498927405cm);
	\begin{scriptsize}
	\draw [fill=black] (7.06,8.32) circle (2.5pt);
	\draw [fill=ffvvqq] (4.7,9.22) circle (2.5pt);
	\draw [fill=ffvvqq] (5.02,9.18) circle (2.5pt);
	\draw [fill=black] (6.44,8.6) circle (2.5pt);
	\end{scriptsize}
	\end{tikzpicture}
  \end{center}	
}

\begin{peppa}
	Свинка Пеппа пам-пара-рам-рам-пам-парара-рам-пам
\end{peppa}


\end{document}