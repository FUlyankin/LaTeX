%!TEX TS-program = xelatex
\documentclass[12pt, a4paper]{article}  

\usepackage{etex} % расширение классического tex в частности позволяет подгружать гораздо больше пакетов, чем мы и займёмся далее

%%%%%%%%%% Математика %%%%%%%%%%
\usepackage{amsmath,amsfonts,amssymb,amsthm,mathtools} 
%\mathtoolsset{showonlyrefs=true}  % Показывать номера только у тех формул, на которые есть \eqref{} в тексте.
%\usepackage{leqno} % Нумерация формул слева


%%%%%%%%%%%%%%%%%%%%%%%% Шрифты %%%%%%%%%%%%%%%%%%%%%%%%%%%%%%%%%
\usepackage{fontspec}         % пакет для подгрузки шрифтов
\setmainfont{Arial}   % задаёт основной шрифт документа

\defaultfontfeatures{Mapping=tex-text}

% why do we need \newfontfamily:
% http://tex.stackexchange.com/questions/91507/
\newfontfamily{\cyrillicfonttt}{Arial}
\newfontfamily{\cyrillicfont}{Arial}
\newfontfamily{\cyrillicfontsf}{Arial}

\usepackage{unicode-math}     % пакет для установки математического шрифта
\setmathfont{Asana Math}      % шрифт для математики
% \setmathfont[math-style=ISO]{Asana Math}
% Можно делать смену начертания с помощью разных стилей

% Конкретный символ из конкретного шрифта
% \setmathfont[range=\int]{Neo Euler}

\usepackage{polyglossia}      % Пакет, который позволяет подгружать русские буквы
\setdefaultlanguage{russian}  % Основной язык документа
\setotherlanguage{english}    % Второстепенный язык документа


%%%%%%%%%% Работа с картинками %%%%%%%%%
\usepackage{graphicx}                  % Для вставки рисунков
\usepackage{graphics}
\graphicspath{{images/}{pictures/}}    % можно указать папки с картинками
\usepackage{wrapfig}                   % Обтекание рисунков и таблиц текстом


%%%%%%%%%% Работа с таблицами %%%%%%%%%%
\usepackage{tabularx}            % новые типы колонок
\usepackage{tabulary}            % и ещё новые типы колонок
\usepackage{array}               % Дополнительная работа с таблицами
\usepackage{longtable}           % Длинные таблицы
\usepackage{multirow}            % Слияние строк в таблице
\usepackage{float}               % возможность позиционировать объекты в нужном месте
\usepackage{booktabs}            % таблицы как в книгах!
\renewcommand{\arraystretch}{1.3} % больше расстояние между строками
\usepackage{rotating}





%%%%%%%%%% Графика и рисование %%%%%%%%%%
\usepackage{tikz, pgfplots}  % язык для рисования графики из latex'a
\usepackage{amscd}                  %Пакеты для рисования
\usepackage[matrix,arrow,curve]{xy} %комунитативных диаграмм


%%%%%%%%%% Теоремы %%%%%%%%%%
\theoremstyle{plain}              % Это стиль по умолчанию.  Есть другие стили. defenition - тоже стиль
\newtheorem{theorem}{Теорема}[section]
\newtheorem{result}{Следствие}[theorem]
% счётчик подчиняется теоремному, нумерация идёт по главам согласованно между собой

\theoremstyle{definition}         % убирает курсив и что-то еще наверное делает ;)
\newtheorem*{defin}{Определение}  % нумерация не идёт вообще

\newtheorem{fignia}{Какая-то фигня}




%%%%%%%%%% Свои команды %%%%%%%%%%
\usepackage{etoolbox}    % логические операторы для своих макросов


% Все свои команды лучше всего определять не по ходу текста, как это сделано в этом документе, а в преамбуле!




% Пакет, который ставит в каждом первом абзаце главы красную строку
% Просто, чтобы эта pdf-ка нормально смотрелась :)
\usepackage{indentfirst}
\setkeys{russian}{babelshorthands=true}
\def \s{$\sigma$}
\def \posl{$x_1 \ldots x_n$}
\DeclareMathOperator{\Cov}{Cov}
\DeclareMathOperator{\Var}{Var}

\newcommand{\com}[2]{$x_#1 \ldots x_#2$}
\newcommand*\Myitem{%
  \item[\color{blue}{\textbullet}]}
  
\newcommand{\llim}[3]{$\lim\limits _{#1 \to #2} #3$  }
\renewcommand{\thefigure}{\thesection : \arabic{figure}}
\renewcommand{\theequation}{Eq.(\arabic{equation})}

\newcommand{\pic}[3]{\begin{figure}[H]
\begin{center}
\includegraphics[scale=#1]{#2}
\caption{#3}
\end{center}
\end{figure}}


\title{Уютная домашка }
\date{\today}

\begin{document}
\maketitle
\newpage


\section{Задание 1.1}
 \begin{itemize}
 \item $\Cov(x)$ \\
 \item $\Var(x)$
 \end{itemize}


\section{Задание 1.2}
\s - алгебра
\section{Задание 1.3}
\posl
\section{Задание 1.4}
\begin{enumerate}
\item \com{a}{z} 
\item \com{1}{6} 
\item \com{(a,b)}{(c,d)}
\end{enumerate}
\section{Задание 1.5}

\begin{itemize}
    \Myitem Первый пункт
    \Myitem Второй пункт
    \Myitem Третий пункт
  \end{itemize}
\section{Задание 1.6}

\llim{x}{0}{\frac{\sin{x}}{x}}


\section{Задание 1.7}

\begin{figure}[H]
\includegraphics[scale=0.5]{w.jpg}
\caption{Весна?}
\end{figure}

\begin{figure}[H]
\begin{center}
\includegraphics[scale=0.5]{winni.jpg}
\caption{Весна!}
\end{center}
\end{figure}



\section{Задание 1.8}


\begin{equation} \label{a} \int_{-\infty}^{+\infty} e^{-x^{2}} = \sqrt{\pi} \end{equation}
 \begin{equation} \label{aa} \frac{1}{\pi} = \frac{2\sqrt{2}}{9801}\sum_{k=0}^{\infty} \frac{(4k)!(1103+26390k)}{(k!)^4 396^{4k}} \end{equation}
 \begin{equation} \label{aaa} \lim_{\alpha \to 0} \frac{\sin{\alpha}}{\alpha} = 1 \end{equation}
\begin{equation} \label{aaaa} A^{-1}= \frac{1}{detA} \begin{pmatrix} 
 A_{1,1} & A_{2,1} & \cdots & A_{n,1} \\
  A_{1,2} & A_{2,2} & \cdots & A_{n,2} \\
  \vdots  & \vdots  & \ddots & \vdots  \\
  A_{1,n} & A_{2,n} & \cdots & A_{n,n}
\end{pmatrix} 
\end{equation}
\begin{multline} \label{aaaaa} e^x = 1 + \frac{x}{1!} + \frac{x^2}{2!} + \frac{x^3}{3!} + \frac{x^4}{4!} + \frac{x^5}{5!} + \frac{x^6}{6!} + \frac{x^7}{7!} + \frac {x^8}{8!} + \frac{x^9}{9!} + \frac{x^{10}}{10!} + \frac{x^{11}}{11!} + \frac{x^{12}}{12!}+ \frac{x^{13}}{13!} + \frac{x^{14}}{14!}+\frac{x^{15}}{15!} = \\
+\frac{x^{16}}{16!}+\frac{x^{17}}{17!}+\frac{x^{18}}{18!}+\frac{x^{19}}{19!}+\frac{x^{20}}{20!} \cdots + \frac{x^n}{n!} \end{multline}

\section{Задание 1.9}

\rotatebox{180}{Сигмой \s в математической статистике обозначают стандартное отклонение.}


\section{Задание 1.10}

\pic{0.5}{images.jpg}{Винни!}






\end{document}