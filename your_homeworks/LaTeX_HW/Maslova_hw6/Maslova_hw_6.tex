\documentclass[t,dvipsnames]{beamer}
\setbeamertemplate{navigation symbols}{}
\usetheme{AnnArbor}
\usecolortheme{beaver}
\useinnertheme{circles}

%%% Работа с русским языком
\usepackage[english,russian]{babel}   %% загружает пакет многоязыковой вёрстки
\usepackage{fontspec}      %% подготавливает загрузку шрифтов Open Type, True Type и др.
\defaultfontfeatures{Ligatures={TeX},Renderer=Basic}  %% свойства шрифтов по умолчанию
\setmainfont[Ligatures={TeX,Historic}]{Times New Roman} %% задаёт основной шрифт документа
\setsansfont{Arial}                    %% задаёт шрифт без засечек
\setmonofont{Courier New}
\usepackage{indentfirst}
\frenchspacing
%%% Работа с картинками
\usepackage{graphicx}  % Для вставки рисунков
\usepackage{graphics} 
\setlength\fboxsep{3pt} % Отступ рамки \fbox{} от рисунка
\setlength\fboxrule{1pt} % Толщина линий рамки \fbox{}
\usepackage{wrapfig} % Обтекание рисунков текстом

\usepackage{xcolor, colortbl}
\usepackage{booktabs}

%%% Работа с таблицами
\usepackage{array,tabularx,tabulary,booktabs} % Дополнительная работа с таблицами
\usepackage{longtable}  % Длинные таблицы
\usepackage{multirow} % Слияние строк в таблице

\definecolor{backgr}{HTML}{fffff3}


\setbeamercolor{block title}{fg=white,bg=Red}
\setbeamerfont{block title}{family=\sffamily}
\setbeamercolor{block body}{bg=white}
\setbeamertemplate{blocks}[rounded][shadow=true]
\setbeamercovered{invisible}

\newenvironment{slide}
{\begin{frame}[fragile,environment=slide]
		\frametitle{\thesection . \insertsection}
	    \framesubtitle{\thesection.\thesubsection. \insertsubsection}}
	{\end{frame}}

\title[Контрактные отношения]{Контрактные отношения}

\author[Маслова Инна]{Маслова Инна}
\institute[РАНХиГС]{
	Российская Академия Народного Хозяйства и  \\ Государственной Службы при Президенте Российской Федерации}
\date[\today]{\today}

\begin{document}
	
	\frame[plain]{\titlepage}	% Титульный слайд
	\frame[plain]{\frametitle{План презентации}\tableofcontents}
	
	\section{Просеивание}
	\subsection{Понятие просеивания}
	\frame{\tableofcontents[currentsection]}
	\begin{slide}
		\uncover<1>{Просеивание --- действия стороны, не обладающей информацией, направленные на разделение различных типов информированной стороны по определенным характеристикам}
		\uncover<2->{\begin{figure}
			\includegraphics[scale=0.8]{scheme}
		\end{figure}}
		
	\end{slide}
    

	\section{Примеры применения просеивания}
	\subsection{Рынок заработной платы}
	\frame{\tableofcontents[currentsection]}
	
	
	\begin{slide}
			
	\begin{columns}[T]
		\begin{column}{.5\textwidth}
			\includegraphics[width=0.8\linewidth]{photo}	
				\end{column}
		\begin{column}{.5\textwidth}
			\begin{block}{Асимметрия информации:}
			Фирма не знает, насколько часто работники склонны менять работу
			\end{block}
	\begin{block}{Решение:}
		\textbf{Просеивание} \\
		Предлагаем контракт:сначала оплата труда --- низкая, а потом, после того, как работник проработал значительный период времени --- зарплата выше рыночной ставки
	\end{block}
	\end{column}
	\end{columns}

	\end{slide}

	\subsection{Влияние правила на рынок поддержанных товаров}
	
    \begin{slide}
	
		\begin{table}[H]
				\begin{tabularx}{\textwidth}{p{0.45\linewidth}|X}
    \hline
	\rowcolor{Red} Правило первоначального владения & Правило добросовестного приобретателя  \\
	\hline
    \only<2->{Есть ассиметрия информации & Не краденые товары продаются по своей цене  \\ \hline}
	
	\only<3->{Товары будут продаваться по цене краденых вещей & Доля краденых товаров – ниже, чем доля «хороших»  \\ \hline}
	
	\only<4->{Еще один «рынок лимонов» &  \\
	\hline}
	\end{tabularx}
\end{table}
	\end{slide}

\end{document}

