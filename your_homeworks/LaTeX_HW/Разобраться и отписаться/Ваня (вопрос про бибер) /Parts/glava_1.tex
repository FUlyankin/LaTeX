\chapter{Теоритические сведения об облигациях}
\section{Понятие облигации}
Согласно Федеральному закону от 22 апреля 1996 г. (ред. от 23.07.2013) «О рынке ценных бумаг» облигация – эмиссионная ценная бумага, закрепляющая право её владельца на получение от эмитента облигации в предусмотренный в ней срок её номинальной стоимости или иного имущественного эквивалента. Облигация может также предусматривать право её владельца на получение дохода до момента погашения облигации (купоны).

Эмитент – агент, выпускающий ценную бумагу.

Основными источниками дохода по облигации являются: 
\begin{itemize}
	\item купонные выплаты
	\item дисконт от номинальной стоимости
\end{itemize}
Оба источника дохода по облигации поддаются точному прогнозированию в части размера платежа и его даты, поскольку облигационный займ предполагает проспект эмиссии, в котором указанные параметры чётко прописаны. Знание ключевых параметров, от которых зависит доход эмитента, позволяет проводить оценку облигаций практически в условиях полной определённости.

\section{Классификация облигаций}

В теории и практике существует большое множество различных классификаций облигаций, однако в рамках данной работы рассматриваются только базовые классификации облигаций, приведённые в таблице \ref{tab:class_bonds}.

\begin{table}[h!]
	\centering
	\caption{Базовые классификации облигации}
	\begin{tabular}{|p{4cm}|Z{4}|p{5cm}|}
		\hline
		\multicolumn{1}{|Z{4}|}{Классификационный признак} & Влияние на оценку стоимости & \multicolumn{1}{c|}{Виды облигаций} \\
		\hline
		Эмитент & Нет   & Государственные, муниципальные, корпоративные \\
		\hline
		По площадке размещения & Нет   & Еврооблигации (евробонды), иностранные, международные, глобальные \\
		\hline
		Легитимации владения & Нет   & Именные, на предъявителя, с отрывным купоном \\
		\hline
		Обеспеченность & Нет   & Необеспеченные, обеспеченные \\
		\hline
		Срок погашения & Да    & Краткосрочные, среднесрочные, долгосрочные, бессрочные \\
		\hline
		Возможность продления срока & Да    & Непродлеваемые, продлеваемые (пролонгируемые) \\
		\hline
		Возможность досрочного погашения & Да    & Безотзывные, отзывные \\
		\hline
		Погашение номинала & Да    &  Одним платежом в конце срока, амортизируемые, смешанные \\
		\hline
		Формирование дохода & Да    & С фиксированным купоном, с плавающим купоном, индексируемые \\
		\hline
		Способ выплаты дохода & Да    & Процентные, бескупонные \\
		\hline
		Обращение & Да    & Конвертируемые, неконвертируемые \\
		\hline
	\end{tabular}%
	\label{tab:class_bonds}%
\end{table}%


Можно выделить несколько классификационных признаков, которые не влияют на оценку стоимости облигации, например, является ли ценная бумага обеспеченной или кто выступает в качестве эмитента облигации. Но существуют такие классификационные признаки, которые напрямую воздействуют на оценку стоимости облигации. 

Рассмотрим подробно каждую классификацию. Анализ эмитентов облигаций позволяет оценить риски вложения в эти ценные бумаги. Так, государственные и муниципальные облигации по сравнению с остальными обычно наименее рискованные. Государственные облигации часто делят на федеральные и субфедеральные. Государственные компании могут также выпускать собственные облигации. Корпоративные облигации различаются по степени риска: от очень надежных до так называемых мусорных (junk) облигаций, представляющих собой, по сути, долги компаний, находящихся в состоянии банкротства.

Для оценки степени риска корпоративных облигаций специальными рисковыми агентствами составляются кредитные рейтинги эмитентов. Кредитные рейтинги выражают мнение агентства относительно способности эмитента своевременно и в полном объеме выполнить свои финансовые обязательства. Они отражают кредитное качество долгового обязательства и относительную вероятность дефолта по нему. (Дефолт – частичная или полная неспособности эмитента расплатиться по своим обязательствам). Прогнозные оценки кредитных рейтингов могут быть полезны инвесторам, принимающим долгосрочные и краткосрочные инвестиционные решения, но они не гарантируют, что инвестиции окупятся или что не произойдёт дефолт. 

\section{Оценка облигации без права досрочного погашения}

Поскольку согласно Федеральному закону «О рынке ценных бумаг» в эмиссионном проспекте облигационного выпуска зафиксированы все даты и суммы выплат по ценной бумаге, при этом выплаты по облигациям совершаются независимо от того, получила ли компания прибыль или нет, то фактически владение облигацией представляет собой верную финансовую ренту. А раз так, то для оценки стоимости облигации можно применить метод дисконтированных денежных потоков, т.е. рассматривать внутреннюю стоимость облигации как текущую стоимость всех денежных потоков, генерируемых этой облигацией.

Основные параметры облигационного займа, которые необходимо знать для определения стоимости облигации:
\begin{Enumerate}
 \item Номинальная стоимость (maturity value/face value/ par value/principal) – стоимость основного долга, подлежащего выплате в конце срока действия облигации. Помимо денежной суммы обязательно указывается валюта, в которой исчисляется номинальная стоимость.
 \item	Купон (coupon/coupon rate) – размер платежа (в денежных единицах или процентах), который выплачивается эмитентом (issuer) держателю облигации (bondholder). По умолчанию это значение равно купону за год с выплатой в конце периода. Если выплаты происходят чаще, то данное условие обязательно прописывается в проспекте эмиссии, более того, указывается точная календарная дата выплат купонов.
 \item Cрок действия облигации (time to maturity) – период времени, на который денежные средства были взяты у инвестора в долг.
 \item Опции отзыва (отзыв, предъявление, конвертация и т.п.) – опции, которые обязательно должны быть прописаны в проспекте эмиссии (issue) и которые изменяют срок действия облигации.
 \item Цена выкупа (redemption price) – денежная сумма с указанием валюты, по которой происходит погашение облигации до окончания срока действия облигации. Данный параметр опционален и относится только к облигациям, предоставляющим право досрочного погашения.
\end{Enumerate}

Этой информации достаточно для того, чтобы сформировать денежные потоки, генерируемые облигацией, и оценить их стоимость.


\section{Оценка бескупонных облигаций}

По бескупонной облигации существует только один платёж в конце срока действия облигации – её номинал (или иная сумма, указанная в проспекте). Поэтому внутренняя стоимость такой облигации ($V_0$) равна текущей стоисоти этого разового платежа. Допустим, облигация номиналом $MV$ погашается через полные $t$ лет, а ставка дисконтирования равна $r$, тогда
\begin{equation}
 V_0 = \frac{MV}{(1+r)^t}
\end{equation}

Чем меньше срок до погашения облигации, тем ближе внутренняя стоимость облигации к её номиналу. По таким облигациям существует только один источ-ник дохода – курсовая разница, которая образуется благодаря различию цены размещения и цены погашения. Именно поэтому такие облигации часто называются {\bf дисконтными}.
