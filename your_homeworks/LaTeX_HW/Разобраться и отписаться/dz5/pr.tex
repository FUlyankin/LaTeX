%!TEX TS-program = xelatex
\documentclass[12pt,a4paper, oneside]{extreport}

%%%%%%%%%% Програмный код %%%%%%%%%%
% \usepackage{minted}
% Включает подсветку команд в программах!
% Нужно, чтобы на компе стоял питон, надо поставить пакет Pygments, в котором он сделан, через pip.

% Для Windows: Жмём win+r, вводим cmd, жмём enter. Открывается консоль.
% Прописываем pip install Pygments
% Заходим в настройки texmaker и там прописываем в PdfLatex или XelaTeX:
% pdflatex -shell-escape -synctex=1 -interaction=nonstopmode %.tex

% Для Linux: Открываем консоль. Убеждаемся, что у вас установлен pip командой pip --version
% Если он не установлен, ставим его: sudo apt-get install python-pip
% Ставим пакет sudo pip install Pygments

% Для Mac: Всё то же самое, что на Linux, но через brew.

% После всего этого вы должны почувствовать себя тру-программистами!
% Документация по пакету хорошая. Сам читал, погуглите!

%%%%%%%%%% Математика %%%%%%%%%%
\usepackage{amsmath,amsfonts,amssymb,amsthm,mathtools}
% Показывать номера только у тех формул, на которые есть \eqref{} в тексте.
%\mathtoolsset{showonlyrefs=true}
%\usepackage{leqno} % Нумерация формул слева


%%%%%%%%%% Шрифты %%%%%%%%
\usepackage{fontspec}         % пакет для подгрузки шрифтов

\defaultfontfeatures{Mapping=tex-text}
\setmainfont{Times New Roman}       % задаёт основной шрифт документа

% Зачем на нужна команда \newfontfamily:
% http://tex.stackexchange.com/questions/91507/
\newfontfamily{\cyrillicfonttt}{Times New Roman}
\newfontfamily{\cyrillicfont}{Times New Roman}
\newfontfamily{\cyrillicfontsf}{Times New Roman}

\usepackage{unicode-math}     % пакет для установки математического шрифта
\setmathfont[math-style=ISO]{Asana Math}      % шрифт для математики

% Конкретный символ из конкретного шрифта
% \setmathfont[range=\int]{Neo Euler}

\usepackage{polyglossia}      % Пакет для выбора языка
% Основной язык документа. В опциях активируются всякие приколы для русского языка (вроде кучи разновидностей тире) из пакета babel
\setdefaultlanguage[babelshorthands=true]{russian}
\setotherlanguage{english}    % Второстепенный язык документа


%%%%%%%%%% Работа с картинками %%%%%%%%%
\usepackage{graphicx}                  % Для вставки рисунков
\usepackage{graphics}
\graphicspath{{images/}{pictures/}}    % можно указать папки с картинками
\usepackage{wrapfig}                   % Обтекание рисунков и таблиц текстом


%%%%%%%%%% Работа с таблицами %%%%%%%%%%
\usepackage{tabularx}            % новые типы колонок
\usepackage{tabulary}            % и ещё новые типы колонок
\usepackage{array,delarray}      % Дополнительная работа с таблицами
\usepackage{longtable}           % Длинные таблицы
\usepackage{multirow}            % Слияние строк в таблице
\usepackage{float}               % возможность позиционировать объекты в нужном месте

\usepackage{booktabs}            % таблицы как в книгах
% Заповеди из документации к booktabs:
% 1. Будь проще! Глазам должно быть комфортно
% 2. Не используйте вертикальные линни
% 3. Не используйте двойные линии. Как правило, достаточно трёх горизонтальных линий
% 4. Единицы измерения - в шапку таблицы
% 5. Не сокращайте .1 вместо 0.1
% 6. Повторяющееся значение повторяйте, а не говорите "то же"
% 7. Есть сомнения? Выравнивай по левому краю!

%  вычисляемые колонки по tabularx
\newcolumntype{C}{>{\centering\arraybackslash}X}
\newcolumntype{L}{>{\raggedright\arraybackslash}X}
\newcolumntype{Y}{>{\arraybackslash}X}
\newcolumntype{Z}{>{\centering\arraybackslash}X}


%%%%%%%%%% Графика и рисование %%%%%%%%%%
\usepackage{tikz, pgfplots}      % язык для рисования графики из latex'a

%%%%%%%%%% Гиперссылки %%%%%%%%%%
\usepackage{xcolor}              % разные цвета

\usepackage{hyperref}
\hypersetup{
	unicode=true,           % позволяет использовать юникодные символы
	colorlinks=true,       	% true - цветные ссылки, false - ссылки в рамках
	urlcolor =blue,         % цвет ссылки на url
	linkcolor=black,        % внутренние ссылки
	citecolor=black,        % на библиографию
	breaklinks              % если ссылка не умещается в одну строку, разбивать ли ее на две части?
}


%%%%%%%%%% Другие приятные пакеты %%%%%%%%%
\usepackage{multicol}       % несколько колонок
\usepackage{verbatim}       % для многострочных комментариев
\usepackage{cmap} % для кодировки шрифтов в pdf

\usepackage{enumitem} % дополнительные плюшки для списков
%  например \begin{enumerate}[resume] позволяет продолжить нумерацию в новом списке

\usepackage{todonotes} % для вставки в документ заметок о том, что  осталось сделать
% \todo{Здесь надо коэффициенты исправить}
% \missingfigure{Здесь будет Последний день Помпеи}
% \listoftodos --- печатает все поставленные \todo'шки



%%%%%%%%%%%%%% ГОСТОВСКИЕ ПРИБАМБАСЫ %%%%%%%%%%%%%%%

%%% размер листа бумаги
\usepackage[paper=a4paper,top=15mm, bottom=15mm,left=35mm,right=10mm,includehead,includefoot]{geometry}

\usepackage{setspace}
\setstretch{1.33}     % Межстрочный интервал
\setlength{\parindent}{1.5em} % Красная строка.


\flushbottom       % Эта команда заставляет LaTeX чуть растягивать строки, чтобы получить идеально прямоугольную страницу
\righthyphenmin=2  % Разрешение переноса двух и более символов
\widowpenalty=10000  % Наказание за вдовствующую строку (одна строка абзаца на этой странице, остальное --- на следующей)
\clubpenalty=10000  % Наказание за сиротствующую строку (омерзительно висящая одинокая строка в начале страницы)
\tolerance=1000     % Ещё какое-то наказание.


% Нумерация страниц сверху по центру
\usepackage{fancyhdr}
\pagestyle{fancy}
\fancyhead{ } % clear all fields
\fancyfoot{ } % clear all fields
\fancyhead[C]{\thepage}
% Чтобы не прорисовывалась черта! 
\renewcommand{\headrulewidth}{0pt}


% Нумерация страниц с надписью "Глава"
\usepackage{etoolbox}
\patchcmd{\chapter}{\thispagestyle{plain}}{\thispagestyle{fancy}}{}{}


%%% Заголовки
\usepackage[indentfirst]{titlesec}{\raggedleft} 
% Заголовки по левому краю    
% опция identfirst устанавливает отступ в первом абзаце 



% В Linux этот пакет сделан косячно. Исправляет это следующий непонятный кусок кода. 
\makeatletter
\patchcmd{\ttlh@hang}{\parindent\z@}{\parindent\z@\leavevmode}{}{}
\patchcmd{\ttlh@hang}{\noindent}{}{}{}
\makeatother


% Редактирования Глав и названий
\titleformat{\chapter}
{\normalfont\large\bfseries}
{\thechapter }{0.5 em}{}

% Редактирование ненумеруемых глав chapter* (Введение и тп) 
\titleformat{name=\chapter,numberless}
{\centering\normalfont\bfseries\large}{}{0.25em}{\normalfont}

% Убирает чеканутые отступы вверху страницы
\titlespacing{\chapter}{0pt}{-20pt}{15 pt} 

% Более низкие уровни 
\titleformat{\section}{\bfseries}{\thesection}{0.5 em}{}
\titleformat{\subsection}{\bfseries}{\thesubsection}{0.5 em}{}


% Содержание. Команды ниже изменяют отступы и рисуют точечки!
\usepackage{titletoc}

\titlecontents{chapter}
[1em] % 
{\normalsize}
{\contentslabel{1 em}}
{\hspace{-1 em}}
{\normalsize\titlerule*[10pt]{.}\contentspage}

\titlecontents{section}
[3 em] % 
{\normalsize}
{\contentslabel{1.75 em}}
{\hspace{-1.75 em}}
{\normalsize\titlerule*[10pt]{.}\contentspage}

\titlecontents{subsection}
[6 em] % 
{\normalsize}
{\contentslabel{3 em}}
{\hspace{-3 em}}
{\normalsize\titlerule*[10pt]{.}\contentspage}


% Правильные подписи под таблицей и рисунком
% Документация к пакету на русском языке! 
\usepackage[tableposition=top, singlelinecheck=false]{caption}
\usepackage{subcaption}

\DeclareCaptionStyle{base}%
[justification=centering,indention=0pt]{}

\DeclareCaptionLabelFormat{gostfigure}{Рисунок #2}
\DeclareCaptionLabelFormat{gosttable}{Таблица #2}

\DeclareCaptionLabelSeparator{gost}{~---~}
\captionsetup{labelsep=gost}

\DeclareCaptionStyle{fig01}%
[margin=5mm,justification=centering]%
{margin={3em,3em}}
\captionsetup*[figure]{style=fig01,labelsep=gost,labelformat=gostfigure,format=hang}

\DeclareCaptionStyle{tab01}%
[margin=5mm,justification=centering]%
{margin={3em,3em}}
\captionsetup*[table]{style=tab01,labelsep=gost,labelformat=gosttable,format=hang}



% межстрочный отступ в таблице
\renewcommand{\arraystretch}{1.2}



% многостраничные таблицы под РОССИЙСКИЙ СТАНДАРТ
% ВНИМАНИЕ! Обязательно за CAPTION !
\usepackage{fr-longtable}

\makeatletter
\LTcapwidth=\textwidth
\def\LT@makecaption#1#2#3{%
	\LT@mcol\LT@cols c{\hbox to\z@{\hss\parbox[t]\LTcapwidth{%
				\sbox\@tempboxa{ #1{#2: } #3}%
				\ifdim\wd\@tempboxa>\hsize
				\hbox to\hsize{\hfil #1#2\mbox{ }}
				\hbox to\hsize{\hfil \parbox[c]{0.9\textwidth}{\centering #3}\hfil }%%
				\else
				\hbox to\hsize{\hfil #1#2\mbox{ }}
				\hbox to\hsize{\hfil #3\hfil}%
				\fi
				\endgraf\vskip 0.5\baselineskip}%
			\hss}}}
\makeatother


%Более гибкие спсики
\usepackage{enumitem}
% сообщаем окружению о том, что существует такая штук как нумерация русскими буквами.
\makeatletter
\AddEnumerateCounter{\asbuk}{\russian@alph}{щ}
\makeatother


%%% ГОСТОВСКИЕ СПИСКИ

% Первый тип списков. Большая буква. 
\newlist{Enumerate}{enumerate}{1}
\setlist[Enumerate,1]{labelsep=0.5em,leftmargin=1.25em,labelwidth=1.25em,
	parsep=0em,itemsep=0em,topsep=0.75ex, before={\parskip=-1em},label=\arabic{Enumeratei}.}


% Второй тип списков. Маленькая буква.
\setlist[enumerate]{label=\arabic{enumi}),parsep=0em,itemsep=0em,topsep=0.75ex, before={\parskip=-1em}}


% Третий тип списков. Два уровня. 
\newlist{twoenumerate}{enumerate}{2}
\setlist[twoenumerate,1]{itemsep=0mm,parsep=0em,topsep=0.75ex,, before={\parskip=-1em},label=\asbuk{twoenumeratei})}
\setlist[twoenumerate,2]{leftmargin=1.3em,itemsep=0mm,parsep=0em,topsep=0ex, before={\parskip=-1em},label=\arabic{twoenumerateii})}


% Четвёртый тип списков. Список с тире.
\setlist[itemize]{label=--,parsep=0em,itemsep=0em,topsep=0.75ex, before={\parskip=-1em},after={\parskip=-1em}}


%%% WARNING WARNING WARNIN!
%%% Если в списке предложения, то должна по госту стоять точка после цифры => команда Enumerate! Если идет перечень маленьких фактов, не обособляемых предложений то после цифры идет скобка ")" => команда enumerate! Если перечень при этом ещё и двууровневый, то twoenumerate.




%%%%%%%%%% Список литературы %%%%%%%%%%

%\usepackage[% 
%backend=biber, %подключение пакета biber (тоже нужен)
%bibstyle=gost-numeric, %подключение одного из четырех главных стилей biblatex-gost 
%sorting=ntvy, %тип сортировки в библиографии
%]{biblatex}

\usepackage[backend=biber,style=gost-numeric]{biblatex}

% Справка по 4 главным стилям для ленивых: 
% gost-inline  ссылки внутри теста в круглых скобках
% gost-footnote подстрочные ссылки
% gost-numeric затекстовые ссылки
% gost-authoryear тоже затекстовые ссылки, но немного другие 

% Подробнее смотри страницу 4 документации. Она на русском. 

% Ещё немного настроек
\DeclareFieldFormat{postnote}{#1} %убирает с. и p.
\renewcommand*{\mkgostheading}[1]{#1} % только лишь убираем курсив с авторов


\addbibresource{document.bib} % сюда нужно вписать свой биб-файлик

% Этот кусок кода выносит русские источники на первое место. Костыль описали авторы пакета в руководстве к нему. Подробнее смотри: 
% https://github.com/odomanov/biblatex-gost/wiki/Как-сделать%2C-чтобы-русскоязычные-источники-предшествовали-остальным
\DeclareSourcemap{
	\maps[datatype=bibtex]{
		\map{
			\step[fieldsource=langid, match=russian, final]
			\step[fieldset=presort, fieldvalue={a}]
		}
		\map{
			\step[fieldsource=langid, notmatch=russian, final]
			\step[fieldset=presort, fieldvalue={z}]
		}
	}
}




\begin{document}


\thispagestyle{empty} % Чтобы избежать нумерации титульника

% Если для какой-то страницы хочется сделать своё уникальное оформление, как например для титульника или списка литературы, то можно использовать окружение \begingroup ... \endgroup. 

\begingroup
\setstretch{1}  
\begin{center}
	\small \bfseries Федеральное государственное бюджетное образовательное учреждение высшего образования
	
	<<РОССИЙСКАЯ АКАДЕМИЯ НАРОДНОГО ХОЗЯЙСТВА и\\ ГОСУДАРСТВЕННОЙ СЛУЖБЫ \\
	при Президенте Российской Федерации>>
	
	\vspace{2ex}
	
	\bfseries
	ЭКОНОМИЧЕСКИЙ ФАКУЛЬТЕТ
	
	НАПРАВЛЕНИЕ 38.03.01 ЭКОНОМИКА
\end{center}

\vfill


\noindent\small Группа ЭФ-15-02
\hfill
\parbox[t]{20em}{\centering\small
	Кафедра <<Макроэкономики>>
	
	\mbox{ }
	
	\textbf{Допустить к защите}
	
	заведующий кафедрой <<Макроэкономики>>
	
	\mbox{ }
	
	\rule{8em}{0.5pt} Н.Л. Шагас
	
	\mbox{ }
	
	<<\rule{2em}{0.5pt}>> \rule{5em}{0.5pt} 201\rule{1em}{0.5pt} г. }

\mbox{ }

\mbox{ }

\begin{center}\bfseries
	НАУЧНО-ИССЛЕДОВАТЕЛЬСКАЯ РАБОТА
	
	\mbox{ }
	
	\large
	ИССЛЕДОВАНИЕ СВЯЗИ ТЕМПОВ ЭКОНОМИЧЕСКОГО РОСТА\\
	И СТРУКТУРЫ ЭКСПОРТА В СТРАНАХ,\\
	НАДЕЛЕННЫХ ПРИРОДНЫМИ РЕСУРСАМИ
\end{center}

\vfill

\noindent\normalsize
студент-бакалавр

\noindent
Найденович Анна Анатольевна
\hfill /\rule{6em}{0.5pt}/\rule{6em}{0.5pt}/

\hfill\makebox[13em]{\hfill\footnotesize (подпись) \hfill\hfill (дата) \hfill}

\noindent
научный руководитель выпускной \\
квалификационной работы

\noindent
к.э.н., доцент Перевышин Юрий Николаевич
\hfill /\rule{6em}{0.5pt}/\rule{6em}{0.5pt}/

\hfill\makebox[13em]{\hfill\footnotesize (подпись) \hfill\hfill (дата) \hfill}

%\noindent
%консультант
%
%\noindent
%д.э.н., профессор Петров Петр Петрович
%\hfill /\rule{6em}{0.5pt}/\rule{6em}{0.5pt}/
%
%\hfill\makebox[13em]{\hfill\footnotesize (подпись) \hfill\hfill (дата) \hfill}

\vfill

\begin{center}
	\normalsize \bfseries МОСКВА \\ 2017 г.
\end{center}
\endgroup 


\tableofcontents  % Команда, которая создаёт оглавление

\chapter*{Введение}
%Включение введения в соодержание
\addcontentsline{toc}{chapter}{Введение}

Так как очень недолго пишу НИР, введения пока никакого нет. Но предположим, что оно здесь когда-нибудь появится.


\chapter{Теоретические исследования влияния наделенности природными ресурсами на экономический рост}

В статье «Natural Resources and Economic Growth: The Role of Investment» Торвальдур Гилфсон и Гилфи Зоега (Thorvaldur Gylfason, Gylfi Zoega) исследуют связь между наделенностью природными ресурсами и экономическим ростом. Авторы утверждают, что опора только на природные ресурсы может негативно повлиять на показатели сбережений, инвестиций и экономического роста, что приведет к снижению потребления и дохода на душу населения в долгосрочной перспективе. 

Есть множество примеров стран, богатых природными ресурсами, которые демонстрируют низкие темпы экономического роста, в то время как страны с небольшими запасами природных ресурсов, процветают. Хороший пример – это Ботсвана и Сьерра-Леоне. Обе страны занимаются экспортом алмазов. Ботсване удалось направить выручку от экспорта драгоценных камней таким образом, что это привело к впечатляющим темпам экономического роста, у этой страны был зафиксирован самый высокий прирост ВВП в период с 1965-го по 1998-й годы. В то же время Сьерра-Леоне осталась погруженной в бедность и междоусобные войны местных полевых командиров, которые сражались за контроль над добычей полезных ископаемых (Olsson, 2006). В 1998-м году, согласно данным Всемирного Банка (2000), Сьерра-Леоне была беднейшей страной в мире. 

Авторы считают, что природные ресурсы являются экзогенным фактором, который препятствует экономическому росту через макроэкономические каналы, а также через институты. Эта гипотеза отличается от утверждений Асемоглу, Джонсона и Робинсона (2001), которые предположили, что именно условия жизни в колониальные времена определяли, примут ли европейцы решение поселиться в данной местности и привнести европейские институты. В соответствии с этой альтернативной гипотезой, экономический успех зиждется на текущих институтах, которые зависят от прежних институтов и определяются жизненными условиями в прошлом - то есть процентом смертности поселенцев в колониальные времена. Гилфсон и Зоега утверждают, что зависимость от природных ресурсов в настоящее время влияет на существующие институты, а также на макроэкономические показатели.

Авторы рассматривают данные по 85 странам с 1965 по 1998 гг. Зависимость от природных ресурсов они измеряют как долю природного капитала в национальном богатстве в 1994 г. - то есть долю природного капитала в совокупном капитале, который включает физический, человеческий и природный капитал. Согласно данным, хорошие показатели роста несовместимы с долей природных ресурсов, превышающей 15~\% национального богатства. 

Авторы обнаружили две отличающихся группы стран. Первая группа состоит из восьми африканских стран (Центральная Африканская Республика, Чад, Гвинея-Бисау, Мадагаскар, Мали, Нигер, Сьерра-Леоне и Замбия) - все с большими запасами природных ресурсов, составляющих более четверти национального богатства, и демонстрирующих низкие показатели экономического роста на душу населения, начиная с 1965-го года. Другая группа также состоит из 8 стран, по большей части азиатских, независимых от природных ресурсов, но чьи экономики росли быстро с 1965-го года (Ботсвана, Китай, Индонезия, Япония, Корея, Малайзия, Маврикий и Таиланд). Остальные 69 стран распределяются между этими двумя экстремумами. 

Что же отличает 8 быстрорастущих экономик во второй группе от 8 отстающих в первой? Эмпирические наблюдения показывают, что ключевой индикатор - это уровни сбережений и инвестиций. В частности, группа зависимых от природных ресурсов стран имеют средний уровень сбережений в размере 5~\% от ВВП (от 2~\% в Гвинее-Бисау до 19~\% в Замбии), в то время как независимые от природных ресурсов страны показывают средний уровень сбережений в размере 32~\% от ВВП со значениями от 28~\% до 35~\%. Аналогичная взаимосвязь возникает, если заменить валовые внутренние сбережения валовыми внутренними инвестициями. В данном случае, первая группа показывает средний уровень инвестиций в 14~\% (от 7~\% в Чаде до 29~\% в Гвинее-Бисау), тогда как вторая группа демонстрирует показатель в 28~\%, который колеблется в конкретных случаях от 26~\% до 31~\%. Таким образом, в данном исследовании авторы акцентируют внимание на сбережениях и инвестициях.

Гилфсон и Зоега в своей работе опираются на более ранние исследования и рассматривают различные каналы влияния изобилия природных ресурсов на экономический рост. Так как природные ресурсы – это фиксированный фактор производства, то они налагают ограничение на потенциал экономического роста. Это ограничение может – в зависимости от характера технологии производства – привести к тому, что рост рабочей силы и запаса капитала будет происходить с убывающей отдачей. Это первая причина неблагоприятного воздействия природных ресурсов на экономический рост. 

Во-вторых, огромные доходы от ресурсной ренты могут создать предпосылки для рентоориентированного поведения со стороны производителей, тем самым отвлекая ресурсы из наиболее социально плодотворной экономической деятельности (Auty, 2001; and Gelb, 1988). Торнелл и Лэйн (Tornell and Lane, 1998) показывают, что сверхдоходы и ресурсный бум могут спровоцировать политическое взаимодействие среди влиятельных заинтересованных групп, которые приводят к дефициту текущего счета, непропорциональному перераспределению бюджетных средств и снижению темпов роста. В крайнем случае, вспыхивают гражданские войны – такие как алмазные войны в Африке – которые не только отвлекают факторы производства из продуктивных сфер, но и разрушают социальные институты и верховенство закона. Коллиер и Хоффлер (Collier and Hoeffler, 1998) доказали опытным путем, что природные ресурсы увеличивают вероятность гражданской войны. Военные расходы, в свою очередь, тормозят рост за счет своих побочных эффектов, влияющих на формирование капитала и распределение ресурсов (Knight et al., 1996). Избыток доходов от природных ресурсов особенно рискован, когда эти ресурсы могут быть извлечены на узкой географической или экономической базе (например, нефть и полезные ископаемые), которая может быть легко захвачена (Isham et al., 2005; Mehlum et al., 2006).

В-третьих, изобилие природных ресурсов может привести к голландской болезни. Рост запасов природных ресурсов связан с ростом сырьевого экспорта, что может вести к росту реального обменного курса валюты, таким образом, возможно снижение производства и экспорта услуг (Corden, 1984). Голландская болезнь может поразить также страны, которые не имеют собственной валюты (например, Гренландия, в которой ходит датская крона; Paldam, 1997). Голландская болезнь может медленно снижать экономический рост, препятствуя производству и экспорту услуг, которые способствуют экономическому росту (Frankel and Romer, 1999) – причем не только росту их количества, но также разнообразия и качества.

В-четвертых, изобилие природных ресурсов может ослабить стимулы частного и государственного секторов к накоплению человеческого капитала вследствие высокого уровня не связанных с заработной платой доходов. Эмпирические данные свидетельствуют о том, что среди стран число учащихся на всех уровнях обратно пропорционально зависимости от природных ресурсов, измеряемой долей рабочей силы, занятой в первичном производстве (Гилфсон и соавт., 1999). Существуют также доказательства того, что государственные расходы на образование по отношению к национальному доходу, среднее количество лет обучения в школах и отношение учеников средней школы к ученикам начальной - все эти показатели обратно пропорциональны доле природного капитала в национальном богатстве (Гилфсон, 2001). 

Наконец, избыток природных ресурсов может вселить в людей ложное ощущение безопасности и побудить правительства упускать из виду потребность в накоплении человеческого капитала, а также в управлении, ориентированном на качественный экономический рост, в том числе контроле за такими факторами, как свободная торговля, бюрократическая эффективность, качество институтов и устойчивое развитие (Rodriguez and Sachs, 1999; Sachs and Warner, 1999). Иначе говоря, избыток природного капитала может вытеснять социальный и человеческий капитал (Paldam and Svendsen, 2000; and Woolcock, 1998). 

Ранее авторы сделали эмпирическое наблюдение, что различия в сбережениях и инвестициях отличают богатые природными ресурсами страны с низким экономическим ростом и бедные ресурсами страны с высокими темпами роста в выборке. Авторы останавливаются на двух возможных объяснениях этой связи. 

Во-первых, механизмы, о которых говорилось ранее – рентоориентированный подход, голландская болезнь, пренебрежение образованием – все это может быть причиной сокращения сбережений и инвестиций. Во-вторых, низкое качество институтов может привести к низким темпам экономического роста. Проблема эндогенности может быть решена либо утверждением, что природные ресурсы могут влиять на институциональную среду, или же утверждением, что текущие низкие темпы экономического роста стран являются результатом институционального наследия: неблагоприятные условия жизни в колониальные времена снижали приток европейцев в колонии и помешали развитию западных институтов. Следуя данному тезису, бывшие колонии по-прежнему страдают от низкого качества институтов, которое сдерживает сбережения, инвестиции и экономический рост. В этом случае проблема не в чрезмерной зависимости от природных ресурсов в настоящее время, а в истории развития институтов. 

Далее авторы проверяют гипотезу о том, что ресурсная зависимость негативно влияет на инвестиции и экономический рост в дополнение к механизмам, рассмотренным ранее, в предположении, что слабые институты тормозят экономический рост. Авторы оценивают следующие уравнения:
\begin{align} 
	g&=f_1(y_0,h,i,b,n,c,p) \label{eq1} \\
	h&=f_2(y_0,b,n,c) \label{eq2} \\
	i&=f_3(y_0,b,n,c) \label{eq3} \\
	c&=f_4(y_0,b,n) \label{eq4}
\end{align}

Темпы экономического роста на душу населения g являются убывающей функцией от начального дохода на душу населения $y_0$, а также от переменной зависимости от природных ресурсов $b$ и роста населения $p$, и возрастающей функцией образования $h$, уровня инвестиций $i$, изобилия природных ресурсов $n$ и набора соответствующих институтов $с$, который измеряется индексом гражданских свобод. Далее, авторы полагают, что инвестиции и образование – убывающие функции от переменной зависимости от природных ресурсов и возрастающие - от изобилия природных ресурсов, а также степени гражданских свобод. Образование и инвестиции также зависят от начального дохода. Авторы полагают, что гражданские свободы - функция от трех экзогенных переменных - первоначального дохода и двух переменных, описывающих природные ресурсы. Авторы обращают внимание на рекурсивность четырех уравнений модели. Гражданские свободы зависят исключительно от описанных экзогенных переменных в соответствии с уравнением \ref{eq4}, так же, как и инвестиции, и образование в уравнениях \ref{eq2} и \ref{eq3}, и, следовательно, экономический рост также зависит от экзогенных переменных в уравнении \ref{eq1}.

Авторы хотят оценить влияние ресурсной зависимости, измеренной в виде доли природных ресурсов в структуре национального богатства, и изобилия природных ресурсов, измеренного в виде природного капитала на человека, на уровень инвестиций и экономический рост. Кроме того, они хотят проверить, какое влияние оказывает включение в модель гражданских свобод – аппроксимации качества институтов.

Авторы выявили прямое влияние доли природного капитала на экономический рост, а также косвенное влияние через образование и инвестиции. Эта связь – от сильной зависимости от природных ресурсов до замедления экономического роста ввиду недостатка инвестиции – не была ранее описана в эконометрических работах. Кроме того, результаты указывают на дополнительное косвенное влияние доли природного капитала на рост за счет гражданских свобод. Авторы отмечают, что увеличение гражданских свобод стимулирует рост напрямую, а также косвенно, стимулируя улучшение образования и, возможно, также инвестиций, даже если последний эффект является статистически незначимым. Эти факторы влияния представляются довольно важными. Например, увеличение гражданских свобод от уровня Турции или Уругвая до, скажем, Швейцарии и Великобритании соответствует увеличению экономического роста в расчете на душу населения на один процентный пункт.

Кроме того, увеличение доли природного капитала на душу населения положительно влияет на рост, образование и инвестиции. При этом суммарный эффект увеличения доли природного капитала на экономический рост сокращается с увеличением благосостояния в расчете на душу населения.

\chapter*{Заключение}
\addcontentsline{toc}{chapter}{Заключение}

PS: сделаем вид, что это заключение по научной работе, а не по одной статье, так как заключения, как и введения, еще нет.

В этой статье авторы предложили взаимосвязь между обеспеченностью природными ресурсами и экономическим ростом через сбережения и инвестиции. Они получили следующие результаты:
\begin{Enumerate}
	\item Накопление физического капитала посредством инвестиций, человеческого капитала посредством среднего образования и социального капитала через гражданские свободы обратно пропорционально доле природного капитала в национальном богатстве.
	\item Экономический рост обратно пропорционален зависимости от природных ресурсов и первоначальному доходу и прямо пропорционален уровню образования, инвестиций и гражданских свобод.
	\item Даже если зависимость от природных ресурсов негативно влияет на инвестиции, образование, гражданские свободы и экономический рост, изобилие природных ресурсов, измеренное ресурсами в расчете на душу населения, прямо пропорционально уровню инвестиций, образования, гражданским свободам и темпам экономического роста.
	\item Влияние природных ресурсов на инвестиции, образование и рост, задокументированное в более ранних исследованиях, сохраняет свою значимость при введении переменной гражданских свобод. Влияние природных ресурсов проявляется в виде опосредованного влияния на инвестиции, образование и рост за счет качества институтов. 
\end{Enumerate}

Таким образом, переменная природных ресурсов сохраняет свою значимость при введении институциональных переменных. Аналогично, институциональные переменные сохраняют свое влияние независимо от наличия переменной природных ресурсов. Авторы делают вывод, что влияние природных ресурсов может проявляться как напрямую через макроэкономические переменные, так и опосредованно через институты.


\nocite{*}  %Чтобы в список литературы напечаталичь все источники из bib-файла

\begingroup
\setstretch{1}
\addcontentsline{toc}{chapter}{Список литературы}
\printbibliography[title = Список использованных источников]
\endgroup

\appendix
\renewcommand{\thechapter}{\Asbuk{chapter}}

%%%%%%%%%% titlesec для приложений
\titleformat{\chapter}
{\normalfont\bfseries\large}{\chaptertitlename~\thechapter}{0.25em}{\normalfont}


\titlecontents{chapter}
[0 em] % 
{\normalsize}
{\makebox[7em][l]{Приложение \thecontentslabel}}
{Приложение }
{\titlerule*[10pt]{.}\contentspage}


\chapter[Таблица с "анализом" эмпирических статей]{Таблица с "анализом" эмпирических статей}\label{app-a}

\begin{center}
	\begin{longtable}{|l|l|l|}
		\caption{Механизм и результат влияния природных ресурсов на экономический рост}\\
		\hline
		\textbf{Статья} & \textbf{Механизм} & \textbf{Влияние ресурсов} \\ \hline 
		\endfirsthead
		
		\multicolumn{3}{c}{\tablename{} \thetable{}: продолжение } \\
		\hline 
		\textbf{Статья}&\textbf{Механизм} & \textbf{Влияние ресурсов} \\
		\hline 
		\endhead
		
		\hline
		\multicolumn{3}{|c|}{Продолжение на следующей странице} \\ \hline
		\endfoot
		\hline
		\endlastfoot
		
		
		Sachs, Warner (1995) & Голландская болезнь & Отрицательное \\
		Mehlum, Moene, Torvik (2005) & Борьба за ресурсную ренту & Пороговое \\
		Tornell, Lane (1996) & Эффект прожорливости & Неоднозначное \\
		Gylfason (2001) & Влияние на человеческий капитал & Отрицательное \\
		
	\end{longtable}
\end{center}


\chapter[Картинка]{Картинка}

\begin{figure}[H]
	\includegraphics[width=1\linewidth]{ranepa}
	\caption{П-патриотизм}
\end{figure}


\end{document}