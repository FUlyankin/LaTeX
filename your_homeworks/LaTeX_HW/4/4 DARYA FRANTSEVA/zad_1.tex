%!TEX TS-program = xelatex
\documentclass[12pt, a4paper]{article}

%%%%%%%%%% Програмный код %%%%%%%%%%
\usepackage{minted}

%%%%%%%%%% Математика %%%%%%%%%%
\usepackage{amsmath,amsfonts,amssymb,amsthm,mathtools}

%%%%%%%%%%%%%%%%%%%%%%%% Шрифты %%%%%%%%%%%%%%%%%%%%%%%%%%%%%%%%%
\usepackage{fontspec}         % пакет для подгрузки шрифтов
\setmainfont{Arial}   % задаёт основной шрифт документа

\newfontfamily{\cyrillicfonttt}{Arial}
\newfontfamily{\cyrillicfont}{Arial}
\newfontfamily{\cyrillicfontsf}{Arial}

\usepackage{unicode-math}     % пакет для установки математического шрифта
\setmathfont{Asana-Math.otf}      % шрифт для математики

\usepackage{polyglossia}      % Пакет, который позволяет подгружать русские буквы
\setdefaultlanguage{russian}  % Основной язык документа
\setotherlanguage{english}    % Второстепенный язык документа

\title{Задание 1.}
\date{\today}
\begin{document}
\maketitle
\section{Игра в 21 очко}
\begin{minted}[breaklines,linenos]{python}
koloda = [6,7,8,9,10,2,3,4,11] * 4
import random
random.shuffle(koloda)
count = 0
while True:
    choice = input('Будете брать карту? y/n\n')
    if choice == 'y':
        current = koloda.pop()
        print('Вам попалась карта достоинством %d' %current)
        count += current
        if count > 21:
            print('Вы проиграли')
            break
        elif count == 21:
            print('Вы набрали 21!')
            break
        else:
            print('У вас %d очков.' %count)
    elif choice == 'n':
        print('У вас %d очков и вы закончили игру.' %count)
        break
\end{minted}
\end{document}