\documentclass[12pt, a4paper]{article}


%%%%%%%%%% Програмный код %%%%%%%%%%
\usepackage{minted}

%%%%%%%%%%%%%%%%%%%%%%%% Шрифты %%%%%%%%%%%%%%%%%%%%%%%%%%%%%%%%%
\usepackage{fontspec}         % пакет для подгрузки шрифтов
\setmainfont{Arial}   % задаёт основной шрифт документа
\usepackage{tikzsymbols}
\usepackage{hyperref}
\hypersetup{
    unicode=true,           
    colorlinks=true,       	
    urlcolor=blue}

\renewcommand{\thesection}{\Smiley{ }}
\usepackage{minted} 
\title{Код}
\date{\today}
\author{Жильцова Алиса}


\begin{document} 

\maketitle

\section{Не мой код}

 Он взят с \href{http://www.r-graph-gallery.com/}{этого крутого сайта} и намного интереснее моих :)
\begin{minted}[breaklines,linenos]{R}
# generate pairs of x-y values
theta = seq(-2 * pi, 2 * pi, length = 300)
x = cos(theta)
y = x + sin(theta) 
 
# set graphical parameters
op = par(bg = "black", mar = rep(0.1, 4))
 
# plot
plot(x, y, type = "n", xlim = c(-8, 8), ylim = c(-1.5, 1.5))
for (i in seq(-2*pi, 2*pi, length = 100))
{
  lines(i*x, y, col = hsv(runif(1, 0.85, 0.95), 1, 1, runif(1, 0.2, 0.5)), 
        lwd = sample(seq(.5, 3, length = 10), 1))          
}
 
# signature
legend("bottomright", legend = "© Gaston Sanchez", bty = "n", text.col = "gray70")
 
\end{minted}
\newpage
\section{Что осталось с певрого курса}
\begin{minted}[breaklines,linenos]{R}
bubble<-function(A){
  for(i in (length(A)-1):1){
    for(j in 1:i){
      if(A[j]>A[j+1]){b<-A[j]
                      A[j]<-A[j+1]
                      A[j+1]<-b}
    }
  }
return(A)
}
A<-c(9,8,5,4,8,1)
print(bubble(A))

func2<-function(A){
  for(i in length(A):2){
    key<-1
    for(j in 1:i){
      if(A[j]>A[key]){key<-j}
    }
  if(key<j){b<-A[j]
            A[j]<-A[key]
             A[key]<-b}
  }
  return(A)
}
print(func2(A))
\end{minted}
\end{document}