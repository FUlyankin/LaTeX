%!TEX TS-program = xelatex
\documentclass[12pt, a4paper]{article}


%%%%%%%%%% Математика %%%%%%%%%%
\usepackage{amsmath,amsfonts,amssymb,amsthm,mathtools}
%\mathtoolsset{showonlyrefs=true}  % Показывать номера только у тех формул, на которые есть \eqref{} в тексте.
%\usepackage{leqno} % Нумерация формул слева

\pagenumbering{gobble} 

%%%%%%%%%%%%%%%%%%%%%%%% Шрифты %%%%%%%%%%%%%%%%%%%%%%%%%%%%%%%%%
\usepackage{fontspec}         % пакет для подгрузки шрифтов
\setmainfont{Arial}   % задаёт основной шрифт документа

% Команда, которая нужна для корректного отображения длинных тире и некоторых других символов.
\defaultfontfeatures{Mapping=tex-text}

% why do we need \newfontfamily:
% http://tex.stackexchange.com/questions/91507/
\newfontfamily{\cyrillicfonttt}{Arial}
\newfontfamily{\cyrillicfont}{Arial}
\newfontfamily{\cyrillicfontsf}{Arial}

% готический шрифт для заголовка 
\newfontfamily\myfont{Deutsch Gothic}

\usepackage{unicode-math}     % пакет для установки математического шрифта

\usepackage{polyglossia}      % Пакет, который позволяет подгружать русские буквы
\setdefaultlanguage{russian}  % Основной язык документа
\setotherlanguage{english}    % Второстепенный язык документа


%%%%%%%%%% Работа с картинками %%%%%%%%%
\usepackage{graphicx}                  % Для вставки рисунков
\usepackage{graphics}
\graphicspath{{images/}{pictures/}}    % можно указать папки с картинками
\usepackage{wrapfig}                   % Обтекание рисунков и таблиц текстом




%%%%%%%%%% Графика и рисование %%%%%%%%%%
\usepackage{tikz, pgfplots}  % язык для рисования графики из latex'a


\usepackage{indentfirst} % установка отступа в первом абзаце главы!!!

% Разные мелочи для русского языка из пакета babel
\setkeys{russian}{babelshorthands=true}

\usepackage[charter]{mathdesign}

\usepackage{psvectorian}

\usepackage[top=2cm, bottom=2cm, outer=0cm, inner=0cm]{geometry}
\begin{document}
	\newenvironment{gfont}{\myfont}{\par}
	
\begin{gfont}
	\begin{center}
\Huge \rput[r](-4pt,6pt){\psvectorian[color=black,height=1.5cm]{21}}
\space \space Контракт \rput[l](-0.5,5pt){\psvectorian[color=black,height=1.5cm]{23}} 
	\end{center}
\end{gfont}
	
\vspace{8cm}

\begin{center}
\Large	Я согласна выполнить эту работу.
\end{center}

\vspace{5cm}



\begin{center}
	
\begin{figure}[h] 
	\centering
	\includegraphics[scale=0.6]{blood.jpg}
	 
\end{figure}

		\Large Место для крови \\
		\vspace{2cm}
		\noindent\rule{8cm}{0.4pt} \\
		\Large Подпись
\end{center}


\tikz[remember picture,overlay] \node[opacity=0.3,inner sep=0pt] at (current page.center){\includegraphics[width=\paperwidth,height=\paperheight]{old-paper.jpg}};

\usetikzlibrary{calc}
\begin{tikzpicture}[remember picture, overlay]
\draw[line width = 4pt] ($(current page.north west) + (1cm,-1cm)$) rectangle ($(current page.south east) + (-1cm,1cm)$);
\end{tikzpicture}

\end{document}