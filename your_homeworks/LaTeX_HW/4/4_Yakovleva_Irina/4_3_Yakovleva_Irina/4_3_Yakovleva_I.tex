%!TEX TS-program = xelatex
\documentclass[14pt,a4paper]{report}
%%%%%%%%%%%%%%%%%%%%%%%% Шрифты %%%%%%%%%%%%%%%%%%%%%%%%%%%%%%%%%
\usepackage{fontspec}         % пакет для подгрузки шрифтов
\setmainfont{Arial}   % задаёт основной шрифт документа
% Команда, которая нужна для корректного отображения длинных тире и некоторых других символов.
\defaultfontfeatures{Mapping=tex-text}
\newfontfamily{\cyrillicfonttt}{Arial}
\newfontfamily{\cyrillicfont}{Arial}
\newfontfamily{\cyrillicfontsf}{Arial}

\usepackage{polyglossia}      % Пакет, который позволяет подгружать русские буквы
\setdefaultlanguage{russian}  % Основной язык документа
\setotherlanguage{english}    % Второстепенный язык документа
% Разные мелочи для русского языка из пакета babel
\setkeys{russian}{babelshorthands=true}
%%%%%%%%%% Работа с картинками %%%%%%%%%
\usepackage{graphicx}                  % Для вставки рисунков
\usepackage{graphics}


\usepackage{hyperref}
\hypersetup{
    unicode=true,           % позволяет использовать юникодные символы
    colorlinks=true,       	% true - цветные ссылки, false - ссылки в рамках
    urlcolor=blue,       % на библиографию
	pdfnewwindow=true,      % при щелчке в pdf на ссылку откроется новый pdf
	breaklinks              % если ссылка не умещается в одну строку, разбивать ли ее на две части?
}

%%%% Оформление %%%%%%%
\usepackage{extsizes} % Возможность сделать 14-й шрифт

% размер листа бумаги
\usepackage[paper=a4paper,top=10mm, bottom=10mm,left=35mm,right=35mm,includefoot,includehead]{geometry}


\usepackage{setspace}
\setstretch{1}  % Межстрочный интервал
\setlength{\parindent}{0mm} % Красная строка.
\setlength{\parskip}{4mm}   % Расстояние между абзацами
% Разные длины в латехе https://en.wikibooks.org/wiki/LaTeX/Lengths
\usepackage{fancyhdr} % Колонтитулы
\pagestyle{fancy}
\renewcommand{\headrulewidth}{0.2pt}  % Толщина линий, отчеркивающих верхний
\renewcommand{\footrulewidth}{0.2pt}  % и нижний колонтитулы
\cfoot{Школа Чародейства и Волшебства <<Хогвартс>>}
\rhead{\thepage}
 
\flushbottom                            % Эта команда заставляет LaTeX чуть растягивать строки, чтобы получить идеально прямоугольную страницу
\righthyphenmin=2                       % Разрешение переноса двух и более символов



\newcommand*{\MyPoint}{\includegraphics[height=7mm]{w.png}}
\renewcommand{\labelitemi}{\MyPoint}
\renewcommand{\thesection}{Приложение \Asbuk{section}}
\setcounter{section}{0}
\begin{document}
\thispagestyle{empty}
\begin{center}
\includegraphics[height=6cm]{h.jpg}
\end{center}

\vspace{1cm}

{ \fontsize{12}{1}\selectfont Мистеру Поттеру}

\vspace{2.5cm}
Дорогой мистер Поттер!

Мы рады проиинформироать Вас, что Вам предоставлено место в~Школе~Чародейтва~и~Волшебства~<<Хогвартс>>.

Пожалуйста,ознакомтесь с приложением к данному письму (списком необходимых книг и предметов).

Занятия начинаются 1 сентября.

Искрене Ваша,

{\fontspec{SANTOS DUMONT}{Minevra MacGonagall}}

Минерва МакГонагалл \\
заместитель директора

\vfill

\begin{center}
Школа~Чародейтва~и~Волшебства~<<Хогвартс>> \\
{ \fontsize{10}{1}\selectfont Директор: Альбус Дамблдор}\\{\fontsize{10}{1}\selectfont (Кавалер магических орденов)}\\
\href{http://economy.ranepa.ru/}{Cайт школы}
\end{center}


\newpage
\section{Список необходимых книг и предметов}
\begin{itemize}
\item Мантия
\item Волшебная палочка
\item Микроэкономика Вэриан
\item Задакник Демидовича
\end{itemize}


\newpage
\section{Список изучаемых предметов}
\begin{itemize}
\item Математический анализ
\item История Магии
\item Микроэкономика
\end{itemize}
\end{document}