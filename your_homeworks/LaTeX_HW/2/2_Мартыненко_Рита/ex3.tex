%!TEX TS-program = xelatex
\documentclass[12pt, a4paper]{article}  

%%%%%%%%%% Математика %%%%%%%%%%
\usepackage{amsmath,amsfonts,amssymb,amsthm,mathtools} 
%\mathtoolsset{showonlyrefs=true}  % Показывать номера только у тех формул, на которые есть \eqref{} в тексте.
%\usepackage{leqno} % Нумерация формул слева


%%%%%%%%%%%%%%% Шрифты %%%%%%%%%%%
\usepackage{fontspec}         % пакет для подгрузки шрифтов
\setmainfont{Arial}   % задаёт основной шрифт документа

\defaultfontfeatures{Mapping=tex-text}

% why do we need \newfontfamily:
% http://tex.stackexchange.com/questions/91507/
\newfontfamily{\cyrillicfonttt}{Arial}
\newfontfamily{\cyrillicfont}{Arial}
\newfontfamily{\cyrillicfontsf}{Arial}
% Грубо говоря иногда polyglossia начинает балбесничать и не видит структуры кириллических шрифтов. Эти трое бравых парней спасают ситуацию и доопределяют те куски, которые Тех не увидел.

\usepackage{unicode-math}     % пакет для установки математического шрифта
\setmathfont{Phorssa}      % шрифт для математики
% \setmathfont[math-style=ISO]{Asana Math}
% Можно делать смену начертания с помощью разных стилей

% Конкретный символ из конкретного шрифта
% \setmathfont[range=\int]{Neo Euler}


\usepackage{polyglossia}      % Пакет, который позволяет подгружать русские буквы
\setdefaultlanguage{russian}  % Основной язык документа
\setotherlanguage{english}    % Второстепенный язык документа


\author{Мартыненко Маргарита} 
\title{Уютная домашка № 2}
\date{\today}

\begin{document}
\maketitle
\newpage
\section{Задание 3}
\subsection{Угроза}
\begin{center}
\fontspec{Phorssa}{Филипп! Завтра утром ты проснёшься, а LaTeX не существует и Word навсегда захватил планету... }
\end{center}
\subsection{Формулы}
 \begin{equation}  \int_{-\infty}^{+\infty} e^{-x^{2}} = \sqrt{\pi} \end{equation}
 \begin{equation}  \frac{1}{\pi} = \frac{2\sqrt{2}}{9801}\sum_{k=0}^{\infty} \frac{(4k)!(1103+26390k)}{(k!)^4 396^{4k}} \end{equation}
 \begin{equation}  \lim_{\alpha \to 0} \frac{\sin{\alpha}}{\alpha} = 1 \end{equation}
\begin{equation}  A^{-1}= \frac{1}{detA} \begin{pmatrix} 
 A_{1,1} & A_{2,1} & \cdots & A_{n,1} \\
  A_{1,2} & A_{2,2} & \cdots & A_{n,2} \\
  \vdots  & \vdots  & \ddots & \vdots  \\
  A_{1,n} & A_{2,n} & \cdots & A_{n,n}
\end{pmatrix} 
\end{equation}
 \begin{equation}  \sin{\alpha}^2 + \cos{\alpha}^2 =1   \end{equation}


\end{document}